 \documentclass[11pt]{article}
%\bibliographystyle{AEA/aer}
%\bibliographystyle{aer}
\usepackage[utf8]{inputenc}
\usepackage[T1]{fontenc}
\usepackage{pdfpages}
\usepackage{color,soul}
\usepackage{booktabs}

%%%%%%%%%%%%%%%%%%%%%%%%%%%%%%%% BIBLIOGRAPHY
\usepackage[threshold=0]{csquotes}
\usepackage[style= authoryear-icomp, 
            backref=true, 
            natbib=true, 
            backend = bibtex, 
            bibencoding=ascii,
            doi=false,
            isbn=false,
            url=false, 
            eprint = false]{biblatex}
\addbibresource{References.bib}

% AER-like bibliography
%(1) Authors in bold: tex.stackexchange.com/questions/178862/make-author-names-bold-in-bibliography-only
\DeclareNameFormat{last-first/first-last-bold}{\mkbibbold{%
  \ifnumequal{\value{listcount}}{1}
    {\iffirstinits
       {\usebibmacro{name:last-first}{#1}{#4}{#5}{#7}}
       {\usebibmacro{name:last-first}{#1}{#3}{#5}{#7}}%
     \ifblank{#3#5}
       {}
       {\usebibmacro{name:revsdelim}}}
    {\iffirstinits
       {\usebibmacro{name:first-last}{#1}{#4}{#5}{#7}}
       {\usebibmacro{name:first-last}{#1}{#3}{#5}{#7}}}%
  \usebibmacro{name:andothers}}}
\DeclareNameAlias{sortname}{last-first/first-last-bold}

% (2) Remove parenthesis in year: tex.stackexchange.com/questions/12254/biblatex-how-to-remove-the-parentheses-around-the-year-in-authoryear-style
\usepackage{xpatch}
\xpatchbibmacro{date+extrayear}{%
  \printtext[parens]%
}{%
  \setunit{\addperiod\space}%
  \printtext%
}{}{}

% (3) Remove "In:" from journal articles: tex.stackexchange.com/questions/10682/suppress-in-biblatex
\renewbibmacro{in:}{%
  \ifentrytype{article}{}{\printtext{\bibstring{in}\intitlepunct}}}

% (4) Parenthesis around volume number: tex.stackexchange.com/questions/81569/biblatex-parentheses-around-the-volume-number-of-an-article
\renewbibmacro*{volume+number+eid}{%
  \printfield{volume}%
%  \setunit*{\adddot}% DELETED
  \setunit*{\addnbthinspace}% NEW (optional); there's also \addnbthinspace
  \printfield{number}%
  \setunit{\addcomma\space}%
  \printfield{eid}}
\DeclareFieldFormat[article]{number}{\mkbibparens{#1}}
\renewcommand*{\bibpagespunct}{\addcolon\space}

% (5) Removing the pp. or p. tex.stackexchange.com/questions/12806/guidelines-for-customizing-biblatex-styles
\DeclareFieldFormat{pages}{#1}

% (6) Same author makes Author year1, year2


%%%%%%%%%%%%%%%%%%%%%%%%%%%%%%%% PACKAGES USED
\usepackage{graphicx}                       % loads images
\usepackage{color}                          % for color in text
\usepackage{float}                          % to place [H]
\usepackage{amsmath}
\usepackage{amssymb}
\usepackage{bbm}
\usepackage{booktabs}
\usepackage{longtable}
\usepackage{bigstrut}
\usepackage{array}

\usepackage{multirow}
\usepackage{xr}
\externaldocument{OA}




\graphicspath{{./Figuras/}}
\usepackage{color}
%\usepackage{morefloats}
%\usepackage{showkeys}
\usepackage[textwidth=1.55in]{todonotes}
\usepackage{rotating}
\usepackage{threeparttable}
\usepackage{lscape}
\usepackage{enumerate}



\usepackage{geometry} % Required to change the page size to A4
\geometry{letterpaper} % Set the page size to be A4 as opposed to the default US Letter
\geometry{margin=2.5cm}
\usepackage[labelfont=bf]{caption}
\usepackage[labelfont=bf, justification=justified]{subcaption}
\captionsetup{position=top}
\captionsetup[subfigure]{justification=centering}


\usepackage{setspace}
\onehalfspacing

\renewcommand{\rmdefault}{ppl}
\definecolor{darkred}{rgb}{0.8, 0.0, 0.0}
\definecolor{navyblue}{rgb}{0.0, 0.0, 0.8}
\definecolor{cadmiumgreen}{rgb}{0.0, 0.6, 0.24}

\usepackage{hyperref}
\hypersetup{
    pdfstartview={FitH},    		% fits the width of the page to the window
    pdftitle={Mitigating the risks of financial inclusion with loan contract terms},    % title
    pdfauthor={CJMS},     % author
    colorlinks=true,       % false: boxed links; true: colored links
    linkcolor=navyblue,          % color of internal links (change box color with linkbordercolor)
    citecolor=darkred,        % color of links to bibliography
    filecolor=magenta,      % color of file links
    urlcolor=darkred           % color of external links
}

\newcommand{\green}[1]{\textcolor{cadmiumgreen}{#1}}

\newcommand{\comment}[1]
{\par {\bfseries \color{blue} #1 \par}}
%\usepackage{showkeys}


\usepackage{tikz}
\usepackage{color} %red, green, blue, yellow, cyan, magenta, black, white
\usepackage{xcolor}
\usepackage{fullpage}
\usetikzlibrary{calc}
\usepackage{multirow,array}
\usepackage{longtable}

\newtheorem{theorem}{Theorem}
\newtheorem{lem}[theorem]{Lemma}
\newtheorem{dfn}{Definition}
\newtheorem{cor}[theorem]{Corollary}
\newtheorem{obs}{Obs}
\newtheorem{rem}{Remark}
\newtheorem{prob}{Problem}

%\usepackage{refcheck}
%%%%%%%%%%%%%%%%%%%%%%%%%%%%%%%% DOCUMENT
\begin{document}


\title{Preliminary results for: Overconfidence and settlement: evidence from Mexican labor courts\thanks{We would like to thank Sebastian Garcia, Isaac Meza, Diana Roman, and Monica Zamudio (``la banda pesada'' and ``los mas aca'') for superhuman research assistance.  All errors are ours.}}
\author{Joyce Sadka \and Enrique Seira  \and Christopher Woodruff }
\date{This draft: \today \\[2 cm]}

%\vspace{.5in}


\maketitle


\section{Preliminary results}



\begin{landscape}

\subsection{Raw Expectations}

 \begin{table}[H]
 \caption{Belief updating winsorizing at 95 percentile}
 \label{Belief_updating}
 \begin{center}
 \scriptsize{% Table generated by Excel2LaTeX from sheet 'exit_vs_initial_exp'
\begin{tabular}{rrrrrrrrrrrrr}
\toprule
      & \multicolumn{4}{c}{Employee}  & \multicolumn{4}{c}{Employee's Lawyer} & \multicolumn{4}{c}{Firm's Lawyer} \\
\midrule
      & \multicolumn{2}{c}{Amount} & \multicolumn{2}{c}{Time} & \multicolumn{2}{c}{Amount} & \multicolumn{2}{c}{Time} & \multicolumn{2}{c}{Amount} & \multicolumn{2}{c}{Time} \\
      & \multicolumn{1}{c}{(1)} & \multicolumn{1}{c}{(2)} & \multicolumn{1}{c}{(1)} & \multicolumn{1}{c}{(2)} & \multicolumn{1}{c}{(1)} & \multicolumn{1}{c}{(2)} & \multicolumn{1}{c}{(1)} & \multicolumn{1}{c}{(2)} & \multicolumn{1}{c}{(1)} & \multicolumn{1}{c}{(2)} & \multicolumn{1}{c}{(1)} & \multicolumn{1}{c}{(2)} \\
Initial survey var (ISV) & \multicolumn{1}{l}{0.461**} & \multicolumn{1}{l}{0.552***} & \multicolumn{1}{l}{0.476***} & \multicolumn{1}{l}{0.219} & \multicolumn{1}{l}{0.414***} & \multicolumn{1}{l}{0.360***} & \multicolumn{1}{l}{0.434***} & \multicolumn{1}{l}{0.247***} & \multicolumn{1}{l}{0.411***} & \multicolumn{1}{l}{0.298*} & \multicolumn{1}{l}{0.288**} & \multicolumn{1}{l}{0.100} \\
      & \multicolumn{1}{l}{(0.202)} & \multicolumn{1}{l}{(0.150)} & \multicolumn{1}{l}{(0.168)} & \multicolumn{1}{l}{(0.188)} & \multicolumn{1}{l}{(0.0794)} & \multicolumn{1}{l}{(0.0891)} & \multicolumn{1}{l}{(0.0982)} & \multicolumn{1}{l}{(0.0871)} & \multicolumn{1}{l}{(0.143)} & \multicolumn{1}{l}{(0.169)} & \multicolumn{1}{l}{(0.118)} & \multicolumn{1}{l}{(0.108)} \\
Control & \multicolumn{1}{l}{44429.0***} & \multicolumn{1}{l}{35769.9*} & \multicolumn{1}{l}{1.004**} & \multicolumn{1}{l}{1.721**} & \multicolumn{1}{l}{29065.5***} & \multicolumn{1}{l}{10042.7} & \multicolumn{1}{l}{1.431***} & \multicolumn{1}{l}{1.390**} & \multicolumn{1}{l}{16466.9***} & \multicolumn{1}{l}{9800.3} & \multicolumn{1}{l}{2.002***} & \multicolumn{1}{l}{2.017***} \\
      & \multicolumn{1}{l}{(10506.4)} & \multicolumn{1}{l}{(20775.0)} & \multicolumn{1}{l}{(0.435)} & \multicolumn{1}{l}{(0.717)} & \multicolumn{1}{l}{(6956.5)} & \multicolumn{1}{l}{(15624.8)} & \multicolumn{1}{l}{(0.389)} & \multicolumn{1}{l}{(0.611)} & \multicolumn{1}{l}{(4303.4)} & \multicolumn{1}{l}{(10418.6)} & \multicolumn{1}{l}{(0.462)} & \multicolumn{1}{l}{(0.559)} \\
Calculator & \multicolumn{1}{l}{-47533.2***} & \multicolumn{1}{l}{-29840.9***} & \multicolumn{1}{l}{-0.775} & \multicolumn{1}{l}{-1.170*} & \multicolumn{1}{l}{-5545.0} & \multicolumn{1}{l}{-6456.2} & \multicolumn{1}{l}{0.290} & \multicolumn{1}{l}{-0.158} & \multicolumn{1}{l}{-831.6} & \multicolumn{1}{l}{-5814.1} & \multicolumn{1}{l}{-0.0266} & \multicolumn{1}{l}{-0.450} \\
      & \multicolumn{1}{l}{(11321.0)} & \multicolumn{1}{l}{(10082.6)} & \multicolumn{1}{l}{(0.569)} & \multicolumn{1}{l}{(0.668)} & \multicolumn{1}{l}{(10009.1)} & \multicolumn{1}{l}{(8621.2)} & \multicolumn{1}{l}{(0.578)} & \multicolumn{1}{l}{(0.514)} & \multicolumn{1}{l}{(6450.5)} & \multicolumn{1}{l}{(6811.2)} & \multicolumn{1}{l}{(0.760)} & \multicolumn{1}{l}{(0.721)} \\
Conciliator & \multicolumn{1}{l}{-19182.5} & \multicolumn{1}{l}{-8996.6} & \multicolumn{1}{l}{0.0967} & \multicolumn{1}{l}{0.260} & \multicolumn{1}{l}{22560.3*} & \multicolumn{1}{l}{19202.8} & \multicolumn{1}{l}{-0.640} & \multicolumn{1}{l}{-0.149} & \multicolumn{1}{l}{501.5} & \multicolumn{1}{l}{-2721.4} & \multicolumn{1}{l}{-0.754} & \multicolumn{1}{l}{-0.638} \\
      & \multicolumn{1}{l}{(16642.8)} & \multicolumn{1}{l}{(16833.3)} & \multicolumn{1}{l}{(0.679)} & \multicolumn{1}{l}{(0.987)} & \multicolumn{1}{l}{(13536.7)} & \multicolumn{1}{l}{(11724.7)} & \multicolumn{1}{l}{(0.572)} & \multicolumn{1}{l}{(0.605)} & \multicolumn{1}{l}{(6192.3)} & \multicolumn{1}{l}{(7596.1)} & \multicolumn{1}{l}{(0.622)} & \multicolumn{1}{l}{(0.600)} \\
Calculator\#\#ISV & \multicolumn{1}{l}{0.406*} & \multicolumn{1}{l}{0.198} & \multicolumn{1}{l}{0.441**} & \multicolumn{1}{l}{0.612**} & \multicolumn{1}{l}{0.0757} & \multicolumn{1}{l}{0.0466} & \multicolumn{1}{l}{-0.0835} & \multicolumn{1}{l}{-0.00185} & \multicolumn{1}{l}{-0.178} & \multicolumn{1}{l}{-0.0774} & \multicolumn{1}{l}{0.0541} & \multicolumn{1}{l}{0.225} \\
      & \multicolumn{1}{l}{(0.210)} & \multicolumn{1}{l}{(0.175)} & \multicolumn{1}{l}{(0.198)} & \multicolumn{1}{l}{(0.233)} & \multicolumn{1}{l}{(0.113)} & \multicolumn{1}{l}{(0.121)} & \multicolumn{1}{l}{(0.137)} & \multicolumn{1}{l}{(0.121)} & \multicolumn{1}{l}{(0.206)} & \multicolumn{1}{l}{(0.232)} & \multicolumn{1}{l}{(0.180)} & \multicolumn{1}{l}{(0.178)} \\
Conciliator\#\#ISV & \multicolumn{1}{l}{0.0701} & \multicolumn{1}{l}{-0.186} & \multicolumn{1}{l}{0.111} & \multicolumn{1}{l}{0.0468} & \multicolumn{1}{l}{-0.212} & \multicolumn{1}{l}{-0.329***} & \multicolumn{1}{l}{0.157} & \multicolumn{1}{l}{-0.0289} & \multicolumn{1}{l}{-0.213} & \multicolumn{1}{l}{-0.0419} & \multicolumn{1}{l}{0.162} & \multicolumn{1}{l}{0.138} \\
      & \multicolumn{1}{l}{(0.268)} & \multicolumn{1}{l}{(0.222)} & \multicolumn{1}{l}{(0.227)} & \multicolumn{1}{l}{(0.323)} & \multicolumn{1}{l}{(0.143)} & \multicolumn{1}{l}{(0.107)} & \multicolumn{1}{l}{(0.145)} & \multicolumn{1}{l}{(0.154)} & \multicolumn{1}{l}{(0.182)} & \multicolumn{1}{l}{(0.216)} & \multicolumn{1}{l}{(0.160)} & \multicolumn{1}{l}{(0.154)} \\
Public Lawyer & \multicolumn{1}{l}{} & \multicolumn{1}{l}{-15677.6*} & \multicolumn{1}{l}{} & \multicolumn{1}{l}{-0.304} & \multicolumn{1}{l}{} & \multicolumn{1}{l}{-28992.9***} & \multicolumn{1}{l}{} & \multicolumn{1}{l}{0.536} & \multicolumn{1}{l}{} & \multicolumn{1}{l}{-1931.1} & \multicolumn{1}{l}{} & \multicolumn{1}{l}{0.205} \\
      & \multicolumn{1}{l}{} & \multicolumn{1}{l}{(8556.0)} & \multicolumn{1}{l}{} & \multicolumn{1}{l}{(0.323)} & \multicolumn{1}{l}{} & \multicolumn{1}{l}{(8097.6)} & \multicolumn{1}{l}{} & \multicolumn{1}{l}{(0.329)} & \multicolumn{1}{l}{} & \multicolumn{1}{l}{(4689.8)} & \multicolumn{1}{l}{} & \multicolumn{1}{l}{(0.382)} \\
Female & \multicolumn{1}{l}{} & \multicolumn{1}{l}{3741.8} & \multicolumn{1}{l}{} & \multicolumn{1}{l}{0.405} & \multicolumn{1}{l}{} & \multicolumn{1}{l}{-529.9} & \multicolumn{1}{l}{} & \multicolumn{1}{l}{0.0196} & \multicolumn{1}{l}{} & \multicolumn{1}{l}{10206.3*} & \multicolumn{1}{l}{} & \multicolumn{1}{l}{0.122} \\
      & \multicolumn{1}{l}{} & \multicolumn{1}{l}{(8486.2)} & \multicolumn{1}{l}{} & \multicolumn{1}{l}{(0.314)} & \multicolumn{1}{l}{} & \multicolumn{1}{l}{(9734.3)} & \multicolumn{1}{l}{} & \multicolumn{1}{l}{(0.241)} & \multicolumn{1}{l}{} & \multicolumn{1}{l}{(5691.8)} & \multicolumn{1}{l}{} & \multicolumn{1}{l}{(0.227)} \\
At will worker & \multicolumn{1}{l}{} & \multicolumn{1}{l}{20182.2} & \multicolumn{1}{l}{} & \multicolumn{1}{l}{-0.150} & \multicolumn{1}{l}{} & \multicolumn{1}{l}{8222.8} & \multicolumn{1}{l}{} & \multicolumn{1}{l}{0.0494} & \multicolumn{1}{l}{} & \multicolumn{1}{l}{-8690.8} & \multicolumn{1}{l}{} & \multicolumn{1}{l}{0.374} \\
      & \multicolumn{1}{l}{} & \multicolumn{1}{l}{(19664.8)} & \multicolumn{1}{l}{} & \multicolumn{1}{l}{(0.448)} & \multicolumn{1}{l}{} & \multicolumn{1}{l}{(12265.6)} & \multicolumn{1}{l}{} & \multicolumn{1}{l}{(0.314)} & \multicolumn{1}{l}{} & \multicolumn{1}{l}{(8447.4)} & \multicolumn{1}{l}{} & \multicolumn{1}{l}{(0.304)} \\
Tenure & \multicolumn{1}{l}{} & \multicolumn{1}{l}{907.2*} & \multicolumn{1}{l}{} & \multicolumn{1}{l}{0.00709} & \multicolumn{1}{l}{} & \multicolumn{1}{l}{21.71} & \multicolumn{1}{l}{} & \multicolumn{1}{l}{0.0129} & \multicolumn{1}{l}{} & \multicolumn{1}{l}{845.3} & \multicolumn{1}{l}{} & \multicolumn{1}{l}{0.0187} \\
      & \multicolumn{1}{l}{} & \multicolumn{1}{l}{(535.5)} & \multicolumn{1}{l}{} & \multicolumn{1}{l}{(0.0307)} & \multicolumn{1}{l}{} & \multicolumn{1}{l}{(670.1)} & \multicolumn{1}{l}{} & \multicolumn{1}{l}{(0.0158)} & \multicolumn{1}{l}{} & \multicolumn{1}{l}{(622.2)} & \multicolumn{1}{l}{} & \multicolumn{1}{l}{(0.0199)} \\
Daily wage & \multicolumn{1}{l}{} & \multicolumn{1}{l}{2.571} & \multicolumn{1}{l}{} & \multicolumn{1}{l}{-0.00000786} & \multicolumn{1}{l}{} & \multicolumn{1}{l}{-0.105} & \multicolumn{1}{l}{} & \multicolumn{1}{l}{0.0000949} & \multicolumn{1}{l}{} & \multicolumn{1}{l}{2.656} & \multicolumn{1}{l}{} & \multicolumn{1}{l}{0.0000361} \\
      & \multicolumn{1}{l}{} & \multicolumn{1}{l}{(3.020)} & \multicolumn{1}{l}{} & \multicolumn{1}{l}{(0.0000687)} & \multicolumn{1}{l}{} & \multicolumn{1}{l}{(2.861)} & \multicolumn{1}{l}{} & \multicolumn{1}{l}{(0.0000588)} & \multicolumn{1}{l}{} & \multicolumn{1}{l}{(1.958)} & \multicolumn{1}{l}{} & \multicolumn{1}{l}{(0.0000285)} \\
Weekly hours & \multicolumn{1}{l}{} & \multicolumn{1}{l}{-95.09} & \multicolumn{1}{l}{} & \multicolumn{1}{l}{-0.00569} & \multicolumn{1}{l}{} & \multicolumn{1}{l}{362.7} & \multicolumn{1}{l}{} & \multicolumn{1}{l}{0.00672} & \multicolumn{1}{l}{} & \multicolumn{1}{l}{3.170} & \multicolumn{1}{l}{} & \multicolumn{1}{l}{0.00153} \\
      & \multicolumn{1}{l}{} & \multicolumn{1}{l}{(296.0)} & \multicolumn{1}{l}{} & \multicolumn{1}{l}{(0.00989)} & \multicolumn{1}{l}{} & \multicolumn{1}{l}{(243.4)} & \multicolumn{1}{l}{} & \multicolumn{1}{l}{(0.00735)} & \multicolumn{1}{l}{} & \multicolumn{1}{l}{(103.2)} & \multicolumn{1}{l}{} & \multicolumn{1}{l}{(0.00484)} \\
      & \multicolumn{1}{l}{} & \multicolumn{1}{l}{} & \multicolumn{1}{l}{} & \multicolumn{1}{l}{} & \multicolumn{1}{l}{} & \multicolumn{1}{l}{} & \multicolumn{1}{l}{} & \multicolumn{1}{l}{} & \multicolumn{1}{l}{} & \multicolumn{1}{l}{} & \multicolumn{1}{l}{} & \multicolumn{1}{l}{} \\
      \midrule
Observations & \multicolumn{1}{l}{102} & \multicolumn{1}{l}{86} & \multicolumn{1}{l}{99} & \multicolumn{1}{l}{83} & \multicolumn{1}{l}{297} & \multicolumn{1}{l}{241} & \multicolumn{1}{l}{297} & \multicolumn{1}{l}{241} & \multicolumn{1}{l}{269} & \multicolumn{1}{l}{218} & \multicolumn{1}{l}{269} & \multicolumn{1}{l}{218} \\
R-squared & \multicolumn{1}{l}{0.567} & \multicolumn{1}{l}{0.672} & \multicolumn{1}{l}{0.374} & \multicolumn{1}{l}{0.298} & \multicolumn{1}{l}{0.389} & \multicolumn{1}{l}{0.443} & \multicolumn{1}{l}{0.253} & \multicolumn{1}{l}{0.144} & \multicolumn{1}{l}{0.146} & \multicolumn{1}{l}{0.185} & \multicolumn{1}{l}{0.145} & \multicolumn{1}{l}{0.116} \\
Dep Var Mean & \multicolumn{2}{c}{73384.4} & \multicolumn{2}{c}{2.489} & \multicolumn{2}{c}{78766.0} & \multicolumn{2}{c}{3.372} & \multicolumn{2}{c}{29204.9} & \multicolumn{2}{c}{3.392} \\
\bottomrule
\end{tabular}%
}
 \end{center}
  \footnotesize
 \textit{Notes:} 
 Dependent variable are expectations measured in the exit survey (for amount and delay respectively). Specification (2) includes basic controls.
 \textit{Do file: } \texttt{exit\_vs\_initial\_exp.do}
 \end{table}

 \begin{table}[H]
 \caption{Belief updating winsorizing at 90 percentile}
 \label{Belief_updating}
 \begin{center}
 \scriptsize{% Table generated by Excel2LaTeX from sheet 'exit_vs_initial_exp'
\begin{tabular}{rrrrrrrrrrrrr}
\toprule
      & \multicolumn{4}{c}{Employee}  & \multicolumn{4}{c}{Employee's Lawyer} & \multicolumn{4}{c}{Firm's Lawyer} \\
\midrule
      & \multicolumn{2}{c}{Amount} & \multicolumn{2}{c}{Time} & \multicolumn{2}{c}{Amount} & \multicolumn{2}{c}{Time} & \multicolumn{2}{c}{Amount} & \multicolumn{2}{c}{Time} \\
      & \multicolumn{1}{c}{(1)} & \multicolumn{1}{c}{(2)} & \multicolumn{1}{c}{(1)} & \multicolumn{1}{c}{(2)} & \multicolumn{1}{c}{(1)} & \multicolumn{1}{c}{(2)} & \multicolumn{1}{c}{(1)} & \multicolumn{1}{c}{(2)} & \multicolumn{1}{c}{(1)} & \multicolumn{1}{c}{(2)} & \multicolumn{1}{c}{(1)} & \multicolumn{1}{c}{(2)} \\
Initial survey var (ISV) & \multicolumn{1}{l}{0.352***} & \multicolumn{1}{l}{0.437***} & \multicolumn{1}{l}{0.472**} & \multicolumn{1}{l}{0.226} & \multicolumn{1}{l}{0.450***} & \multicolumn{1}{l}{0.398***} & \multicolumn{1}{l}{0.453***} & \multicolumn{1}{l}{0.327***} & \multicolumn{1}{l}{0.333***} & \multicolumn{1}{l}{0.292***} & \multicolumn{1}{l}{0.256**} & \multicolumn{1}{l}{0.105} \\
      & \multicolumn{1}{l}{(0.115)} & \multicolumn{1}{l}{(0.118)} & \multicolumn{1}{l}{(0.181)} & \multicolumn{1}{l}{(0.203)} & \multicolumn{1}{l}{(0.0650)} & \multicolumn{1}{l}{(0.0702)} & \multicolumn{1}{l}{(0.0869)} & \multicolumn{1}{l}{(0.0901)} & \multicolumn{1}{l}{(0.0964)} & \multicolumn{1}{l}{(0.112)} & \multicolumn{1}{l}{(0.105)} & \multicolumn{1}{l}{(0.107)} \\
Control & \multicolumn{1}{l}{43824.1***} & \multicolumn{1}{l}{30844.0} & \multicolumn{1}{l}{1.027**} & \multicolumn{1}{l}{1.714**} & \multicolumn{1}{l}{24899.9***} & \multicolumn{1}{l}{15309.6} & \multicolumn{1}{l}{1.315***} & \multicolumn{1}{l}{1.064*} & \multicolumn{1}{l}{14857.8***} & \multicolumn{1}{l}{14796.6**} & \multicolumn{1}{l}{2.064***} & \multicolumn{1}{l}{2.002***} \\
      & \multicolumn{1}{l}{(9318.2)} & \multicolumn{1}{l}{(20414.6)} & \multicolumn{1}{l}{(0.460)} & \multicolumn{1}{l}{(0.731)} & \multicolumn{1}{l}{(4634.5)} & \multicolumn{1}{l}{(11678.6)} & \multicolumn{1}{l}{(0.353)} & \multicolumn{1}{l}{(0.576)} & \multicolumn{1}{l}{(2915.6)} & \multicolumn{1}{l}{(6141.1)} & \multicolumn{1}{l}{(0.414)} & \multicolumn{1}{l}{(0.553)} \\
Calculator & \multicolumn{1}{l}{-38545.7***} & \multicolumn{1}{l}{-25686.2***} & \multicolumn{1}{l}{-0.786} & \multicolumn{1}{l}{-1.147} & \multicolumn{1}{l}{-7291.2} & \multicolumn{1}{l}{-6441.5} & \multicolumn{1}{l}{0.193} & \multicolumn{1}{l}{-0.159} & \multicolumn{1}{l}{-2381.7} & \multicolumn{1}{l}{-3471.4} & \multicolumn{1}{l}{-0.116} & \multicolumn{1}{l}{-0.390} \\
      & \multicolumn{1}{l}{(10063.8)} & \multicolumn{1}{l}{(9652.4)} & \multicolumn{1}{l}{(0.594)} & \multicolumn{1}{l}{(0.693)} & \multicolumn{1}{l}{(7549.8)} & \multicolumn{1}{l}{(7360.0)} & \multicolumn{1}{l}{(0.535)} & \multicolumn{1}{l}{(0.510)} & \multicolumn{1}{l}{(4094.5)} & \multicolumn{1}{l}{(4720.6)} & \multicolumn{1}{l}{(0.659)} & \multicolumn{1}{l}{(0.663)} \\
Conciliator & \multicolumn{1}{l}{-20660.1} & \multicolumn{1}{l}{-11282.9} & \multicolumn{1}{l}{0.117} & \multicolumn{1}{l}{0.277} & \multicolumn{1}{l}{9384.7} & \multicolumn{1}{l}{8749.8} & \multicolumn{1}{l}{-0.128} & \multicolumn{1}{l}{0.147} & \multicolumn{1}{l}{-2128.4} & \multicolumn{1}{l}{-4365.7} & \multicolumn{1}{l}{-0.639} & \multicolumn{1}{l}{-0.639} \\
      & \multicolumn{1}{l}{(15124.2)} & \multicolumn{1}{l}{(16621.4)} & \multicolumn{1}{l}{(0.715)} & \multicolumn{1}{l}{(1.047)} & \multicolumn{1}{l}{(8837.6)} & \multicolumn{1}{l}{(8322.3)} & \multicolumn{1}{l}{(0.543)} & \multicolumn{1}{l}{(0.601)} & \multicolumn{1}{l}{(4061.6)} & \multicolumn{1}{l}{(4962.1)} & \multicolumn{1}{l}{(0.575)} & \multicolumn{1}{l}{(0.607)} \\
Calculator\#\#ISV & \multicolumn{1}{l}{0.329**} & \multicolumn{1}{l}{0.157} & \multicolumn{1}{l}{0.431**} & \multicolumn{1}{l}{0.602**} & \multicolumn{1}{l}{0.0866} & \multicolumn{1}{l}{0.0586} & \multicolumn{1}{l}{-0.0558} & \multicolumn{1}{l}{0.0117} & \multicolumn{1}{l}{-0.0452} & \multicolumn{1}{l}{-0.0627} & \multicolumn{1}{l}{0.0751} & \multicolumn{1}{l}{0.206} \\
      & \multicolumn{1}{l}{(0.143)} & \multicolumn{1}{l}{(0.150)} & \multicolumn{1}{l}{(0.211)} & \multicolumn{1}{l}{(0.249)} & \multicolumn{1}{l}{(0.105)} & \multicolumn{1}{l}{(0.111)} & \multicolumn{1}{l}{(0.126)} & \multicolumn{1}{l}{(0.124)} & \multicolumn{1}{l}{(0.141)} & \multicolumn{1}{l}{(0.162)} & \multicolumn{1}{l}{(0.157)} & \multicolumn{1}{l}{(0.161)} \\
Conciliator\#\#ISV & \multicolumn{1}{l}{0.162} & \multicolumn{1}{l}{-0.0745} & \multicolumn{1}{l}{0.0960} & \multicolumn{1}{l}{0.0450} & \multicolumn{1}{l}{-0.108} & \multicolumn{1}{l}{-0.215*} & \multicolumn{1}{l}{0.0259} & \multicolumn{1}{l}{-0.0952} & \multicolumn{1}{l}{-0.129} & \multicolumn{1}{l}{-0.0119} & \multicolumn{1}{l}{0.140} & \multicolumn{1}{l}{0.142} \\
      & \multicolumn{1}{l}{(0.155)} & \multicolumn{1}{l}{(0.204)} & \multicolumn{1}{l}{(0.252)} & \multicolumn{1}{l}{(0.355)} & \multicolumn{1}{l}{(0.115)} & \multicolumn{1}{l}{(0.120)} & \multicolumn{1}{l}{(0.137)} & \multicolumn{1}{l}{(0.157)} & \multicolumn{1}{l}{(0.134)} & \multicolumn{1}{l}{(0.156)} & \multicolumn{1}{l}{(0.148)} & \multicolumn{1}{l}{(0.157)} \\
Public Lawyer & \multicolumn{1}{l}{} & \multicolumn{1}{l}{-14926.0*} & \multicolumn{1}{l}{} & \multicolumn{1}{l}{-0.301} & \multicolumn{1}{l}{} & \multicolumn{1}{l}{-24184.8***} & \multicolumn{1}{l}{} & \multicolumn{1}{l}{0.474} & \multicolumn{1}{l}{} & \multicolumn{1}{l}{-896.4} & \multicolumn{1}{l}{} & \multicolumn{1}{l}{0.182} \\
      & \multicolumn{1}{l}{} & \multicolumn{1}{l}{(8055.2)} & \multicolumn{1}{l}{} & \multicolumn{1}{l}{(0.322)} & \multicolumn{1}{l}{} & \multicolumn{1}{l}{(6258.5)} & \multicolumn{1}{l}{} & \multicolumn{1}{l}{(0.308)} & \multicolumn{1}{l}{} & \multicolumn{1}{l}{(4230.6)} & \multicolumn{1}{l}{} & \multicolumn{1}{l}{(0.377)} \\
Female & \multicolumn{1}{l}{} & \multicolumn{1}{l}{5043.3} & \multicolumn{1}{l}{} & \multicolumn{1}{l}{0.402} & \multicolumn{1}{l}{} & \multicolumn{1}{l}{1441.0} & \multicolumn{1}{l}{} & \multicolumn{1}{l}{0.0283} & \multicolumn{1}{l}{} & \multicolumn{1}{l}{3659.0} & \multicolumn{1}{l}{} & \multicolumn{1}{l}{0.103} \\
      & \multicolumn{1}{l}{} & \multicolumn{1}{l}{(8191.7)} & \multicolumn{1}{l}{} & \multicolumn{1}{l}{(0.316)} & \multicolumn{1}{l}{} & \multicolumn{1}{l}{(6502.1)} & \multicolumn{1}{l}{} & \multicolumn{1}{l}{(0.230)} & \multicolumn{1}{l}{} & \multicolumn{1}{l}{(3085.8)} & \multicolumn{1}{l}{} & \multicolumn{1}{l}{(0.221)} \\
At will worker & \multicolumn{1}{l}{} & \multicolumn{1}{l}{11931.6} & \multicolumn{1}{l}{} & \multicolumn{1}{l}{-0.0984} & \multicolumn{1}{l}{} & \multicolumn{1}{l}{7972.7} & \multicolumn{1}{l}{} & \multicolumn{1}{l}{0.0650} & \multicolumn{1}{l}{} & \multicolumn{1}{l}{-2191.7} & \multicolumn{1}{l}{} & \multicolumn{1}{l}{0.372} \\
      & \multicolumn{1}{l}{} & \multicolumn{1}{l}{(16031.8)} & \multicolumn{1}{l}{} & \multicolumn{1}{l}{(0.455)} & \multicolumn{1}{l}{} & \multicolumn{1}{l}{(7967.2)} & \multicolumn{1}{l}{} & \multicolumn{1}{l}{(0.310)} & \multicolumn{1}{l}{} & \multicolumn{1}{l}{(5317.3)} & \multicolumn{1}{l}{} & \multicolumn{1}{l}{(0.302)} \\
Tenure & \multicolumn{1}{l}{} & \multicolumn{1}{l}{970.0**} & \multicolumn{1}{l}{} & \multicolumn{1}{l}{0.00271} & \multicolumn{1}{l}{} & \multicolumn{1}{l}{375.3} & \multicolumn{1}{l}{} & \multicolumn{1}{l}{0.0132} & \multicolumn{1}{l}{} & \multicolumn{1}{l}{235.1} & \multicolumn{1}{l}{} & \multicolumn{1}{l}{0.0165} \\
      & \multicolumn{1}{l}{} & \multicolumn{1}{l}{(440.5)} & \multicolumn{1}{l}{} & \multicolumn{1}{l}{(0.0276)} & \multicolumn{1}{l}{} & \multicolumn{1}{l}{(544.0)} & \multicolumn{1}{l}{} & \multicolumn{1}{l}{(0.0154)} & \multicolumn{1}{l}{} & \multicolumn{1}{l}{(295.4)} & \multicolumn{1}{l}{} & \multicolumn{1}{l}{(0.0193)} \\
Daily wage & \multicolumn{1}{l}{} & \multicolumn{1}{l}{2.500} & \multicolumn{1}{l}{} & \multicolumn{1}{l}{-0.00000784} & \multicolumn{1}{l}{} & \multicolumn{1}{l}{-0.973} & \multicolumn{1}{l}{} & \multicolumn{1}{l}{0.0000920} & \multicolumn{1}{l}{} & \multicolumn{1}{l}{0.478} & \multicolumn{1}{l}{} & \multicolumn{1}{l}{0.0000402} \\
      & \multicolumn{1}{l}{} & \multicolumn{1}{l}{(2.990)} & \multicolumn{1}{l}{} & \multicolumn{1}{l}{(0.0000695)} & \multicolumn{1}{l}{} & \multicolumn{1}{l}{(1.950)} & \multicolumn{1}{l}{} & \multicolumn{1}{l}{(0.0000608)} & \multicolumn{1}{l}{} & \multicolumn{1}{l}{(0.897)} & \multicolumn{1}{l}{} & \multicolumn{1}{l}{(0.0000278)} \\
Weekly hours & \multicolumn{1}{l}{} & \multicolumn{1}{l}{31.15} & \multicolumn{1}{l}{} & \multicolumn{1}{l}{-0.00551} & \multicolumn{1}{l}{} & \multicolumn{1}{l}{197.6} & \multicolumn{1}{l}{} & \multicolumn{1}{l}{0.00686} & \multicolumn{1}{l}{} & \multicolumn{1}{l}{-18.61} & \multicolumn{1}{l}{} & \multicolumn{1}{l}{0.00161} \\
      & \multicolumn{1}{l}{} & \multicolumn{1}{l}{(304.1)} & \multicolumn{1}{l}{} & \multicolumn{1}{l}{(0.00970)} & \multicolumn{1}{l}{} & \multicolumn{1}{l}{(183.4)} & \multicolumn{1}{l}{} & \multicolumn{1}{l}{(0.00713)} & \multicolumn{1}{l}{} & \multicolumn{1}{l}{(56.73)} & \multicolumn{1}{l}{} & \multicolumn{1}{l}{(0.00482)} \\
      & \multicolumn{1}{l}{} & \multicolumn{1}{l}{} & \multicolumn{1}{l}{} & \multicolumn{1}{l}{} & \multicolumn{1}{l}{} & \multicolumn{1}{l}{} & \multicolumn{1}{l}{} & \multicolumn{1}{l}{} & \multicolumn{1}{l}{} & \multicolumn{1}{l}{} & \multicolumn{1}{l}{} & \multicolumn{1}{l}{} \\
       \midrule
Observations & \multicolumn{1}{l}{102} & \multicolumn{1}{l}{86} & \multicolumn{1}{l}{99} & \multicolumn{1}{l}{83} & \multicolumn{1}{l}{297} & \multicolumn{1}{l}{241} & \multicolumn{1}{l}{297} & \multicolumn{1}{l}{241} & \multicolumn{1}{l}{269} & \multicolumn{1}{l}{218} & \multicolumn{1}{l}{269} & \multicolumn{1}{l}{218} \\
R-squared & \multicolumn{1}{l}{0.504} & \multicolumn{1}{l}{0.551} & \multicolumn{1}{l}{0.342} & \multicolumn{1}{l}{0.283} & \multicolumn{1}{l}{0.365} & \multicolumn{1}{l}{0.392} & \multicolumn{1}{l}{0.219} & \multicolumn{1}{l}{0.178} & \multicolumn{1}{l}{0.155} & \multicolumn{1}{l}{0.157} & \multicolumn{1}{l}{0.124} & \multicolumn{1}{l}{0.112} \\
Dep Var Mean & \multicolumn{2}{c}{60679.5} & \multicolumn{2}{c}{2.436} & \multicolumn{2}{c}{60779.4} & \multicolumn{2}{c}{3.222} & \multicolumn{2}{c}{22178.9} & \multicolumn{2}{c}{3.276} \\
\bottomrule
\end{tabular}%
}
 \end{center}
  \footnotesize
 \textit{Notes:} 
 Dependent variable are expectations measured in the exit survey (for amount and delay respectively). Specification (2) includes basic controls.
 \textit{Do file: } \texttt{exit\_vs\_initial\_exp.do}
 \end{table}


 \begin{table}[H]
 \caption{Belief updating (amounts in log)}
 \label{Belief_updating_log}
 \begin{center}
 \scriptsize{% Table generated by Excel2LaTeX from sheet 'exit_vs_initial_exp_log'
\begin{tabular}{rrrrrrrrrrrrr}
\toprule
      & \multicolumn{4}{c}{Employee}  & \multicolumn{4}{c}{Employee's Lawyer} & \multicolumn{4}{c}{Firm's Lawyer} \\
\midrule
      & \multicolumn{2}{c}{Amount} & \multicolumn{2}{c}{Time} & \multicolumn{2}{c}{Amount} & \multicolumn{2}{c}{Time} & \multicolumn{2}{c}{Amount} & \multicolumn{2}{c}{Time} \\
      & \multicolumn{1}{c}{(1)} & \multicolumn{1}{c}{(2)} & \multicolumn{1}{c}{(1)} & \multicolumn{1}{c}{(2)} & \multicolumn{1}{c}{(1)} & \multicolumn{1}{c}{(2)} & \multicolumn{1}{c}{(1)} & \multicolumn{1}{c}{(2)} & \multicolumn{1}{c}{(1)} & \multicolumn{1}{c}{(2)} & \multicolumn{1}{c}{(1)} & \multicolumn{1}{c}{(2)} \\
Initial survey var (ISV) & \multicolumn{1}{l}{0.282**} & \multicolumn{1}{l}{0.295*} & \multicolumn{1}{l}{0.516**} & \multicolumn{1}{l}{0.187} & \multicolumn{1}{l}{0.223*} & \multicolumn{1}{l}{0.217} & \multicolumn{1}{l}{0.594***} & \multicolumn{1}{l}{0.236***} & \multicolumn{1}{l}{0.369**} & \multicolumn{1}{l}{0.493***} & \multicolumn{1}{l}{0.211***} & \multicolumn{1}{l}{0.0582} \\
      & \multicolumn{1}{l}{(0.142)} & \multicolumn{1}{l}{(0.160)} & \multicolumn{1}{l}{(0.212)} & \multicolumn{1}{l}{(0.123)} & \multicolumn{1}{l}{(0.117)} & \multicolumn{1}{l}{(0.138)} & \multicolumn{1}{l}{(0.185)} & \multicolumn{1}{l}{(0.0876)} & \multicolumn{1}{l}{(0.146)} & \multicolumn{1}{l}{(0.150)} & \multicolumn{1}{l}{(0.0625)} & \multicolumn{1}{l}{(0.108)} \\
Control & \multicolumn{1}{l}{8.062***} & \multicolumn{1}{l}{9.061***} & \multicolumn{1}{l}{0.911*} & \multicolumn{1}{l}{1.777**} & \multicolumn{1}{l}{7.734***} & \multicolumn{1}{l}{7.980***} & \multicolumn{1}{l}{0.896} & \multicolumn{1}{l}{1.425**} & \multicolumn{1}{l}{4.832***} & \multicolumn{1}{l}{3.799**} & \multicolumn{1}{l}{2.255***} & \multicolumn{1}{l}{2.172***} \\
      & \multicolumn{1}{l}{(1.494)} & \multicolumn{1}{l}{(1.821)} & \multicolumn{1}{l}{(0.519)} & \multicolumn{1}{l}{(0.677)} & \multicolumn{1}{l}{(1.206)} & \multicolumn{1}{l}{(1.530)} & \multicolumn{1}{l}{(0.638)} & \multicolumn{1}{l}{(0.615)} & \multicolumn{1}{l}{(1.332)} & \multicolumn{1}{l}{(1.550)} & \multicolumn{1}{l}{(0.299)} & \multicolumn{1}{l}{(0.562)} \\
Calculator & \multicolumn{1}{l}{-2.925} & \multicolumn{1}{l}{-3.508} & \multicolumn{1}{l}{-0.710} & \multicolumn{1}{l}{-1.262**} & \multicolumn{1}{l}{-2.271} & \multicolumn{1}{l}{-2.621} & \multicolumn{1}{l}{0.772} & \multicolumn{1}{l}{-0.191} & \multicolumn{1}{l}{1.782} & \multicolumn{1}{l}{1.544} & \multicolumn{1}{l}{-0.219} & \multicolumn{1}{l}{-0.452} \\
      & \multicolumn{1}{l}{(2.760)} & \multicolumn{1}{l}{(3.257)} & \multicolumn{1}{l}{(0.652)} & \multicolumn{1}{l}{(0.587)} & \multicolumn{1}{l}{(2.006)} & \multicolumn{1}{l}{(2.134)} & \multicolumn{1}{l}{(0.810)} & \multicolumn{1}{l}{(0.518)} & \multicolumn{1}{l}{(2.041)} & \multicolumn{1}{l}{(2.158)} & \multicolumn{1}{l}{(0.718)} & \multicolumn{1}{l}{(0.770)} \\
Conciliator & \multicolumn{1}{l}{-5.687*} & \multicolumn{1}{l}{-4.337} & \multicolumn{1}{l}{0.285} & \multicolumn{1}{l}{0.203} & \multicolumn{1}{l}{2.520} & \multicolumn{1}{l}{2.644} & \multicolumn{1}{l}{-0.247} & \multicolumn{1}{l}{-0.248} & \multicolumn{1}{l}{1.959} & \multicolumn{1}{l}{2.280} & \multicolumn{1}{l}{-0.507} & \multicolumn{1}{l}{-0.701} \\
      & \multicolumn{1}{l}{(3.122)} & \multicolumn{1}{l}{(3.429)} & \multicolumn{1}{l}{(0.750)} & \multicolumn{1}{l}{(0.935)} & \multicolumn{1}{l}{(1.557)} & \multicolumn{1}{l}{(1.708)} & \multicolumn{1}{l}{(0.754)} & \multicolumn{1}{l}{(0.596)} & \multicolumn{1}{l}{(1.650)} & \multicolumn{1}{l}{(1.703)} & \multicolumn{1}{l}{(0.418)} & \multicolumn{1}{l}{(0.574)} \\
Calculator\#\#ISV & \multicolumn{1}{l}{0.200} & \multicolumn{1}{l}{0.279} & \multicolumn{1}{l}{0.462*} & \multicolumn{1}{l}{0.660***} & \multicolumn{1}{l}{0.207} & \multicolumn{1}{l}{0.244} & \multicolumn{1}{l}{-0.221} & \multicolumn{1}{l}{0.00684} & \multicolumn{1}{l}{-0.195} & \multicolumn{1}{l}{-0.190} & \multicolumn{1}{l}{0.127} & \multicolumn{1}{l}{0.239} \\
      & \multicolumn{1}{l}{(0.255)} & \multicolumn{1}{l}{(0.300)} & \multicolumn{1}{l}{(0.242)} & \multicolumn{1}{l}{(0.184)} & \multicolumn{1}{l}{(0.191)} & \multicolumn{1}{l}{(0.204)} & \multicolumn{1}{l}{(0.220)} & \multicolumn{1}{l}{(0.122)} & \multicolumn{1}{l}{(0.218)} & \multicolumn{1}{l}{(0.230)} & \multicolumn{1}{l}{(0.165)} & \multicolumn{1}{l}{(0.192)} \\
Conciliator\#\#ISV & \multicolumn{1}{l}{0.371} & \multicolumn{1}{l}{0.222} & \multicolumn{1}{l}{0.0788} & \multicolumn{1}{l}{0.0762} & \multicolumn{1}{l}{-0.241} & \multicolumn{1}{l}{-0.267} & \multicolumn{1}{l}{0.0408} & \multicolumn{1}{l}{-0.00172} & \multicolumn{1}{l}{-0.295} & \multicolumn{1}{l}{-0.310} & \multicolumn{1}{l}{0.106} & \multicolumn{1}{l}{0.158} \\
      & \multicolumn{1}{l}{(0.274)} & \multicolumn{1}{l}{(0.329)} & \multicolumn{1}{l}{(0.258)} & \multicolumn{1}{l}{(0.289)} & \multicolumn{1}{l}{(0.159)} & \multicolumn{1}{l}{(0.174)} & \multicolumn{1}{l}{(0.212)} & \multicolumn{1}{l}{(0.151)} & \multicolumn{1}{l}{(0.183)} & \multicolumn{1}{l}{(0.191)} & \multicolumn{1}{l}{(0.0948)} & \multicolumn{1}{l}{(0.143)} \\
Public Lawyer & \multicolumn{1}{l}{} & \multicolumn{1}{l}{0.0670} & \multicolumn{1}{l}{} & \multicolumn{1}{l}{-0.331} & \multicolumn{1}{l}{} & \multicolumn{1}{l}{-1.358**} & \multicolumn{1}{l}{} & \multicolumn{1}{l}{0.548*} & \multicolumn{1}{l}{} & \multicolumn{1}{l}{0.517} & \multicolumn{1}{l}{} & \multicolumn{1}{l}{0.111} \\
      & \multicolumn{1}{l}{} & \multicolumn{1}{l}{(0.583)} & \multicolumn{1}{l}{} & \multicolumn{1}{l}{(0.324)} & \multicolumn{1}{l}{} & \multicolumn{1}{l}{(0.531)} & \multicolumn{1}{l}{} & \multicolumn{1}{l}{(0.328)} & \multicolumn{1}{l}{} & \multicolumn{1}{l}{(0.680)} & \multicolumn{1}{l}{} & \multicolumn{1}{l}{(0.382)} \\
Female & \multicolumn{1}{l}{} & \multicolumn{1}{l}{-0.403} & \multicolumn{1}{l}{} & \multicolumn{1}{l}{0.412} & \multicolumn{1}{l}{} & \multicolumn{1}{l}{0.00239} & \multicolumn{1}{l}{} & \multicolumn{1}{l}{0.0211} & \multicolumn{1}{l}{} & \multicolumn{1}{l}{-0.0441} & \multicolumn{1}{l}{} & \multicolumn{1}{l}{0.144} \\
      & \multicolumn{1}{l}{} & \multicolumn{1}{l}{(0.700)} & \multicolumn{1}{l}{} & \multicolumn{1}{l}{(0.314)} & \multicolumn{1}{l}{} & \multicolumn{1}{l}{(0.478)} & \multicolumn{1}{l}{} & \multicolumn{1}{l}{(0.240)} & \multicolumn{1}{l}{} & \multicolumn{1}{l}{(0.562)} & \multicolumn{1}{l}{} & \multicolumn{1}{l}{(0.233)} \\
At will worker & \multicolumn{1}{l}{} & \multicolumn{1}{l}{0.713} & \multicolumn{1}{l}{} & \multicolumn{1}{l}{-0.112} & \multicolumn{1}{l}{} & \multicolumn{1}{l}{0.350} & \multicolumn{1}{l}{} & \multicolumn{1}{l}{0.0479} & \multicolumn{1}{l}{} & \multicolumn{1}{l}{-1.836**} & \multicolumn{1}{l}{} & \multicolumn{1}{l}{0.451} \\
      & \multicolumn{1}{l}{} & \multicolumn{1}{l}{(0.639)} & \multicolumn{1}{l}{} & \multicolumn{1}{l}{(0.463)} & \multicolumn{1}{l}{} & \multicolumn{1}{l}{(0.529)} & \multicolumn{1}{l}{} & \multicolumn{1}{l}{(0.313)} & \multicolumn{1}{l}{} & \multicolumn{1}{l}{(0.920)} & \multicolumn{1}{l}{} & \multicolumn{1}{l}{(0.311)} \\
Tenure & \multicolumn{1}{l}{} & \multicolumn{1}{l}{0.0619*} & \multicolumn{1}{l}{} & \multicolumn{1}{l}{0.00686} & \multicolumn{1}{l}{} & \multicolumn{1}{l}{-0.00336} & \multicolumn{1}{l}{} & \multicolumn{1}{l}{0.0129} & \multicolumn{1}{l}{} & \multicolumn{1}{l}{-0.0158} & \multicolumn{1}{l}{} & \multicolumn{1}{l}{0.0202} \\
      & \multicolumn{1}{l}{} & \multicolumn{1}{l}{(0.0339)} & \multicolumn{1}{l}{} & \multicolumn{1}{l}{(0.0313)} & \multicolumn{1}{l}{} & \multicolumn{1}{l}{(0.0290)} & \multicolumn{1}{l}{} & \multicolumn{1}{l}{(0.0158)} & \multicolumn{1}{l}{} & \multicolumn{1}{l}{(0.0480)} & \multicolumn{1}{l}{} & \multicolumn{1}{l}{(0.0205)} \\
Daily wage & \multicolumn{1}{l}{} & \multicolumn{1}{l}{0.000101} & \multicolumn{1}{l}{} & \multicolumn{1}{l}{-0.00000595} & \multicolumn{1}{l}{} & \multicolumn{1}{l}{-0.000114} & \multicolumn{1}{l}{} & \multicolumn{1}{l}{0.0000963} & \multicolumn{1}{l}{} & \multicolumn{1}{l}{-0.0000850} & \multicolumn{1}{l}{} & \multicolumn{1}{l}{0.00000579} \\
      & \multicolumn{1}{l}{} & \multicolumn{1}{l}{(0.000140)} & \multicolumn{1}{l}{} & \multicolumn{1}{l}{(0.0000689)} & \multicolumn{1}{l}{} & \multicolumn{1}{l}{(0.000171)} & \multicolumn{1}{l}{} & \multicolumn{1}{l}{(0.0000587)} & \multicolumn{1}{l}{} & \multicolumn{1}{l}{(0.000180)} & \multicolumn{1}{l}{} & \multicolumn{1}{l}{(0.0000395)} \\
Weekly hours & \multicolumn{1}{l}{} & \multicolumn{1}{l}{-0.0303} & \multicolumn{1}{l}{} & \multicolumn{1}{l}{-0.00556} & \multicolumn{1}{l}{} & \multicolumn{1}{l}{-0.00156} & \multicolumn{1}{l}{} & \multicolumn{1}{l}{0.00681} & \multicolumn{1}{l}{} & \multicolumn{1}{l}{0.00795} & \multicolumn{1}{l}{} & \multicolumn{1}{l}{0.00153} \\
      & \multicolumn{1}{l}{} & \multicolumn{1}{l}{(0.0222)} & \multicolumn{1}{l}{} & \multicolumn{1}{l}{(0.00988)} & \multicolumn{1}{l}{} & \multicolumn{1}{l}{(0.0106)} & \multicolumn{1}{l}{} & \multicolumn{1}{l}{(0.00734)} & \multicolumn{1}{l}{} & \multicolumn{1}{l}{(0.00846)} & \multicolumn{1}{l}{} & \multicolumn{1}{l}{(0.00476)} \\
      & \multicolumn{1}{l}{} & \multicolumn{1}{l}{} & \multicolumn{1}{l}{} & \multicolumn{1}{l}{} & \multicolumn{1}{l}{} & \multicolumn{1}{l}{} & \multicolumn{1}{l}{} & \multicolumn{1}{l}{} & \multicolumn{1}{l}{} & \multicolumn{1}{l}{} & \multicolumn{1}{l}{} & \multicolumn{1}{l}{} \\
      \midrule
Observations & \multicolumn{1}{l}{102} & \multicolumn{1}{l}{86} & \multicolumn{1}{l}{99} & \multicolumn{1}{l}{83} & \multicolumn{1}{l}{297} & \multicolumn{1}{l}{241} & \multicolumn{1}{l}{297} & \multicolumn{1}{l}{241} & \multicolumn{1}{l}{269} & \multicolumn{1}{l}{218} & \multicolumn{1}{l}{269} & \multicolumn{1}{l}{218} \\
R-squared & \multicolumn{1}{l}{0.314} & \multicolumn{1}{l}{0.366} & \multicolumn{1}{l}{0.349} & \multicolumn{1}{l}{0.313} & \multicolumn{1}{l}{0.0852} & \multicolumn{1}{l}{0.115} & \multicolumn{1}{l}{0.289} & \multicolumn{1}{l}{0.144} & \multicolumn{1}{l}{0.0510} & \multicolumn{1}{l}{0.122} & \multicolumn{1}{l}{0.152} & \multicolumn{1}{l}{0.111} \\
Dep Var Mean & \multicolumn{2}{c}{9.704} & \multicolumn{2}{c}{2.608} & \multicolumn{2}{c}{9.987} & \multicolumn{2}{c}{3.464} & \multicolumn{2}{c}{7.593} & \multicolumn{2}{c}{3.454} \\
\bottomrule
\end{tabular}%
}
 \end{center}
  \footnotesize
 \textit{Notes:} 
 Dependent variable are expectations measured in the exit survey (for amount (in log pesos) and delay respectively). Specification (2) includes basic controls.
 \textit{Do file: } \texttt{exit\_vs\_initial\_exp\_log.do}
 \end{table}





\subsection{Relative OC Expectations}

 \begin{table}[H]
 \caption{Belief updating (relative) winsorizing at 95 percentile}
 \label{Belief_updating}
 \begin{center}
 \scriptsize{% Table generated by Excel2LaTeX from sheet 'exit_vs_initial_exp'
\begin{tabular}{rrrrrrrrrrrrr}
\toprule
      & \multicolumn{4}{c}{Employee}  & \multicolumn{4}{c}{Employee's Lawyer} & \multicolumn{4}{c}{Firm's Lawyer} \\
\midrule
      & \multicolumn{2}{c}{Amount} & \multicolumn{2}{c}{Time} & \multicolumn{2}{c}{Amount} & \multicolumn{2}{c}{Time} & \multicolumn{2}{c}{Amount} & \multicolumn{2}{c}{Time} \\
      & \multicolumn{1}{c}{(1)} & \multicolumn{1}{c}{(2)} & \multicolumn{1}{c}{(1)} & \multicolumn{1}{c}{(2)} & \multicolumn{1}{c}{(1)} & \multicolumn{1}{c}{(2)} & \multicolumn{1}{c}{(1)} & \multicolumn{1}{c}{(2)} & \multicolumn{1}{c}{(1)} & \multicolumn{1}{c}{(2)} & \multicolumn{1}{c}{(1)} & \multicolumn{1}{c}{(2)} \\
Initial survey var (ISV) & \multicolumn{1}{l}{2.067*} & \multicolumn{1}{l}{2.028*} & \multicolumn{1}{l}{0.256*} & \multicolumn{1}{l}{0.273*} & \multicolumn{1}{l}{0.476***} & \multicolumn{1}{l}{0.451***} & \multicolumn{1}{l}{0.432***} & \multicolumn{1}{l}{0.392***} & \multicolumn{1}{l}{0.716***} & \multicolumn{1}{l}{0.673***} & \multicolumn{1}{l}{0.119} & \multicolumn{1}{l}{0.0927} \\
      & \multicolumn{1}{l}{(1.162)} & \multicolumn{1}{l}{(1.133)} & \multicolumn{1}{l}{(0.145)} & \multicolumn{1}{l}{(0.148)} & \multicolumn{1}{l}{(0.106)} & \multicolumn{1}{l}{(0.106)} & \multicolumn{1}{l}{(0.0666)} & \multicolumn{1}{l}{(0.0766)} & \multicolumn{1}{l}{(0.229)} & \multicolumn{1}{l}{(0.236)} & \multicolumn{1}{l}{(0.103)} & \multicolumn{1}{l}{(0.101)} \\
Control & \multicolumn{1}{l}{2.840} & \multicolumn{1}{l}{-5.992} & \multicolumn{1}{l}{0.551*} & \multicolumn{1}{l}{0.626} & \multicolumn{1}{l}{1.002**} & \multicolumn{1}{l}{-3.071*} & \multicolumn{1}{l}{0.666***} & \multicolumn{1}{l}{0.352} & \multicolumn{1}{l}{-0.00461} & \multicolumn{1}{l}{-1.532} & \multicolumn{1}{l}{1.246***} & \multicolumn{1}{l}{1.142**} \\
      & \multicolumn{1}{l}{(2.683)} & \multicolumn{1}{l}{(5.728)} & \multicolumn{1}{l}{(0.279)} & \multicolumn{1}{l}{(0.594)} & \multicolumn{1}{l}{(0.424)} & \multicolumn{1}{l}{(1.580)} & \multicolumn{1}{l}{(0.217)} & \multicolumn{1}{l}{(0.463)} & \multicolumn{1}{l}{(0.176)} & \multicolumn{1}{l}{(1.029)} & \multicolumn{1}{l}{(0.314)} & \multicolumn{1}{l}{(0.450)} \\
Calculator & \multicolumn{1}{l}{-3.667} & \multicolumn{1}{l}{-5.003} & \multicolumn{1}{l}{-0.465} & \multicolumn{1}{l}{-0.474} & \multicolumn{1}{l}{-0.312} & \multicolumn{1}{l}{-0.351} & \multicolumn{1}{l}{0.0784} & \multicolumn{1}{l}{0.0894} & \multicolumn{1}{l}{0.426} & \multicolumn{1}{l}{0.555} & \multicolumn{1}{l}{-0.225} & \multicolumn{1}{l}{-0.257} \\
      & \multicolumn{1}{l}{(2.754)} & \multicolumn{1}{l}{(3.243)} & \multicolumn{1}{l}{(0.384)} & \multicolumn{1}{l}{(0.414)} & \multicolumn{1}{l}{(0.637)} & \multicolumn{1}{l}{(0.614)} & \multicolumn{1}{l}{(0.340)} & \multicolumn{1}{l}{(0.353)} & \multicolumn{1}{l}{(0.564)} & \multicolumn{1}{l}{(0.592)} & \multicolumn{1}{l}{(0.475)} & \multicolumn{1}{l}{(0.473)} \\
Conciliator & \multicolumn{1}{l}{-2.482} & \multicolumn{1}{l}{-1.991} & \multicolumn{1}{l}{0.213} & \multicolumn{1}{l}{0.286} & \multicolumn{1}{l}{0.305} & \multicolumn{1}{l}{0.212} & \multicolumn{1}{l}{0.319} & \multicolumn{1}{l}{0.302} & \multicolumn{1}{l}{0.496} & \multicolumn{1}{l}{0.588} & \multicolumn{1}{l}{-0.678} & \multicolumn{1}{l}{-0.635} \\
      & \multicolumn{1}{l}{(2.759)} & \multicolumn{1}{l}{(2.451)} & \multicolumn{1}{l}{(0.630)} & \multicolumn{1}{l}{(0.679)} & \multicolumn{1}{l}{(0.808)} & \multicolumn{1}{l}{(0.790)} & \multicolumn{1}{l}{(0.386)} & \multicolumn{1}{l}{(0.383)} & \multicolumn{1}{l}{(0.414)} & \multicolumn{1}{l}{(0.404)} & \multicolumn{1}{l}{(0.421)} & \multicolumn{1}{l}{(0.412)} \\
Calculator\#\#ISV & \multicolumn{1}{l}{0.216} & \multicolumn{1}{l}{0.397} & \multicolumn{1}{l}{0.602***} & \multicolumn{1}{l}{0.573***} & \multicolumn{1}{l}{0.0647} & \multicolumn{1}{l}{0.0676} & \multicolumn{1}{l}{-0.0956} & \multicolumn{1}{l}{-0.0862} & \multicolumn{1}{l}{-0.00470} & \multicolumn{1}{l}{-0.0389} & \multicolumn{1}{l}{0.220} & \multicolumn{1}{l}{0.235} \\
      & \multicolumn{1}{l}{(1.667)} & \multicolumn{1}{l}{(1.649)} & \multicolumn{1}{l}{(0.176)} & \multicolumn{1}{l}{(0.192)} & \multicolumn{1}{l}{(0.155)} & \multicolumn{1}{l}{(0.141)} & \multicolumn{1}{l}{(0.110)} & \multicolumn{1}{l}{(0.115)} & \multicolumn{1}{l}{(0.390)} & \multicolumn{1}{l}{(0.393)} & \multicolumn{1}{l}{(0.155)} & \multicolumn{1}{l}{(0.152)} \\
Conciliator\#\#ISV & \multicolumn{1}{l}{-1.528} & \multicolumn{1}{l}{-1.596} & \multicolumn{1}{l}{0.163} & \multicolumn{1}{l}{0.118} & \multicolumn{1}{l}{-0.163} & \multicolumn{1}{l}{-0.216} & \multicolumn{1}{l}{-0.234*} & \multicolumn{1}{l}{-0.223*} & \multicolumn{1}{l}{-0.114} & \multicolumn{1}{l}{-0.170} & \multicolumn{1}{l}{0.239} & \multicolumn{1}{l}{0.215} \\
      & \multicolumn{1}{l}{(1.171)} & \multicolumn{1}{l}{(1.113)} & \multicolumn{1}{l}{(0.260)} & \multicolumn{1}{l}{(0.289)} & \multicolumn{1}{l}{(0.195)} & \multicolumn{1}{l}{(0.175)} & \multicolumn{1}{l}{(0.131)} & \multicolumn{1}{l}{(0.129)} & \multicolumn{1}{l}{(0.368)} & \multicolumn{1}{l}{(0.370)} & \multicolumn{1}{l}{(0.153)} & \multicolumn{1}{l}{(0.143)} \\
Public Lawyer & \multicolumn{1}{l}{} & \multicolumn{1}{l}{1.261} & \multicolumn{1}{l}{} & \multicolumn{1}{l}{-0.252} & \multicolumn{1}{l}{} & \multicolumn{1}{l}{-1.908***} & \multicolumn{1}{l}{} & \multicolumn{1}{l}{0.443} & \multicolumn{1}{l}{} & \multicolumn{1}{l}{0.0298} & \multicolumn{1}{l}{} & \multicolumn{1}{l}{0.417} \\
      & \multicolumn{1}{l}{} & \multicolumn{1}{l}{(3.393)} & \multicolumn{1}{l}{} & \multicolumn{1}{l}{(0.294)} & \multicolumn{1}{l}{} & \multicolumn{1}{l}{(0.548)} & \multicolumn{1}{l}{} & \multicolumn{1}{l}{(0.306)} & \multicolumn{1}{l}{} & \multicolumn{1}{l}{(0.521)} & \multicolumn{1}{l}{} & \multicolumn{1}{l}{(0.388)} \\
Female & \multicolumn{1}{l}{} & \multicolumn{1}{l}{4.287} & \multicolumn{1}{l}{} & \multicolumn{1}{l}{0.356} & \multicolumn{1}{l}{} & \multicolumn{1}{l}{1.138} & \multicolumn{1}{l}{} & \multicolumn{1}{l}{-0.0831} & \multicolumn{1}{l}{} & \multicolumn{1}{l}{1.392**} & \multicolumn{1}{l}{} & \multicolumn{1}{l}{0.00616} \\
      & \multicolumn{1}{l}{} & \multicolumn{1}{l}{(4.334)} & \multicolumn{1}{l}{} & \multicolumn{1}{l}{(0.291)} & \multicolumn{1}{l}{} & \multicolumn{1}{l}{(0.937)} & \multicolumn{1}{l}{} & \multicolumn{1}{l}{(0.211)} & \multicolumn{1}{l}{} & \multicolumn{1}{l}{(0.601)} & \multicolumn{1}{l}{} & \multicolumn{1}{l}{(0.214)} \\
At will worker & \multicolumn{1}{l}{} & \multicolumn{1}{l}{-4.455} & \multicolumn{1}{l}{} & \multicolumn{1}{l}{-0.0237} & \multicolumn{1}{l}{} & \multicolumn{1}{l}{-2.363**} & \multicolumn{1}{l}{} & \multicolumn{1}{l}{0.169} & \multicolumn{1}{l}{} & \multicolumn{1}{l}{-1.871***} & \multicolumn{1}{l}{} & \multicolumn{1}{l}{0.641**} \\
      & \multicolumn{1}{l}{} & \multicolumn{1}{l}{(3.807)} & \multicolumn{1}{l}{} & \multicolumn{1}{l}{(0.511)} & \multicolumn{1}{l}{} & \multicolumn{1}{l}{(0.927)} & \multicolumn{1}{l}{} & \multicolumn{1}{l}{(0.273)} & \multicolumn{1}{l}{} & \multicolumn{1}{l}{(0.586)} & \multicolumn{1}{l}{} & \multicolumn{1}{l}{(0.307)} \\
Tenure & \multicolumn{1}{l}{} & \multicolumn{1}{l}{0.00845} & \multicolumn{1}{l}{} & \multicolumn{1}{l}{-0.0103} & \multicolumn{1}{l}{} & \multicolumn{1}{l}{-0.0874*} & \multicolumn{1}{l}{} & \multicolumn{1}{l}{-0.0108} & \multicolumn{1}{l}{} & \multicolumn{1}{l}{0.00307} & \multicolumn{1}{l}{} & \multicolumn{1}{l}{-0.00290} \\
      & \multicolumn{1}{l}{} & \multicolumn{1}{l}{(0.116)} & \multicolumn{1}{l}{} & \multicolumn{1}{l}{(0.0193)} & \multicolumn{1}{l}{} & \multicolumn{1}{l}{(0.0449)} & \multicolumn{1}{l}{} & \multicolumn{1}{l}{(0.0133)} & \multicolumn{1}{l}{} & \multicolumn{1}{l}{(0.0501)} & \multicolumn{1}{l}{} & \multicolumn{1}{l}{(0.0169)} \\
Daily wage & \multicolumn{1}{l}{} & \multicolumn{1}{l}{0.000177} & \multicolumn{1}{l}{} & \multicolumn{1}{l}{0.0000120} & \multicolumn{1}{l}{} & \multicolumn{1}{l}{-0.0000586} & \multicolumn{1}{l}{} & \multicolumn{1}{l}{0.0000373} & \multicolumn{1}{l}{} & \multicolumn{1}{l}{0.000250} & \multicolumn{1}{l}{} & \multicolumn{1}{l}{-0.0000500*} \\
      & \multicolumn{1}{l}{} & \multicolumn{1}{l}{(0.000246)} & \multicolumn{1}{l}{} & \multicolumn{1}{l}{(0.0000681)} & \multicolumn{1}{l}{} & \multicolumn{1}{l}{(0.000199)} & \multicolumn{1}{l}{} & \multicolumn{1}{l}{(0.0000464)} & \multicolumn{1}{l}{} & \multicolumn{1}{l}{(0.000199)} & \multicolumn{1}{l}{} & \multicolumn{1}{l}{(0.0000261)} \\
Weekly hours & \multicolumn{1}{l}{} & \multicolumn{1}{l}{0.133} & \multicolumn{1}{l}{} & \multicolumn{1}{l}{-0.00146} & \multicolumn{1}{l}{} & \multicolumn{1}{l}{0.0847***} & \multicolumn{1}{l}{} & \multicolumn{1}{l}{0.00627} & \multicolumn{1}{l}{} & \multicolumn{1}{l}{0.0158} & \multicolumn{1}{l}{} & \multicolumn{1}{l}{0.00201} \\
      & \multicolumn{1}{l}{} & \multicolumn{1}{l}{(0.0958)} & \multicolumn{1}{l}{} & \multicolumn{1}{l}{(0.00854)} & \multicolumn{1}{l}{} & \multicolumn{1}{l}{(0.0304)} & \multicolumn{1}{l}{} & \multicolumn{1}{l}{(0.00605)} & \multicolumn{1}{l}{} & \multicolumn{1}{l}{(0.0168)} & \multicolumn{1}{l}{} & \multicolumn{1}{l}{(0.00430)} \\
      & \multicolumn{1}{l}{} & \multicolumn{1}{l}{} & \multicolumn{1}{l}{} & \multicolumn{1}{l}{} & \multicolumn{1}{l}{} & \multicolumn{1}{l}{} & \multicolumn{1}{l}{} & \multicolumn{1}{l}{} & \multicolumn{1}{l}{} & \multicolumn{1}{l}{} & \multicolumn{1}{l}{} & \multicolumn{1}{l}{} \\
       \midrule
Observations & \multicolumn{1}{l}{86} & \multicolumn{1}{l}{86} & \multicolumn{1}{l}{83} & \multicolumn{1}{l}{83} & \multicolumn{1}{l}{239} & \multicolumn{1}{l}{239} & \multicolumn{1}{l}{239} & \multicolumn{1}{l}{239} & \multicolumn{1}{l}{216} & \multicolumn{1}{l}{216} & \multicolumn{1}{l}{216} & \multicolumn{1}{l}{216} \\
R-squared & \multicolumn{1}{l}{0.305} & \multicolumn{1}{l}{0.342} & \multicolumn{1}{l}{0.328} & \multicolumn{1}{l}{0.348} & \multicolumn{1}{l}{0.472} & \multicolumn{1}{l}{0.526} & \multicolumn{1}{l}{0.199} & \multicolumn{1}{l}{0.213} & \multicolumn{1}{l}{0.268} & \multicolumn{1}{l}{0.328} & \multicolumn{1}{l}{0.125} & \multicolumn{1}{l}{0.155} \\
Dep Var Mean & \multicolumn{2}{c}{5.418} & \multicolumn{2}{c}{1.091} & \multicolumn{2}{c}{3.049} & \multicolumn{2}{c}{1.738} & \multicolumn{2}{c}{1.072} & \multicolumn{2}{c}{1.664} \\
\bottomrule
\end{tabular}%
}
 \end{center}
  \footnotesize
 \textit{Notes:} 
 Dependent variable are expectations measured in the exit survey (for amount and delay respectively). Specification (2) includes basic controls.
 \textit{Do file: } \texttt{exit\_vs\_initial\_exp\_rel.do}
 \end{table}

 \begin{table}[H]
 \caption{Belief updating (relative) winsorizing at 90 percentile}
 \label{Belief_updating}
 \begin{center}
 \scriptsize{% Table generated by Excel2LaTeX from sheet 'exit_vs_initial_exp'
\begin{tabular}{rrrrrrrrrrrrr}
\toprule
      & \multicolumn{4}{c}{Employee}  & \multicolumn{4}{c}{Employee's Lawyer} & \multicolumn{4}{c}{Firm's Lawyer} \\
\midrule
      & \multicolumn{2}{c}{Amount} & \multicolumn{2}{c}{Time} & \multicolumn{2}{c}{Amount} & \multicolumn{2}{c}{Time} & \multicolumn{2}{c}{Amount} & \multicolumn{2}{c}{Time} \\
      & \multicolumn{1}{c}{(1)} & \multicolumn{1}{c}{(2)} & \multicolumn{1}{c}{(1)} & \multicolumn{1}{c}{(2)} & \multicolumn{1}{c}{(1)} & \multicolumn{1}{c}{(2)} & \multicolumn{1}{c}{(1)} & \multicolumn{1}{c}{(2)} & \multicolumn{1}{c}{(1)} & \multicolumn{1}{c}{(2)} & \multicolumn{1}{c}{(1)} & \multicolumn{1}{c}{(2)} \\
Initial survey var (ISV) & \multicolumn{1}{l}{0.729***} & \multicolumn{1}{l}{0.673***} & \multicolumn{1}{l}{0.346**} & \multicolumn{1}{l}{0.352**} & \multicolumn{1}{l}{0.523***} & \multicolumn{1}{l}{0.476***} & \multicolumn{1}{l}{0.499***} & \multicolumn{1}{l}{0.463***} & \multicolumn{1}{l}{0.376***} & \multicolumn{1}{l}{0.332***} & \multicolumn{1}{l}{0.125} & \multicolumn{1}{l}{0.0962} \\
      & \multicolumn{1}{l}{(0.110)} & \multicolumn{1}{l}{(0.128)} & \multicolumn{1}{l}{(0.169)} & \multicolumn{1}{l}{(0.176)} & \multicolumn{1}{l}{(0.0642)} & \multicolumn{1}{l}{(0.0701)} & \multicolumn{1}{l}{(0.0664)} & \multicolumn{1}{l}{(0.0780)} & \multicolumn{1}{l}{(0.106)} & \multicolumn{1}{l}{(0.105)} & \multicolumn{1}{l}{(0.101)} & \multicolumn{1}{l}{(0.0994)} \\
Control & \multicolumn{1}{l}{0.978**} & \multicolumn{1}{l}{2.031} & \multicolumn{1}{l}{0.452*} & \multicolumn{1}{l}{0.536} & \multicolumn{1}{l}{0.445**} & \multicolumn{1}{l}{-0.493} & \multicolumn{1}{l}{0.489**} & \multicolumn{1}{l}{0.264} & \multicolumn{1}{l}{-0.209*} & \multicolumn{1}{l}{-0.485} & \multicolumn{1}{l}{1.139***} & \multicolumn{1}{l}{1.031***} \\
      & \multicolumn{1}{l}{(0.462)} & \multicolumn{1}{l}{(1.277)} & \multicolumn{1}{l}{(0.259)} & \multicolumn{1}{l}{(0.567)} & \multicolumn{1}{l}{(0.215)} & \multicolumn{1}{l}{(0.697)} & \multicolumn{1}{l}{(0.214)} & \multicolumn{1}{l}{(0.436)} & \multicolumn{1}{l}{(0.109)} & \multicolumn{1}{l}{(0.327)} & \multicolumn{1}{l}{(0.293)} & \multicolumn{1}{l}{(0.380)} \\
Calculator & \multicolumn{1}{l}{-1.195**} & \multicolumn{1}{l}{-1.183**} & \multicolumn{1}{l}{-0.340} & \multicolumn{1}{l}{-0.367} & \multicolumn{1}{l}{-0.197} & \multicolumn{1}{l}{-0.322} & \multicolumn{1}{l}{0.0759} & \multicolumn{1}{l}{0.111} & \multicolumn{1}{l}{0.0348} & \multicolumn{1}{l}{0.0591} & \multicolumn{1}{l}{-0.0687} & \multicolumn{1}{l}{-0.109} \\
      & \multicolumn{1}{l}{(0.485)} & \multicolumn{1}{l}{(0.495)} & \multicolumn{1}{l}{(0.386)} & \multicolumn{1}{l}{(0.414)} & \multicolumn{1}{l}{(0.338)} & \multicolumn{1}{l}{(0.327)} & \multicolumn{1}{l}{(0.344)} & \multicolumn{1}{l}{(0.358)} & \multicolumn{1}{l}{(0.182)} & \multicolumn{1}{l}{(0.185)} & \multicolumn{1}{l}{(0.448)} & \multicolumn{1}{l}{(0.443)} \\
Conciliator & \multicolumn{1}{l}{-0.685} & \multicolumn{1}{l}{-0.774} & \multicolumn{1}{l}{0.248} & \multicolumn{1}{l}{0.290} & \multicolumn{1}{l}{0.131} & \multicolumn{1}{l}{-0.0381} & \multicolumn{1}{l}{0.501} & \multicolumn{1}{l}{0.495} & \multicolumn{1}{l}{0.0691} & \multicolumn{1}{l}{0.117} & \multicolumn{1}{l}{-0.600} & \multicolumn{1}{l}{-0.561} \\
      & \multicolumn{1}{l}{(0.748)} & \multicolumn{1}{l}{(0.756)} & \multicolumn{1}{l}{(0.565)} & \multicolumn{1}{l}{(0.605)} & \multicolumn{1}{l}{(0.355)} & \multicolumn{1}{l}{(0.359)} & \multicolumn{1}{l}{(0.364)} & \multicolumn{1}{l}{(0.366)} & \multicolumn{1}{l}{(0.165)} & \multicolumn{1}{l}{(0.169)} & \multicolumn{1}{l}{(0.384)} & \multicolumn{1}{l}{(0.370)} \\
Calculator\#\#ISV & \multicolumn{1}{l}{0.181} & \multicolumn{1}{l}{0.222} & \multicolumn{1}{l}{0.497**} & \multicolumn{1}{l}{0.476**} & \multicolumn{1}{l}{-0.0419} & \multicolumn{1}{l}{-0.0311} & \multicolumn{1}{l}{-0.0900} & \multicolumn{1}{l}{-0.0906} & \multicolumn{1}{l}{-0.0558} & \multicolumn{1}{l}{-0.0597} & \multicolumn{1}{l}{0.153} & \multicolumn{1}{l}{0.172} \\
      & \multicolumn{1}{l}{(0.126)} & \multicolumn{1}{l}{(0.152)} & \multicolumn{1}{l}{(0.213)} & \multicolumn{1}{l}{(0.230)} & \multicolumn{1}{l}{(0.106)} & \multicolumn{1}{l}{(0.0963)} & \multicolumn{1}{l}{(0.111)} & \multicolumn{1}{l}{(0.117)} & \multicolumn{1}{l}{(0.159)} & \multicolumn{1}{l}{(0.151)} & \multicolumn{1}{l}{(0.143)} & \multicolumn{1}{l}{(0.139)} \\
Conciliator\#\#ISV & \multicolumn{1}{l}{-0.179} & \multicolumn{1}{l}{-0.184} & \multicolumn{1}{l}{0.0724} & \multicolumn{1}{l}{0.0533} & \multicolumn{1}{l}{-0.182} & \multicolumn{1}{l}{-0.163} & \multicolumn{1}{l}{-0.316**} & \multicolumn{1}{l}{-0.307**} & \multicolumn{1}{l}{-0.0535} & \multicolumn{1}{l}{-0.0616} & \multicolumn{1}{l}{0.215} & \multicolumn{1}{l}{0.192} \\
      & \multicolumn{1}{l}{(0.178)} & \multicolumn{1}{l}{(0.181)} & \multicolumn{1}{l}{(0.285)} & \multicolumn{1}{l}{(0.317)} & \multicolumn{1}{l}{(0.135)} & \multicolumn{1}{l}{(0.133)} & \multicolumn{1}{l}{(0.128)} & \multicolumn{1}{l}{(0.129)} & \multicolumn{1}{l}{(0.151)} & \multicolumn{1}{l}{(0.147)} & \multicolumn{1}{l}{(0.142)} & \multicolumn{1}{l}{(0.133)} \\
Public Lawyer & \multicolumn{1}{l}{} & \multicolumn{1}{l}{-1.055**} & \multicolumn{1}{l}{} & \multicolumn{1}{l}{-0.219} & \multicolumn{1}{l}{} & \multicolumn{1}{l}{-1.312***} & \multicolumn{1}{l}{} & \multicolumn{1}{l}{0.401} & \multicolumn{1}{l}{} & \multicolumn{1}{l}{-0.0755} & \multicolumn{1}{l}{} & \multicolumn{1}{l}{0.363} \\
      & \multicolumn{1}{l}{} & \multicolumn{1}{l}{(0.449)} & \multicolumn{1}{l}{} & \multicolumn{1}{l}{(0.285)} & \multicolumn{1}{l}{} & \multicolumn{1}{l}{(0.396)} & \multicolumn{1}{l}{} & \multicolumn{1}{l}{(0.259)} & \multicolumn{1}{l}{} & \multicolumn{1}{l}{(0.218)} & \multicolumn{1}{l}{} & \multicolumn{1}{l}{(0.329)} \\
Female & \multicolumn{1}{l}{} & \multicolumn{1}{l}{0.0747} & \multicolumn{1}{l}{} & \multicolumn{1}{l}{0.384} & \multicolumn{1}{l}{} & \multicolumn{1}{l}{1.035**} & \multicolumn{1}{l}{} & \multicolumn{1}{l}{-0.0888} & \multicolumn{1}{l}{} & \multicolumn{1}{l}{0.540***} & \multicolumn{1}{l}{} & \multicolumn{1}{l}{-0.0342} \\
      & \multicolumn{1}{l}{} & \multicolumn{1}{l}{(0.517)} & \multicolumn{1}{l}{} & \multicolumn{1}{l}{(0.285)} & \multicolumn{1}{l}{} & \multicolumn{1}{l}{(0.427)} & \multicolumn{1}{l}{} & \multicolumn{1}{l}{(0.192)} & \multicolumn{1}{l}{} & \multicolumn{1}{l}{(0.173)} & \multicolumn{1}{l}{} & \multicolumn{1}{l}{(0.186)} \\
At will worker & \multicolumn{1}{l}{} & \multicolumn{1}{l}{-0.0961} & \multicolumn{1}{l}{} & \multicolumn{1}{l}{-0.00293} & \multicolumn{1}{l}{} & \multicolumn{1}{l}{-1.304***} & \multicolumn{1}{l}{} & \multicolumn{1}{l}{0.118} & \multicolumn{1}{l}{} & \multicolumn{1}{l}{-0.723***} & \multicolumn{1}{l}{} & \multicolumn{1}{l}{0.561**} \\
      & \multicolumn{1}{l}{} & \multicolumn{1}{l}{(0.503)} & \multicolumn{1}{l}{} & \multicolumn{1}{l}{(0.473)} & \multicolumn{1}{l}{} & \multicolumn{1}{l}{(0.396)} & \multicolumn{1}{l}{} & \multicolumn{1}{l}{(0.249)} & \multicolumn{1}{l}{} & \multicolumn{1}{l}{(0.181)} & \multicolumn{1}{l}{} & \multicolumn{1}{l}{(0.261)} \\
Tenure & \multicolumn{1}{l}{} & \multicolumn{1}{l}{-0.0182} & \multicolumn{1}{l}{} & \multicolumn{1}{l}{-0.00927} & \multicolumn{1}{l}{} & \multicolumn{1}{l}{-0.0389**} & \multicolumn{1}{l}{} & \multicolumn{1}{l}{-0.00731} & \multicolumn{1}{l}{} & \multicolumn{1}{l}{-0.0115} & \multicolumn{1}{l}{} & \multicolumn{1}{l}{-0.00373} \\
      & \multicolumn{1}{l}{} & \multicolumn{1}{l}{(0.0192)} & \multicolumn{1}{l}{} & \multicolumn{1}{l}{(0.0185)} & \multicolumn{1}{l}{} & \multicolumn{1}{l}{(0.0188)} & \multicolumn{1}{l}{} & \multicolumn{1}{l}{(0.0129)} & \multicolumn{1}{l}{} & \multicolumn{1}{l}{(0.0118)} & \multicolumn{1}{l}{} & \multicolumn{1}{l}{(0.0154)} \\
Daily wage & \multicolumn{1}{l}{} & \multicolumn{1}{l}{-0.0000977} & \multicolumn{1}{l}{} & \multicolumn{1}{l}{0.0000183} & \multicolumn{1}{l}{} & \multicolumn{1}{l}{-0.000153*} & \multicolumn{1}{l}{} & \multicolumn{1}{l}{0.0000449} & \multicolumn{1}{l}{} & \multicolumn{1}{l}{0.0000328} & \multicolumn{1}{l}{} & \multicolumn{1}{l}{-0.0000509**} \\
      & \multicolumn{1}{l}{} & \multicolumn{1}{l}{(0.0000783)} & \multicolumn{1}{l}{} & \multicolumn{1}{l}{(0.0000616)} & \multicolumn{1}{l}{} & \multicolumn{1}{l}{(0.0000882)} & \multicolumn{1}{l}{} & \multicolumn{1}{l}{(0.0000431)} & \multicolumn{1}{l}{} & \multicolumn{1}{l}{(0.0000505)} & \multicolumn{1}{l}{} & \multicolumn{1}{l}{(0.0000216)} \\
Weekly hours & \multicolumn{1}{l}{} & \multicolumn{1}{l}{-0.00982} & \multicolumn{1}{l}{} & \multicolumn{1}{l}{-0.00194} & \multicolumn{1}{l}{} & \multicolumn{1}{l}{0.0250**} & \multicolumn{1}{l}{} & \multicolumn{1}{l}{0.00434} & \multicolumn{1}{l}{} & \multicolumn{1}{l}{0.00311} & \multicolumn{1}{l}{} & \multicolumn{1}{l}{0.00284} \\
      & \multicolumn{1}{l}{} & \multicolumn{1}{l}{(0.0154)} & \multicolumn{1}{l}{} & \multicolumn{1}{l}{(0.00837)} & \multicolumn{1}{l}{} & \multicolumn{1}{l}{(0.0120)} & \multicolumn{1}{l}{} & \multicolumn{1}{l}{(0.00550)} & \multicolumn{1}{l}{} & \multicolumn{1}{l}{(0.00499)} & \multicolumn{1}{l}{} & \multicolumn{1}{l}{(0.00381)} \\
      & \multicolumn{1}{l}{} & \multicolumn{1}{l}{} & \multicolumn{1}{l}{} & \multicolumn{1}{l}{} & \multicolumn{1}{l}{} & \multicolumn{1}{l}{} & \multicolumn{1}{l}{} & \multicolumn{1}{l}{} & \multicolumn{1}{l}{} & \multicolumn{1}{l}{} & \multicolumn{1}{l}{} & \multicolumn{1}{l}{} \\
       \midrule
Observations & \multicolumn{1}{l}{86} & \multicolumn{1}{l}{86} & \multicolumn{1}{l}{83} & \multicolumn{1}{l}{83} & \multicolumn{1}{l}{239} & \multicolumn{1}{l}{239} & \multicolumn{1}{l}{239} & \multicolumn{1}{l}{239} & \multicolumn{1}{l}{216} & \multicolumn{1}{l}{216} & \multicolumn{1}{l}{216} & \multicolumn{1}{l}{216} \\
R-squared & \multicolumn{1}{l}{0.595} & \multicolumn{1}{l}{0.619} & \multicolumn{1}{l}{0.304} & \multicolumn{1}{l}{0.330} & \multicolumn{1}{l}{0.492} & \multicolumn{1}{l}{0.555} & \multicolumn{1}{l}{0.216} & \multicolumn{1}{l}{0.228} & \multicolumn{1}{l}{0.235} & \multicolumn{1}{l}{0.310} & \multicolumn{1}{l}{0.122} & \multicolumn{1}{l}{0.154} \\
Dep Var Mean & \multicolumn{2}{c}{1.546} & \multicolumn{2}{c}{1.029} & \multicolumn{2}{c}{1.693} & \multicolumn{2}{c}{1.667} & \multicolumn{2}{c}{0.00580} & \multicolumn{2}{c}{1.571} \\
\bottomrule
\end{tabular}%
}
 \end{center}
  \footnotesize
 \textit{Notes:} 
 Dependent variable are expectations measured in the exit survey (for amount and delay respectively). Specification (2) includes basic controls.
 \textit{Do file: } \texttt{exit\_vs\_initial\_exp\_rel.do}
 \end{table}


 \begin{table}[H]
 \caption{Belief updating (relative) (amounts in log)}
 \label{Belief_updating_log}
 \begin{center}
 \scriptsize{% Table generated by Excel2LaTeX from sheet 'exit_vs_initial_exp_log'
\begin{tabular}{rrrrrrrrrrrrr}
\toprule
      & \multicolumn{4}{c}{Employee}  & \multicolumn{4}{c}{Employee's Lawyer} & \multicolumn{4}{c}{Firm's Lawyer} \\
\midrule
      & \multicolumn{2}{c}{Amount} & \multicolumn{2}{c}{Time} & \multicolumn{2}{c}{Amount} & \multicolumn{2}{c}{Time} & \multicolumn{2}{c}{Amount} & \multicolumn{2}{c}{Time} \\
      & \multicolumn{1}{c}{(1)} & \multicolumn{1}{c}{(2)} & \multicolumn{1}{c}{(1)} & \multicolumn{1}{c}{(2)} & \multicolumn{1}{c}{(1)} & \multicolumn{1}{c}{(2)} & \multicolumn{1}{c}{(1)} & \multicolumn{1}{c}{(2)} & \multicolumn{1}{c}{(1)} & \multicolumn{1}{c}{(2)} & \multicolumn{1}{c}{(1)} & \multicolumn{1}{c}{(2)} \\
Initial survey var (ISV) & \multicolumn{1}{l}{0.972***} & \multicolumn{1}{l}{0.972***} & \multicolumn{1}{l}{0.178*} & \multicolumn{1}{l}{0.199**} & \multicolumn{1}{l}{0.930***} & \multicolumn{1}{l}{0.927***} & \multicolumn{1}{l}{0.348***} & \multicolumn{1}{l}{0.303***} & \multicolumn{1}{l}{0.838***} & \multicolumn{1}{l}{0.832***} & \multicolumn{1}{l}{0.0850} & \multicolumn{1}{l}{0.0724} \\
      & \multicolumn{1}{l}{(0.000000379)} & \multicolumn{1}{l}{(0.00000134)} & \multicolumn{1}{l}{(0.0908)} & \multicolumn{1}{l}{(0.0902)} & \multicolumn{1}{l}{(0.00767)} & \multicolumn{1}{l}{(0.00662)} & \multicolumn{1}{l}{(0.0725)} & \multicolumn{1}{l}{(0.0793)} & \multicolumn{1}{l}{(0.122)} & \multicolumn{1}{l}{(0.120)} & \multicolumn{1}{l}{(0.100)} & \multicolumn{1}{l}{(0.0990)} \\
Control & \multicolumn{1}{l}{-0.0272***} & \multicolumn{1}{l}{-0.0272***} & \multicolumn{1}{l}{0.626**} & \multicolumn{1}{l}{0.707} & \multicolumn{1}{l}{-0.0700***} & \multicolumn{1}{l}{-0.0735***} & \multicolumn{1}{l}{0.909***} & \multicolumn{1}{l}{0.643} & \multicolumn{1}{l}{-0.162} & \multicolumn{1}{l}{-0.169} & \multicolumn{1}{l}{1.373***} & \multicolumn{1}{l}{1.289**} \\
      & \multicolumn{1}{l}{(0.000210)} & \multicolumn{1}{l}{(0.000283)} & \multicolumn{1}{l}{(0.256)} & \multicolumn{1}{l}{(0.604)} & \multicolumn{1}{l}{(0.00767)} & \multicolumn{1}{l}{(0.00661)} & \multicolumn{1}{l}{(0.258)} & \multicolumn{1}{l}{(0.554)} & \multicolumn{1}{l}{(0.122)} & \multicolumn{1}{l}{(0.120)} & \multicolumn{1}{l}{(0.331)} & \multicolumn{1}{l}{(0.518)} \\
Calculator & \multicolumn{1}{l}{-0.000311} & \multicolumn{1}{l}{-0.000334} & \multicolumn{1}{l}{-0.622*} & \multicolumn{1}{l}{-0.613} & \multicolumn{1}{l}{0.134*} & \multicolumn{1}{l}{0.104} & \multicolumn{1}{l}{0.0348} & \multicolumn{1}{l}{0.0143} & \multicolumn{1}{l}{0.166} & \multicolumn{1}{l}{0.166} & \multicolumn{1}{l}{-0.308} & \multicolumn{1}{l}{-0.309} \\
      & \multicolumn{1}{l}{(0.000222)} & \multicolumn{1}{l}{(0.000250)} & \multicolumn{1}{l}{(0.355)} & \multicolumn{1}{l}{(0.390)} & \multicolumn{1}{l}{(0.0757)} & \multicolumn{1}{l}{(0.0769)} & \multicolumn{1}{l}{(0.394)} & \multicolumn{1}{l}{(0.399)} & \multicolumn{1}{l}{(0.125)} & \multicolumn{1}{l}{(0.124)} & \multicolumn{1}{l}{(0.523)} & \multicolumn{1}{l}{(0.525)} \\
Conciliator & \multicolumn{1}{l}{-0.834***} & \multicolumn{1}{l}{-0.866***} & \multicolumn{1}{l}{0.166} & \multicolumn{1}{l}{0.251} & \multicolumn{1}{l}{0.143***} & \multicolumn{1}{l}{0.135***} & \multicolumn{1}{l}{0.122} & \multicolumn{1}{l}{0.108} & \multicolumn{1}{l}{-0.0494} & \multicolumn{1}{l}{-0.0680} & \multicolumn{1}{l}{-0.803*} & \multicolumn{1}{l}{-0.728*} \\
      & \multicolumn{1}{l}{(0.189)} & \multicolumn{1}{l}{(0.193)} & \multicolumn{1}{l}{(0.614)} & \multicolumn{1}{l}{(0.662)} & \multicolumn{1}{l}{(0.0104)} & \multicolumn{1}{l}{(0.0131)} & \multicolumn{1}{l}{(0.421)} & \multicolumn{1}{l}{(0.416)} & \multicolumn{1}{l}{(0.254)} & \multicolumn{1}{l}{(0.260)} & \multicolumn{1}{l}{(0.442)} & \multicolumn{1}{l}{(0.433)} \\
Calculator\#\#ISV & \multicolumn{1}{l}{0} & \multicolumn{1}{l}{0} & \multicolumn{1}{l}{0.755***} & \multicolumn{1}{l}{0.716***} & \multicolumn{1}{l}{0.134*} & \multicolumn{1}{l}{0.104} & \multicolumn{1}{l}{-0.0929} & \multicolumn{1}{l}{-0.0744} & \multicolumn{1}{l}{0.166} & \multicolumn{1}{l}{0.166} & \multicolumn{1}{l}{0.267} & \multicolumn{1}{l}{0.268} \\
      & \multicolumn{1}{l}{(.)} & \multicolumn{1}{l}{(.)} & \multicolumn{1}{l}{(0.122)} & \multicolumn{1}{l}{(0.144)} & \multicolumn{1}{l}{(0.0758)} & \multicolumn{1}{l}{(0.0770)} & \multicolumn{1}{l}{(0.130)} & \multicolumn{1}{l}{(0.130)} & \multicolumn{1}{l}{(0.125)} & \multicolumn{1}{l}{(0.124)} & \multicolumn{1}{l}{(0.176)} & \multicolumn{1}{l}{(0.176)} \\
Conciliator\#\#ISV & \multicolumn{1}{l}{-0.834***} & \multicolumn{1}{l}{-0.866***} & \multicolumn{1}{l}{0.216} & \multicolumn{1}{l}{0.159} & \multicolumn{1}{l}{0.143***} & \multicolumn{1}{l}{0.135***} & \multicolumn{1}{l}{-0.142} & \multicolumn{1}{l}{-0.132} & \multicolumn{1}{l}{-0.0495} & \multicolumn{1}{l}{-0.0681} & \multicolumn{1}{l}{0.278*} & \multicolumn{1}{l}{0.243*} \\
      & \multicolumn{1}{l}{(0.189)} & \multicolumn{1}{l}{(0.193)} & \multicolumn{1}{l}{(0.210)} & \multicolumn{1}{l}{(0.238)} & \multicolumn{1}{l}{(0.0104)} & \multicolumn{1}{l}{(0.0131)} & \multicolumn{1}{l}{(0.141)} & \multicolumn{1}{l}{(0.139)} & \multicolumn{1}{l}{(0.254)} & \multicolumn{1}{l}{(0.260)} & \multicolumn{1}{l}{(0.155)} & \multicolumn{1}{l}{(0.146)} \\
Public Lawyer & \multicolumn{1}{l}{} & \multicolumn{1}{l}{-0.000226} & \multicolumn{1}{l}{} & \multicolumn{1}{l}{-0.288} & \multicolumn{1}{l}{} & \multicolumn{1}{l}{-0.000132**} & \multicolumn{1}{l}{} & \multicolumn{1}{l}{0.500} & \multicolumn{1}{l}{} & \multicolumn{1}{l}{0.0000300} & \multicolumn{1}{l}{} & \multicolumn{1}{l}{0.392} \\
      & \multicolumn{1}{l}{} & \multicolumn{1}{l}{(0.000154)} & \multicolumn{1}{l}{} & \multicolumn{1}{l}{(0.301)} & \multicolumn{1}{l}{} & \multicolumn{1}{l}{(0.0000560)} & \multicolumn{1}{l}{} & \multicolumn{1}{l}{(0.331)} & \multicolumn{1}{l}{} & \multicolumn{1}{l}{(0.000164)} & \multicolumn{1}{l}{} & \multicolumn{1}{l}{(0.422)} \\
Female & \multicolumn{1}{l}{} & \multicolumn{1}{l}{0.000281} & \multicolumn{1}{l}{} & \multicolumn{1}{l}{0.336} & \multicolumn{1}{l}{} & \multicolumn{1}{l}{0.000198**} & \multicolumn{1}{l}{} & \multicolumn{1}{l}{-0.0615} & \multicolumn{1}{l}{} & \multicolumn{1}{l}{-0.00000557} & \multicolumn{1}{l}{} & \multicolumn{1}{l}{0.0498} \\
      & \multicolumn{1}{l}{} & \multicolumn{1}{l}{(0.000362)} & \multicolumn{1}{l}{} & \multicolumn{1}{l}{(0.292)} & \multicolumn{1}{l}{} & \multicolumn{1}{l}{(0.000101)} & \multicolumn{1}{l}{} & \multicolumn{1}{l}{(0.245)} & \multicolumn{1}{l}{} & \multicolumn{1}{l}{(0.000200)} & \multicolumn{1}{l}{} & \multicolumn{1}{l}{(0.239)} \\
At will worker & \multicolumn{1}{l}{} & \multicolumn{1}{l}{-0.000171} & \multicolumn{1}{l}{} & \multicolumn{1}{l}{0.0683} & \multicolumn{1}{l}{} & \multicolumn{1}{l}{-0.000112**} & \multicolumn{1}{l}{} & \multicolumn{1}{l}{0.213} & \multicolumn{1}{l}{} & \multicolumn{1}{l}{-0.000167*} & \multicolumn{1}{l}{} & \multicolumn{1}{l}{0.660**} \\
      & \multicolumn{1}{l}{} & \multicolumn{1}{l}{(0.000126)} & \multicolumn{1}{l}{} & \multicolumn{1}{l}{(0.566)} & \multicolumn{1}{l}{} & \multicolumn{1}{l}{(0.0000480)} & \multicolumn{1}{l}{} & \multicolumn{1}{l}{(0.305)} & \multicolumn{1}{l}{} & \multicolumn{1}{l}{(0.0000947)} & \multicolumn{1}{l}{} & \multicolumn{1}{l}{(0.324)} \\
Tenure & \multicolumn{1}{l}{} & \multicolumn{1}{l}{0.00000332} & \multicolumn{1}{l}{} & \multicolumn{1}{l}{-0.0121} & \multicolumn{1}{l}{} & \multicolumn{1}{l}{-0.00000157} & \multicolumn{1}{l}{} & \multicolumn{1}{l}{-0.0153} & \multicolumn{1}{l}{} & \multicolumn{1}{l}{-0.0000167} & \multicolumn{1}{l}{} & \multicolumn{1}{l}{-0.00111} \\
      & \multicolumn{1}{l}{} & \multicolumn{1}{l}{(0.0000130)} & \multicolumn{1}{l}{} & \multicolumn{1}{l}{(0.0195)} & \multicolumn{1}{l}{} & \multicolumn{1}{l}{(0.00000321)} & \multicolumn{1}{l}{} & \multicolumn{1}{l}{(0.0139)} & \multicolumn{1}{l}{} & \multicolumn{1}{l}{(0.0000107)} & \multicolumn{1}{l}{} & \multicolumn{1}{l}{(0.0190)} \\
Daily wage & \multicolumn{1}{l}{} & \multicolumn{1}{l}{-2.59e-08} & \multicolumn{1}{l}{} & \multicolumn{1}{l}{0.0000155} & \multicolumn{1}{l}{} & \multicolumn{1}{l}{-1.46e-08} & \multicolumn{1}{l}{} & \multicolumn{1}{l}{0.0000205} & \multicolumn{1}{l}{} & \multicolumn{1}{l}{-7.39e-09} & \multicolumn{1}{l}{} & \multicolumn{1}{l}{-0.0000522*} \\
      & \multicolumn{1}{l}{} & \multicolumn{1}{l}{(1.92e-08)} & \multicolumn{1}{l}{} & \multicolumn{1}{l}{(0.0000692)} & \multicolumn{1}{l}{} & \multicolumn{1}{l}{(8.94e-09)} & \multicolumn{1}{l}{} & \multicolumn{1}{l}{(0.0000490)} & \multicolumn{1}{l}{} & \multicolumn{1}{l}{(1.08e-08)} & \multicolumn{1}{l}{} & \multicolumn{1}{l}{(0.0000283)} \\
Weekly hours & \multicolumn{1}{l}{} & \multicolumn{1}{l}{0.000000261} & \multicolumn{1}{l}{} & \multicolumn{1}{l}{-0.00135} & \multicolumn{1}{l}{} & \multicolumn{1}{l}{0.00000313} & \multicolumn{1}{l}{} & \multicolumn{1}{l}{0.00601} & \multicolumn{1}{l}{} & \multicolumn{1}{l}{0.0000103} & \multicolumn{1}{l}{} & \multicolumn{1}{l}{0.000679} \\
      & \multicolumn{1}{l}{} & \multicolumn{1}{l}{(0.00000463)} & \multicolumn{1}{l}{} & \multicolumn{1}{l}{(0.00872)} & \multicolumn{1}{l}{} & \multicolumn{1}{l}{(0.00000199)} & \multicolumn{1}{l}{} & \multicolumn{1}{l}{(0.00659)} & \multicolumn{1}{l}{} & \multicolumn{1}{l}{(0.00000905)} & \multicolumn{1}{l}{} & \multicolumn{1}{l}{(0.00463)} \\
      & \multicolumn{1}{l}{} & \multicolumn{1}{l}{} & \multicolumn{1}{l}{} & \multicolumn{1}{l}{} & \multicolumn{1}{l}{} & \multicolumn{1}{l}{} & \multicolumn{1}{l}{} & \multicolumn{1}{l}{} & \multicolumn{1}{l}{} & \multicolumn{1}{l}{} & \multicolumn{1}{l}{} & \multicolumn{1}{l}{} \\
       \midrule
Observations & \multicolumn{1}{l}{86} & \multicolumn{1}{l}{86} & \multicolumn{1}{l}{83} & \multicolumn{1}{l}{83} & \multicolumn{1}{l}{239} & \multicolumn{1}{l}{239} & \multicolumn{1}{l}{239} & \multicolumn{1}{l}{239} & \multicolumn{1}{l}{216} & \multicolumn{1}{l}{216} & \multicolumn{1}{l}{216} & \multicolumn{1}{l}{216} \\
R-squared & \multicolumn{1}{l}{1.000} & \multicolumn{1}{l}{1.000} & \multicolumn{1}{l}{0.413} & \multicolumn{1}{l}{0.430} & \multicolumn{1}{l}{0.992} & \multicolumn{1}{l}{0.992} & \multicolumn{1}{l}{0.147} & \multicolumn{1}{l}{0.160} & \multicolumn{1}{l}{0.782} & \multicolumn{1}{l}{0.788} & \multicolumn{1}{l}{0.125} & \multicolumn{1}{l}{0.151} \\
Dep Var Mean & \multicolumn{2}{c}{0.895} & \multicolumn{2}{c}{1.126} & \multicolumn{2}{c}{-0.999} & \multicolumn{2}{c}{1.789} & \multicolumn{2}{c}{-0.999} & \multicolumn{2}{c}{1.716} \\
\bottomrule
\end{tabular}%
}
 \end{center}
  \footnotesize
 \textit{Notes:} 
 Dependent variable are expectations measured in the exit survey (for amount (in log pesos) and delay respectively). Specification (2) includes basic controls.
 \textit{Do file: } \texttt{exit\_vs\_initial\_exp\_log\_rel.do}
 \end{table}


\end{landscape}


\subsection{Private Lawyer interaction}
% INTERACTION WITH PRIVATE LAWYER
\begin{table}[H]
    \caption{Treatment Efects by Month}
    \label{Table_effects}
    \begin{center}
    \scriptsize{% Table generated by Excel2LaTeX from sheet 'Minicortes_2_int'
\begin{tabular}{rrrrrrrrrr}
\toprule
      & \multicolumn{9}{c}{Treatment on conciliation} \\
\midrule
      & \multicolumn{2}{c}{Same day} & \multicolumn{2}{c}{June} & \multicolumn{1}{c}{July } & \multicolumn{1}{c}{Aug} & \multicolumn{1}{c}{Sept} & \multicolumn{2}{c}{Oct} \\
\multicolumn{1}{l}{Control} & \multicolumn{1}{l}{0.0505***} & \multicolumn{1}{l}{0.0515***} & \multicolumn{1}{l}{0.0968***} & \multicolumn{1}{l}{0.100***} & \multicolumn{1}{l}{0.116***} & \multicolumn{1}{l}{0.132***} & \multicolumn{1}{l}{0.169***} & \multicolumn{1}{l}{0.185***} & \multicolumn{1}{l}{0.208***} \\
\multicolumn{1}{l}{} & \multicolumn{1}{l}{(0.0113)} & \multicolumn{1}{l}{(0.0134)} & \multicolumn{1}{l}{(0.0153)} & \multicolumn{1}{l}{(0.0184)} & \multicolumn{1}{l}{(0.0166)} & \multicolumn{1}{l}{(0.0176)} & \multicolumn{1}{l}{(0.0195)} & \multicolumn{1}{l}{(0.0202)} & \multicolumn{1}{l}{(0.0248)} \\
\multicolumn{1}{l}{Calculator} & \multicolumn{1}{l}{0.0470**} & \multicolumn{1}{l}{0.0601**} & \multicolumn{1}{l}{0.0497**} & \multicolumn{1}{l}{0.0643**} & \multicolumn{1}{l}{0.0506*} & \multicolumn{1}{l}{0.0457*} & \multicolumn{1}{l}{0.0306} & \multicolumn{1}{l}{0.0455} & \multicolumn{1}{l}{0.0569} \\
\multicolumn{1}{l}{} & \multicolumn{1}{l}{(0.0193)} & \multicolumn{1}{l}{(0.0240)} & \multicolumn{1}{l}{(0.0243)} & \multicolumn{1}{l}{(0.0299)} & \multicolumn{1}{l}{(0.0258)} & \multicolumn{1}{l}{(0.0268)} & \multicolumn{1}{l}{(0.0288)} & \multicolumn{1}{l}{(0.0301)} & \multicolumn{1}{l}{(0.0375)} \\
\multicolumn{1}{l}{Conciliator} & \multicolumn{1}{l}{0.0419**} & \multicolumn{1}{l}{0.0414*} & \multicolumn{1}{l}{0.0472**} & \multicolumn{1}{l}{0.0496*} & \multicolumn{1}{l}{0.0475*} & \multicolumn{1}{l}{0.0503*} & \multicolumn{1}{l}{0.0372} & \multicolumn{1}{l}{0.0455} & \multicolumn{1}{l}{0.0347} \\
\multicolumn{1}{l}{} & \multicolumn{1}{l}{(0.0189)} & \multicolumn{1}{l}{(0.0220)} & \multicolumn{1}{l}{(0.0239)} & \multicolumn{1}{l}{(0.0282)} & \multicolumn{1}{l}{(0.0254)} & \multicolumn{1}{l}{(0.0267)} & \multicolumn{1}{l}{(0.0287)} & \multicolumn{1}{l}{(0.0299)} & \multicolumn{1}{l}{(0.0358)} \\
\multicolumn{1}{l}{Public Lawyer (PL)} & \multicolumn{1}{l}{} & \multicolumn{1}{l}{0.0423} & \multicolumn{1}{l}{} & \multicolumn{1}{l}{0.0287} & \multicolumn{1}{l}{} & \multicolumn{1}{l}{} & \multicolumn{1}{l}{} & \multicolumn{1}{l}{} & \multicolumn{1}{l}{0.0176} \\
\multicolumn{1}{l}{} & \multicolumn{1}{l}{} & \multicolumn{1}{l}{(0.0534)} & \multicolumn{1}{l}{} & \multicolumn{1}{l}{(0.0632)} & \multicolumn{1}{l}{} & \multicolumn{1}{l}{} & \multicolumn{1}{l}{} & \multicolumn{1}{l}{} & \multicolumn{1}{l}{(0.0793)} \\
\multicolumn{1}{l}{Calculator\#\#PL} & \multicolumn{1}{l}{} & \multicolumn{1}{l}{-0.0110} & \multicolumn{1}{l}{} & \multicolumn{1}{l}{-0.0219} & \multicolumn{1}{l}{} & \multicolumn{1}{l}{} & \multicolumn{1}{l}{} & \multicolumn{1}{l}{} & \multicolumn{1}{l}{-0.0541} \\
\multicolumn{1}{l}{} & \multicolumn{1}{l}{} & \multicolumn{1}{l}{(0.0823)} & \multicolumn{1}{l}{} & \multicolumn{1}{l}{(0.0929)} & \multicolumn{1}{l}{} & \multicolumn{1}{l}{} & \multicolumn{1}{l}{} & \multicolumn{1}{l}{} & \multicolumn{1}{l}{(0.110)} \\
\multicolumn{1}{l}{Conciliator\#\#PL} & \multicolumn{1}{l}{} & \multicolumn{1}{l}{0.0921} & \multicolumn{1}{l}{} & \multicolumn{1}{l}{0.0941} & \multicolumn{1}{l}{} & \multicolumn{1}{l}{} & \multicolumn{1}{l}{} & \multicolumn{1}{l}{} & \multicolumn{1}{l}{0.0577} \\
\multicolumn{1}{l}{} & \multicolumn{1}{l}{} & \multicolumn{1}{l}{(0.106)} & \multicolumn{1}{l}{} & \multicolumn{1}{l}{(0.116)} & \multicolumn{1}{l}{} & \multicolumn{1}{l}{} & \multicolumn{1}{l}{} & \multicolumn{1}{l}{} & \multicolumn{1}{l}{(0.130)} \\
\multicolumn{1}{l}{} & \multicolumn{1}{l}{} & \multicolumn{1}{l}{} & \multicolumn{1}{l}{} & \multicolumn{1}{l}{} & \multicolumn{1}{l}{} & \multicolumn{1}{l}{} & \multicolumn{1}{l}{} & \multicolumn{1}{l}{} & \multicolumn{1}{l}{} \\
\midrule
\multicolumn{1}{l}{Observations} & \multicolumn{1}{c}{1103} & \multicolumn{1}{c}{892} & \multicolumn{1}{c}{1095} & \multicolumn{1}{c}{886} & \multicolumn{1}{c}{1095} & \multicolumn{1}{c}{1095} & \multicolumn{1}{c}{1095} & \multicolumn{1}{c}{1095} & \multicolumn{1}{c}{886} \\
\multicolumn{1}{l}{R-squared} & \multicolumn{1}{c}{0.00609} & \multicolumn{1}{c}{0.0141} & \multicolumn{1}{c}{0.00470} & \multicolumn{1}{c}{0.00947} & \multicolumn{1}{c}{0.00429} & \multicolumn{1}{c}{0.00382} & \multicolumn{1}{c}{0.00171} & \multicolumn{1}{c}{0.00275} & \multicolumn{1}{c}{0.00357} \\
\multicolumn{1}{l}{DepVarMean} & \multicolumn{1}{c}{0.0798} & \multicolumn{1}{c}{0.0798} & \multicolumn{1}{c}{0.129} & \multicolumn{1}{c}{0.129} & \multicolumn{1}{c}{0.148} & \multicolumn{1}{c}{0.163} & \multicolumn{1}{c}{0.192} & \multicolumn{1}{c}{0.216} & \multicolumn{1}{c}{0.216} \\
\multicolumn{1}{l}{Calc=Conc} & \multicolumn{1}{c}{0.0268} & \multicolumn{1}{c}{0.0602} & \multicolumn{1}{c}{0.0484} & \multicolumn{1}{c}{0.0790} & \multicolumn{1}{c}{0.0625} & \multicolumn{1}{c}{0.0598} & \multicolumn{1}{c}{0.196} & \multicolumn{1}{c}{0.128} & \multicolumn{1}{c}{0.332} \\
\multicolumn{1}{l}{Calc=Conc=0} & \multicolumn{1}{c}{0.0183} & \multicolumn{1}{c}{0.0246} & \multicolumn{1}{c}{0.0564} & \multicolumn{1}{c}{0.0619} & \multicolumn{1}{c}{0.0762} & \multicolumn{1}{c}{0.104} & \multicolumn{1}{c}{0.377} & \multicolumn{1}{c}{0.206} & \multicolumn{1}{c}{0.302} \\
\bottomrule
\end{tabular}%
}
    \end{center}
    \footnotesize
    \textit{Notes:} 
    Calc=Conc and Calc=Conc=0 shows the p-value for the respective tests.
    \textit{Do file: } \texttt{mini\_cortes\_2\_int.do}
\end{table}


\begin{table}[H]
    \caption{Triple interaction}
    \label{Table_effects}
    \begin{center}
    \scriptsize{% Table generated by Excel2LaTeX from sheet 'Minicortes_3int'
\begin{tabular}{lrrrrrrr}
\toprule
      & \multicolumn{7}{c}{Treatment on conciliation} \\
\midrule
      & \multicolumn{2}{c}{Same day} & \multicolumn{1}{c}{June} & \multicolumn{1}{c}{July } & \multicolumn{1}{c}{Aug} & \multicolumn{1}{c}{Sept} & \multicolumn{1}{c}{Oct} \\
Control & \multicolumn{1}{l}{0.0505***} & \multicolumn{1}{l}{0.0405***} & \multicolumn{1}{l}{0.0868***} & \multicolumn{1}{l}{0.110***} & \multicolumn{1}{l}{0.132***} & \multicolumn{1}{l}{0.187***} & \multicolumn{1}{l}{0.205***} \\
      & \multicolumn{1}{l}{(0.0113)} & \multicolumn{1}{l}{(0.0133)} & \multicolumn{1}{l}{(0.0192)} & \multicolumn{1}{l}{(0.0213)} & \multicolumn{1}{l}{(0.0231)} & \multicolumn{1}{l}{(0.0265)} & \multicolumn{1}{l}{(0.0275)} \\
Calculator & \multicolumn{1}{l}{0.0470**} & \multicolumn{1}{l}{0.0223} & \multicolumn{1}{l}{0.0206} & \multicolumn{1}{l}{0.0172} & \multicolumn{1}{l}{0.00904} & \multicolumn{1}{l}{-0.0116} & \multicolumn{1}{l}{0.00915} \\
      & \multicolumn{1}{l}{(0.0193)} & \multicolumn{1}{l}{(0.0216)} & \multicolumn{1}{l}{(0.0290)} & \multicolumn{1}{l}{(0.0316)} & \multicolumn{1}{l}{(0.0337)} & \multicolumn{1}{l}{(0.0377)} & \multicolumn{1}{l}{(0.0399)} \\
3.treatment & \multicolumn{1}{l}{0.0419**} & \multicolumn{1}{l}{0.0143} & \multicolumn{1}{l}{0.0187} & \multicolumn{1}{l}{0.0212} & \multicolumn{1}{l}{0.0153} & \multicolumn{1}{l}{-0.0142} & \multicolumn{1}{l}{-0.0114} \\
      & \multicolumn{1}{l}{(0.0189)} & \multicolumn{1}{l}{(0.0200)} & \multicolumn{1}{l}{(0.0278)} & \multicolumn{1}{l}{(0.0306)} & \multicolumn{1}{l}{(0.0327)} & \multicolumn{1}{l}{(0.0363)} & \multicolumn{1}{l}{(0.0377)} \\
Public Lawyer (PL) & \multicolumn{1}{l}{} & \multicolumn{1}{l}{-0.0405***} & \multicolumn{1}{l}{-0.0868***} & \multicolumn{1}{l}{-0.110***} & \multicolumn{1}{l}{-0.132***} & \multicolumn{1}{l}{-0.187***} & \multicolumn{1}{l}{-0.150**} \\
      & \multicolumn{1}{l}{} & \multicolumn{1}{l}{(0.0133)} & \multicolumn{1}{l}{(0.0192)} & \multicolumn{1}{l}{(0.0213)} & \multicolumn{1}{l}{(0.0231)} & \multicolumn{1}{l}{(0.0265)} & \multicolumn{1}{l}{(0.0609)} \\
Calculator\#\#PL & \multicolumn{1}{l}{} & \multicolumn{1}{l}{-0.0223} & \multicolumn{1}{l}{-0.0206} & \multicolumn{1}{l}{-0.0172} & \multicolumn{1}{l}{-0.00904} & \multicolumn{1}{l}{0.0672} & \multicolumn{1}{l}{-0.00915} \\
      & \multicolumn{1}{l}{} & \multicolumn{1}{l}{(0.0216)} & \multicolumn{1}{l}{(0.0290)} & \multicolumn{1}{l}{(0.0316)} & \multicolumn{1}{l}{(0.0337)} & \multicolumn{1}{l}{(0.0661)} & \multicolumn{1}{l}{(0.0866)} \\
Conciliator\#\#PL & \multicolumn{1}{l}{} & \multicolumn{1}{l}{0.0524} & \multicolumn{1}{l}{0.0479} & \multicolumn{1}{l}{0.0455} & \multicolumn{1}{l}{0.0514} & \multicolumn{1}{l}{0.148} & \multicolumn{1}{l}{0.0892} \\
      & \multicolumn{1}{l}{} & \multicolumn{1}{l}{(0.0679)} & \multicolumn{1}{l}{(0.0705)} & \multicolumn{1}{l}{(0.0717)} & \multicolumn{1}{l}{(0.0726)} & \multicolumn{1}{l}{(0.0955)} & \multicolumn{1}{l}{(0.110)} \\
Employee Present (EP) & \multicolumn{1}{l}{} & \multicolumn{1}{l}{0.0595} & \multicolumn{1}{l}{0.0732} & \multicolumn{1}{l}{0.0704} & \multicolumn{1}{l}{0.0676} & \multicolumn{1}{l}{0.0128} & \multicolumn{1}{l}{0.0145} \\
      & \multicolumn{1}{l}{} & \multicolumn{1}{l}{(0.0447)} & \multicolumn{1}{l}{(0.0556)} & \multicolumn{1}{l}{(0.0587)} & \multicolumn{1}{l}{(0.0614)} & \multicolumn{1}{l}{(0.0628)} & \multicolumn{1}{l}{(0.0651)} \\
Calculator\#\#EP & \multicolumn{1}{l}{} & \multicolumn{1}{l}{0.219**} & \multicolumn{1}{l}{0.251***} & \multicolumn{1}{l}{0.280***} & \multicolumn{1}{l}{0.291***} & \multicolumn{1}{l}{0.312***} & \multicolumn{1}{l}{0.271***} \\
      & \multicolumn{1}{l}{} & \multicolumn{1}{l}{(0.0864)} & \multicolumn{1}{l}{(0.0960)} & \multicolumn{1}{l}{(0.0987)} & \multicolumn{1}{l}{(0.101)} & \multicolumn{1}{l}{(0.102)} & \multicolumn{1}{l}{(0.104)} \\
Conciliator\#\#EP & \multicolumn{1}{l}{} & \multicolumn{1}{l}{0.188**} & \multicolumn{1}{l}{0.217**} & \multicolumn{1}{l}{0.194**} & \multicolumn{1}{l}{0.227**} & \multicolumn{1}{l}{0.303***} & \multicolumn{1}{l}{0.303***} \\
      & \multicolumn{1}{l}{} & \multicolumn{1}{l}{(0.0848)} & \multicolumn{1}{l}{(0.0956)} & \multicolumn{1}{l}{(0.0978)} & \multicolumn{1}{l}{(0.101)} & \multicolumn{1}{l}{(0.102)} & \multicolumn{1}{l}{(0.104)} \\
PL\#\#EP & \multicolumn{1}{l}{} & \multicolumn{1}{l}{0.155} & \multicolumn{1}{l}{0.234*} & \multicolumn{1}{l}{0.237*} & \multicolumn{1}{l}{0.240*} & \multicolumn{1}{l}{0.449***} & \multicolumn{1}{l}{0.391**} \\
      & \multicolumn{1}{l}{} & \multicolumn{1}{l}{(0.119)} & \multicolumn{1}{l}{(0.140)} & \multicolumn{1}{l}{(0.142)} & \multicolumn{1}{l}{(0.143)} & \multicolumn{1}{l}{(0.153)} & \multicolumn{1}{l}{(0.163)} \\
Calculator\#\#EPL\#\#EP & \multicolumn{1}{l}{} & \multicolumn{1}{l}{-0.139} & \multicolumn{1}{l}{-0.206} & \multicolumn{1}{l}{-0.235} & \multicolumn{1}{l}{-0.246} & \multicolumn{1}{l}{-0.476**} & \multicolumn{1}{l}{-0.321} \\
      & \multicolumn{1}{l}{} & \multicolumn{1}{l}{(0.179)} & \multicolumn{1}{l}{(0.199)} & \multicolumn{1}{l}{(0.200)} & \multicolumn{1}{l}{(0.201)} & \multicolumn{1}{l}{(0.215)} & \multicolumn{1}{l}{(0.225)} \\
Conciliator\#\#EPL\#\#EP & \multicolumn{1}{l}{} & \multicolumn{1}{l}{0.102} & \multicolumn{1}{l}{0.123} & \multicolumn{1}{l}{0.146} & \multicolumn{1}{l}{0.113} & \multicolumn{1}{l}{-0.183} & \multicolumn{1}{l}{-0.128} \\
      & \multicolumn{1}{l}{} & \multicolumn{1}{l}{(0.243)} & \multicolumn{1}{l}{(0.244)} & \multicolumn{1}{l}{(0.245)} & \multicolumn{1}{l}{(0.246)} & \multicolumn{1}{l}{(0.259)} & \multicolumn{1}{l}{(0.266)} \\
      & \multicolumn{1}{l}{} & \multicolumn{1}{l}{} & \multicolumn{1}{l}{} & \multicolumn{1}{l}{} & \multicolumn{1}{l}{} & \multicolumn{1}{l}{} & \multicolumn{1}{l}{} \\
      \midrule
Observations & \multicolumn{1}{c}{1103} & \multicolumn{1}{c}{892} & \multicolumn{1}{c}{886} & \multicolumn{1}{c}{886} & \multicolumn{1}{c}{886} & \multicolumn{1}{c}{886} & \multicolumn{1}{c}{886} \\
R-squared & \multicolumn{1}{c}{0.00609} & \multicolumn{1}{c}{0.116} & \multicolumn{1}{c}{0.110} & \multicolumn{1}{c}{0.0989} & \multicolumn{1}{c}{0.0961} & \multicolumn{1}{c}{0.0798} & \multicolumn{1}{c}{0.0693} \\
DepVarMean & \multicolumn{1}{c}{0.0798} & \multicolumn{1}{c}{0.0798} & \multicolumn{1}{c}{0.129} & \multicolumn{1}{c}{0.148} & \multicolumn{1}{c}{0.163} & \multicolumn{1}{c}{0.192} & \multicolumn{1}{c}{0.216} \\
Calc=Conc & \multicolumn{1}{c}{0.0268} & \multicolumn{1}{c}{0.474} & \multicolumn{1}{c}{0.500} & \multicolumn{1}{c}{0.489} & \multicolumn{1}{c}{0.641} & \multicolumn{1}{c}{0.695} & \multicolumn{1}{c}{0.763} \\
Calc=Conc=0 & \multicolumn{1}{c}{0.0183} & \multicolumn{1}{c}{0.558} & \multicolumn{1}{c}{0.719} & \multicolumn{1}{c}{0.762} & \multicolumn{1}{c}{0.896} & \multicolumn{1}{c}{0.919} & \multicolumn{1}{c}{0.867} \\
\bottomrule
\end{tabular}%
}
    \end{center}
    \footnotesize
    \textit{Notes:} 
    Calc=Conc and Calc=Conc=0 shows the p-value for the respective tests.
    \textit{Do file: } \texttt{mini\_cortes\_3int.do}
\end{table}




\begin{table}[H]
  \centering
  \caption{Tab presence of employee and public lawyer}
\begin{tabular}{rccr}
\toprule
      & \multicolumn{2}{c}{Employee presence} &  \\
\midrule
\multicolumn{1}{c}{Public Lawyer} & 0     & 1     & \multicolumn{1}{c}{Total} \\
\multicolumn{1}{c}{0} & 666   & 137   & \multicolumn{1}{c}{803} \\
\multicolumn{1}{c}{1} & 51    & 38    & \multicolumn{1}{c}{89} \\
\multicolumn{1}{c}{Total} & 717   & 175   & \multicolumn{1}{c}{892} \\
\bottomrule
\end{tabular}%

\end{table}%





\end{document}