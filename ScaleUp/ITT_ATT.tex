\documentclass[12pt]{article}
\usepackage{graphicx}
\usepackage{float}
\usepackage{amsmath}
\usepackage{amscd}
\usepackage{hyperref}
\usepackage{enumerate}
\usepackage{amsfonts}
\usepackage{amssymb}
\usepackage[utf8]{inputenc}
\usepackage{amsthm}
\usepackage{subcaption}
\usepackage{listings}
\usepackage{lscape}
\usepackage{tikz}
\usepackage{color} %red, green, blue, yellow, cyan, magenta, black, white
\usepackage{fullpage}
\usepackage{mathtools}
\usepackage{booktabs}
\usepackage{longtable}

\definecolor{mygreen}{RGB}{28,172,0} % color values Red, Green, Blue
\definecolor{mylilas}{RGB}{170,55,241}

\newtheorem{theorem}{Teorema}
\newtheorem{lem}[theorem]{Lemma}
\newtheorem{dfn}{Definición}
\newtheorem{cor}[theorem]{Corolario}
\newtheorem{obs}{Obs}
\newtheorem{rem}{Remark}
\newtheorem{prob}{Problema}

\newcommand*\circled[1]{\tikz[baseline=(char.base)]{
            \node[shape=circle,draw,inner sep=.05pt] (char) {#1};}}
            

\newtheoremstyle{named}{}{}{\itshape}{}{\bfseries}{.}{.5em} {\thmnote{#3 }#1}
\theoremstyle{named}
\newtheorem*{namedtheorem}{}


\newcounter{exercisecounter}
\newenvironment{ex}{\begin{quote}%
    \refstepcounter{exercisecounter}%
  \textbf{Ejercicio \arabic{exercisecounter}}%
  \quad
}{%
\end{quote}%
}
\newcounter{ejemplocounter}
\newenvironment{ej}{\begin{quote}%
    \refstepcounter{ejemplocounter}%
  \textbf{Ejemplo \arabic{ejemplocounter}}%
  \quad
}{%
\end{quote}%
}

\renewcommand{\d}[1]{\ensuremath{\operatorname{d}\!{#1}}}

\DeclarePairedDelimiter{\ceil}{\lceil}{\rceil}


\newcommand{\folder}{./Effect}

\begin{document}


\title{Treatment effects}

\author{Instituto Tecnológico Autónomo de México}
\date{\today}
\maketitle


\hrulefill


\section{Some Summary Statistics}

\vspace{7mm}

\subsection*{Administrative Stats}

Summary statistics of basic subject characteristics. All variables coming from casefile details are tested for balance between survey and non-survey populations.

\begin{center}
\scriptsize{
\begin{table}[!htbp] \centering 
  \caption{} 
  \label{} 
\begin{tabular}{@{\extracolsep{5pt}} cccccccc} 
\\[-1.8ex]\hline 
\hline \\[-1.8ex] 
Statistic & Treatment & Control & Treatment.1 & Control.1 & t-stat & Treatment.2 & Control.2 \\ 
\hline \\[-1.8ex] 
Wage & $1,522$ & $396$ & $620.698$ & $580.082$ & $0.797$ & $787.098$ & $930.839$ \\ 
Female & $1,522$ & $396$ & $0.445$ & $0.444$ & $0.013$ & $0.497$ & $0.498$ \\ 
Public Lawyer & $1,522$ & $396$ & $0.064$ & $0.071$ & $$-$0.486$ & $0.244$ & $0.257$ \\ 
Tenure & $1,522$ & $396$ & $3.898$ & $4.073$ & $$-$0.555$ & $5.299$ & $5.668$ \\ 
Bought durable goods recently & $170$ & $25$ & $0.082$ & $0.080$ & $$ & $0.276$ & $0.277$ \\ 
Working at the time & $172$ & $24$ & $0.465$ & $0.458$ & $$ & $0.500$ & $0.509$ \\ 
Looking for a job & $168$ & $24$ & $0.571$ & $0.500$ & $$ & $0.496$ & $0.511$ \\ 
\hline \\[-1.8ex] 
\end{tabular} 
\end{table} 
}
\end{center}


\pagebreak

\begin{center}
\scriptsize{% Table generated by Excel2LaTeX from sheet 'TAB_survey'
\begin{tabular}{rr}
\toprule
\multicolumn{2}{c}{Anger employee} \\
\midrule
\midrule
\multicolumn{1}{l}{A lot} & 143 \\
\multicolumn{1}{l}{Fairly} & 41 \\
\multicolumn{1}{l}{Little } & 18 \\
\multicolumn{1}{l}{Nothing} & 37 \\
\midrule
\multicolumn{2}{c}{Education employee} \\
\midrule
\midrule
\multicolumn{1}{l}{Elementary} & 16 \\
\multicolumn{1}{l}{Secondary} & 65 \\
\multicolumn{1}{l}{High-School} & 68 \\
\multicolumn{1}{l}{+ High School} & 90 \\
\midrule
\multicolumn{2}{c}{\#Cases employee's lawyer} \\
\midrule
\midrule
\multicolumn{1}{l}{1-10} & 106 \\
\multicolumn{1}{l}{11-30} & 126 \\
\multicolumn{1}{l}{31-100} & 256 \\
\multicolumn{1}{l}{+ 100} & 490 \\
\midrule
\multicolumn{2}{c}{\#Cases firm's lawyer} \\
\midrule
\midrule
\multicolumn{1}{l}{1-10} & 50 \\
\multicolumn{1}{l}{11-30} & 54 \\
\multicolumn{1}{l}{31-100} & 124 \\
\multicolumn{1}{l}{+ 100} & 368 \\
\bottomrule
\end{tabular}%
}
\end{center}

\pagebreak

\begin{landscape}
\begin{center}
\scriptsize{% Table generated by Excel2LaTeX from sheet 'ITT_ATT_report'
\begin{tabular}{lrrlllrrr}
      & \multicolumn{4}{c}{Conciliation} &       & \multicolumn{1}{c}{Calculator Plaintiff} & \multicolumn{1}{c}{Calculator Defendant} & \multicolumn{1}{c}{Calculator Both} \\
\cmidrule{2-9}      & \multicolumn{1}{c}{ITT} & \multicolumn{1}{c}{ITT} & \multicolumn{1}{c}{ATT (Plaintiff)} & \multicolumn{1}{c}{ATT  (Defendant)} & \multicolumn{1}{c}{ATT (Both)} & \multicolumn{3}{c}{First Stage} \\
\cmidrule{2-9}Treatment  & \multicolumn{1}{l}{0.0377*} & \multicolumn{1}{l}{0.0324***} &       &       &       & \multicolumn{1}{l}{0.679***} & \multicolumn{1}{l}{0.390***} & \multicolumn{1}{l}{0.292***} \\
      & \multicolumn{1}{l}{(0.0182)} & \multicolumn{1}{l}{(0.0109)} &       &       &       & \multicolumn{1}{l}{(0.0163)} & \multicolumn{1}{l}{(0.0173)} & \multicolumn{1}{l}{(0.0195)} \\
Calculator  & \multicolumn{1}{l}{} & \multicolumn{1}{l}{} & 0.0478** & 0.0831*** & 0.111** &       &       &  \\
(instrumented with ITT) & \multicolumn{1}{l}{} & \multicolumn{1}{l}{} & (0.0165) & (0.0280) & (0.0389) &       &       &  \\
Constant  & \multicolumn{1}{l}{0.0924***} & \multicolumn{1}{l}{0.0594***} & 0.0581*** & 0.0606*** & 0.0605*** & \multicolumn{1}{l}{0.0282} & \multicolumn{1}{l}{-0.0143} & \multicolumn{1}{l}{-0.00946} \\
      & \multicolumn{1}{l}{(0.0129)} & \multicolumn{1}{l}{(0.0145)} & (0.0149) & (0.0142) & (0.0144) & \multicolumn{1}{l}{(0.0298)} & \multicolumn{1}{l}{(0.0183)} & \multicolumn{1}{l}{(0.0189)} \\
      &       &       &       &       &       &       &       &  \\
\midrule
Observations & \multicolumn{1}{l}{3114} & \multicolumn{1}{l}{3114} & 3114  & 3114  & 3114  & \multicolumn{1}{l}{3114} & \multicolumn{1}{l}{3114} & \multicolumn{1}{l}{3114} \\
Dummy subcourts & \multicolumn{1}{l}{NO} & \multicolumn{1}{l}{YES} & YES   & YES   & YES   & \multicolumn{1}{l}{YES} & \multicolumn{1}{l}{YES} & \multicolumn{1}{l}{YES} \\
R-squared & \multicolumn{1}{l}{0.00281} & \multicolumn{1}{l}{0.0339} & 0.0349 & 0.0603 & 0.0584 & \multicolumn{1}{l}{0.384} & \multicolumn{1}{l}{0.156} & \multicolumn{1}{l}{0.107} \\
DepVarMean & \multicolumn{1}{l}{0.119} & \multicolumn{1}{l}{0.119} & 0.119 & 0.119 & 0.119 & \multicolumn{1}{l}{0.500} & \multicolumn{1}{l}{0.290} & \multicolumn{1}{l}{0.290} \\
\% Treated &       &       & 0.500 & 0.290 & 0.221 &       &       &  \\
\bottomrule
\bottomrule
\end{tabular}%
}
\end{center}

\end{landscape}

\subsection*{Expectation Stats}

Summary statistics for expectations.

\begin{center}
\scriptsize{
\begin{table}[!htbp] \centering 
  \caption{} 
  \label{} 
\begin{tabular}{@{\extracolsep{5pt}}lccccc} 
\\[-1.8ex]\hline 
\hline \\[-1.8ex] 
Statistic & \multicolumn{1}{c}{N} & \multicolumn{1}{c}{Mean} & \multicolumn{1}{c}{St. Dev.} & \multicolumn{1}{c}{Min} & \multicolumn{1}{c}{Max} \\ 
\hline \\[-1.8ex] 
Payment prob. (baseline) & 248 & 78.782 & 24.885 & 10 & 100 \\ 
Payment amount (baseline) & 210 & 69,297.190 & 104,987.700 & 0.000 & 850,000.000 \\ 
Payment prob. (exit) & 186 & 70.452 & 28.466 & 10 & 100 \\ 
Payment amount (exit) & 157 & 58,355.640 & 73,370.290 & 0.000 & 500,000.000 \\ 
\hline \\[-1.8ex] 
\end{tabular} 
\end{table} 
}
\end{center}

\pagebreak

\subsection*{Update in beliefs}

The update in beliefs is measured in relative terms, this is the percentage deviation of the exit survey with respect to the entrance survey
\[\frac{exit-initial}{initial}\]

We provide two measures: Update in beliefs in probability and payment.\\

\begin{center}
\scriptsize{
\begin{table}[htbp]\centering \caption{Update in beleifs \label{sumstat}}
\begin{tabular}{l c c c c c}\hline\hline
\multicolumn{1}{c}{\textbf{Variable}} & \textbf{Mean}
 & \textbf{Std. Dev.}& \textbf{Min.} &  \textbf{Max.} & \textbf{N}\\ \hline
update\_prob\_a & -0.134 & 0.25 & -0.9 & 0.125 & 122\\
update\_pago\_a & -0.318 & 0.443 & -1 & 0 & 89\\
update\_prob\_ra & -0.141 & 0.241 & -0.98 & 0 & 543\\
update\_pago\_ra & -0.345 & 0.437 & -1 & 0 & 494\\
update\_prob\_rd & -0.036 & 0.203 & -0.99 & 0.625 & 330\\
update\_pago\_rd & -0.303 & 0.425 & -1 & 0 & 252\\
\hline\end{tabular}
\end{table}
}
\end{center}

\begin{figure}[H]
\label{diff}
\begin{center}
\includegraphics[width=\textwidth]{./Figures/belief.pdf}
\end{center}
{\footnotesize \textit{The histograms are trimmed at the 75 percentile in the case for amount. Width of bins are \$30,000 pesos for the case of amount and 10\% for the case of probability.}}
\end{figure}


\begin{figure}[H]
\label{update}
\begin{center}
\includegraphics[width=\textwidth]{./Figures/update_belief.pdf}
\end{center}
{\footnotesize \textit{The histograms measures the update in relative terms. It displays the distribution from the percentile 5 to 95.}}
\end{figure}

\pagebreak


\begin{figure}[H]
\label{update}
\begin{center}
\includegraphics[width=\textwidth]{./Figures/updatebeleif_amount.pdf}
\end{center}
{\footnotesize \textit{Notes: Graphs on belief updating. They show the following statistic: $\theta=\left|\frac{P-exitsurvey}{P-initialsurvey}\right|$. Where $P$ is the prediction made by the calculator and $exit$ and $initial$ survey variables denote what each part answered on the survey for amount won. Note that when $\theta=1$ we have no update with respect to the prediction. And when $\theta<1$ they update in the direction of what calculator says. The extreme case $\theta=0$ implies perfect updating with respect to what the calculator says. }}
{\footnotesize \textit{Do file: } \texttt{Update\_beleifs.do}}
\end{figure}


\pagebreak

The following table displays a regression of conciliation against the statistics of updating
\[\theta=\left|\frac{P-exitsurvey}{P-initialsurvey}\right|\]

\begin{center}
\scriptsize{% Table generated by Excel2LaTeX from sheet 'con_vs_update'
\begin{tabular}{lccccccccc}
      & \multicolumn{9}{c}{Conciliation (Scale Up)} \\
\midrule
      & \multicolumn{3}{c}{Employee} & \multicolumn{3}{c}{Employee's Lawyer} & \multicolumn{3}{c}{Firm's Lawyer} \\
\midrule
\midrule
Update in beleifs & \multicolumn{1}{l}{0.341**} & \multicolumn{1}{l}{0.0389} & \multicolumn{1}{l}{3.307} & \multicolumn{1}{l}{-0.0499} & \multicolumn{1}{l}{-0.0108} & \multicolumn{1}{l}{0.106} & \multicolumn{1}{l}{0.00390} & \multicolumn{1}{l}{0.0158} & \multicolumn{1}{l}{0.00990} \\
      & \multicolumn{1}{l}{(0.119)} & \multicolumn{1}{l}{(0.0664)} & \multicolumn{1}{l}{(2.446)} & \multicolumn{1}{l}{(0.0689)} & \multicolumn{1}{l}{(0.0507)} & \multicolumn{1}{l}{(0.113)} & \multicolumn{1}{l}{(0.0895)} & \multicolumn{1}{l}{(0.0714)} & \multicolumn{1}{l}{(0.0897)} \\
Employee present (EP) &       & \multicolumn{1}{l}{-0.384**} &       &       & \multicolumn{1}{l}{0.395*} &       &       & \multicolumn{1}{l}{0.947***} &  \\
      &       & \multicolumn{1}{l}{(0.171)} &       &       & \multicolumn{1}{l}{(0.211)} &       &       & \multicolumn{1}{l}{(0.191)} &  \\
EP\#\#Update in beleifs &       & \multicolumn{1}{l}{0.673***} &       &       & \multicolumn{1}{l}{-0.255} &       &       & \multicolumn{1}{l}{-0.736***} &  \\
      &       & \multicolumn{1}{l}{(0.191)} &       &       & \multicolumn{1}{l}{(0.237)} &       &       & \multicolumn{1}{l}{(0.195)} &  \\
Treatment &       &       & \multicolumn{1}{l}{2.720} & \multicolumn{1}{l}{} & \multicolumn{1}{l}{} & \multicolumn{1}{l}{0.195**} & \multicolumn{1}{l}{} & \multicolumn{1}{l}{} & \multicolumn{1}{l}{0.204***} \\
      &       &       & \multicolumn{1}{l}{(2.074)} & \multicolumn{1}{l}{} & \multicolumn{1}{l}{} & \multicolumn{1}{l}{(0.0857)} & \multicolumn{1}{l}{} & \multicolumn{1}{l}{} & \multicolumn{1}{l}{(0.0257)} \\
Treatment\#\#Update in beleifs &       &       & \multicolumn{1}{l}{-2.982} &       &       & \multicolumn{1}{l}{-0.163} &       &       & \multicolumn{1}{l}{0} \\
      &       &       & \multicolumn{1}{l}{(2.434)} &       &       & \multicolumn{1}{l}{(0.178)} &       &       & \multicolumn{1}{l}{(.)} \\
Constant  & \multicolumn{1}{l}{-0.0997} & \multicolumn{1}{l}{0.0273} & \multicolumn{1}{l}{-2.807} & \multicolumn{1}{l}{0.179**} & \multicolumn{1}{l}{0.121**} & \multicolumn{1}{l}{-0.00811} & \multicolumn{1}{l}{0.196*} & \multicolumn{1}{l}{0.139} & \multicolumn{1}{l}{-0.00990} \\
      & \multicolumn{1}{l}{(0.0925)} & \multicolumn{1}{l}{(0.0607)} & \multicolumn{1}{l}{(2.076)} & \multicolumn{1}{l}{(0.0658)} & \multicolumn{1}{l}{(0.0526)} & \multicolumn{1}{l}{(0.0168)} & \multicolumn{1}{l}{(0.0974)} & \multicolumn{1}{l}{(0.0809)} & \multicolumn{1}{l}{(0.0897)} \\
      &       &       &       &       &       &       &       &       &  \\
\midrule
Observations & \multicolumn{1}{l}{120} & \multicolumn{1}{l}{120} & \multicolumn{1}{l}{120} & \multicolumn{1}{l}{478} & \multicolumn{1}{l}{478} & \multicolumn{1}{l}{478} & \multicolumn{1}{l}{301} & \multicolumn{1}{l}{301} & \multicolumn{1}{l}{301} \\
R-squared & \multicolumn{1}{l}{0.0151} & \multicolumn{1}{l}{0.120} & \multicolumn{1}{l}{0.0234} & \multicolumn{1}{l}{0.00135} & \multicolumn{1}{l}{0.0343} & \multicolumn{1}{l}{0.00255} & \multicolumn{1}{l}{0.00000552} & \multicolumn{1}{l}{0.0685} & \multicolumn{1}{l}{0.00510} \\
Update var mean & \multicolumn{3}{c}{0.954} & \multicolumn{3}{c}{0.895} & \multicolumn{3}{c}{0.914} \\
\bottomrule
\bottomrule
\end{tabular}%
}
\end{center}



Now we test difference in initial and exit survey in expectations. The following table shows the results.

\begin{center}
\scriptsize{% Table generated by Excel2LaTeX from sheet 'ttest'
\begin{tabular}{rrrr}
\toprule
      & \multicolumn{1}{c}{Initial} & \multicolumn{1}{c}{Exit} & \multicolumn{1}{c}{p-value} \\
\midrule
\multicolumn{4}{c}{Probability} \\
Employee  & \multicolumn{1}{l}{80.579} & \multicolumn{1}{l}{71.908} & \multicolumn{1}{l}{0} \\
      & \multicolumn{1}{l}{(23.937)} & \multicolumn{1}{l}{(28.544)} & \multicolumn{1}{l}{} \\
Employees Lawyer & \multicolumn{1}{l}{76.575} & \multicolumn{1}{l}{67.291} & \multicolumn{1}{l}{0} \\
      & \multicolumn{1}{l}{(20.368)} & \multicolumn{1}{l}{(24.549)} & \multicolumn{1}{l}{} \\
Firms Lawyer & \multicolumn{1}{l}{40.941} & \multicolumn{1}{l}{42.203} & \multicolumn{1}{l}{0.202} \\
      & \multicolumn{1}{l}{(23.17)} & \multicolumn{1}{l}{(23.148)} & \multicolumn{1}{l}{} \\
\multicolumn{4}{c}{Payment levels} \\
Employee  & \multicolumn{1}{l}{66988.388} & \multicolumn{1}{l}{61143.114} & \multicolumn{1}{l}{0.409} \\
      & \multicolumn{1}{l}{(108493.426)} & \multicolumn{1}{l}{(78846.58)} & \multicolumn{1}{l}{} \\
Employees Lawyer & \multicolumn{1}{l}{104848.984} & \multicolumn{1}{l}{94671.345} & \multicolumn{1}{l}{0} \\
      & \multicolumn{1}{l}{(126404.274)} & \multicolumn{1}{l}{(117284.239)} & \multicolumn{1}{l}{} \\
Firms Lawyer & \multicolumn{1}{l}{64471.81} & \multicolumn{1}{l}{60609.318} & \multicolumn{1}{l}{0.312} \\
      & \multicolumn{1}{l}{(99824.872)} & \multicolumn{1}{l}{(88106.17)} & \multicolumn{1}{l}{} \\
\multicolumn{4}{c}{Payment logs} \\
Employee  & \multicolumn{1}{l}{10.361} & \multicolumn{1}{l}{10.403} & \multicolumn{1}{l}{0.58} \\
      & \multicolumn{1}{l}{(1.547)} & \multicolumn{1}{l}{(1.4)} & \multicolumn{1}{l}{} \\
Employees Lawyer & \multicolumn{1}{l}{10.88} & \multicolumn{1}{l}{10.791} & \multicolumn{1}{l}{0.002} \\
      & \multicolumn{1}{l}{(1.618)} & \multicolumn{1}{l}{(1.572)} & \multicolumn{1}{l}{} \\
Firms Lawyer & \multicolumn{1}{l}{9.577} & \multicolumn{1}{l}{9.6} & \multicolumn{1}{l}{0.831} \\
      & \multicolumn{1}{l}{(3.188)} & \multicolumn{1}{l}{(3.024)} & \multicolumn{1}{l}{} \\
\bottomrule
\end{tabular}%
}
\end{center}

\pagebreak

\section{ITT and ATT regressions}

\begin{landscape}
\subsection*{ITT and Second Stage ATT}

\begin{center}
\scriptsize{% Table generated by Excel2LaTeX from sheet 'ITT_ATT'
\begin{tabular}{lrrlrlrlrrr}
      & \multicolumn{7}{c}{Conciliation}                      & \multicolumn{1}{c}{Calculator Plaintiff} & \multicolumn{1}{c}{Calculator Defendant} & \multicolumn{1}{c}{Calculator Both} \\
\cmidrule{2-11}      & \multicolumn{1}{c}{ITT} & \multicolumn{1}{c}{ITT} & \multicolumn{2}{c}{ATT (Plaintiff)} & \multicolumn{2}{c}{ATT  (Defendant)} & \multicolumn{1}{c}{ATT (Both)} & \multicolumn{3}{c}{First Stage} \\
\cmidrule{2-11}Treatment  & \multicolumn{1}{l}{0.0540*} & \multicolumn{1}{l}{0.0486**} &       &       &       &       &       & \multicolumn{1}{l}{0.234***} & \multicolumn{1}{l}{0.146***} & \multicolumn{1}{l}{0.489***} \\
      & \multicolumn{1}{l}{(0.0264)} & \multicolumn{1}{l}{(0.0202)} &       &       &       &       &       & \multicolumn{1}{l}{(0.0182)} & \multicolumn{1}{l}{(0.0117)} & \multicolumn{1}{l}{(0.0263)} \\
Calculator  & \multicolumn{1}{l}{} & \multicolumn{1}{l}{} & 0.208** &       & 0.333** &       & 0.0995** &       &       &  \\
(instrumented with ITT) & \multicolumn{1}{l}{} & \multicolumn{1}{l}{} & (0.0891) &       & (0.136) &       & (0.0417) &       &       &  \\
Calculator  &       &       &       & \multicolumn{1}{l}{-0.0364} &       & \multicolumn{1}{l}{-0.0388} &       &       &       &  \\
(instrumented with ITT \& ITT*EP) &       &       &       & \multicolumn{1}{l}{(0.0688)} &       & \multicolumn{1}{l}{(0.108)} &       &       &       &  \\
Constant  & \multicolumn{1}{l}{0.159***} & \multicolumn{1}{l}{0.124***} & 0.119*** & \multicolumn{1}{l}{0.127***} & 0.126*** & \multicolumn{1}{l}{0.0992***} & 0.125*** & \multicolumn{1}{l}{0.0234} & \multicolumn{1}{l}{-0.00657} & \multicolumn{1}{l}{-0.00841} \\
      & \multicolumn{1}{l}{(0.0184)} & \multicolumn{1}{l}{(0.0286)} & (0.0323) & \multicolumn{1}{l}{(0.0284)} & (0.0261) & \multicolumn{1}{l}{(0.0335)} & (0.0280) & \multicolumn{1}{l}{(0.0247)} & \multicolumn{1}{l}{(0.0217)} & \multicolumn{1}{l}{(0.0287)} \\
      &       &       &       &       &       &       &       &       &       &  \\
\midrule
Observations & \multicolumn{1}{l}{1285} & \multicolumn{1}{l}{1285} & 1285  & \multicolumn{1}{l}{1285} & 1285  & \multicolumn{1}{l}{1285} & 1285  & \multicolumn{1}{l}{1285} & \multicolumn{1}{l}{1285} & \multicolumn{1}{l}{1285} \\
Dummy subcourts & \multicolumn{1}{l}{NO} & \multicolumn{1}{l}{YES} & YES   & \multicolumn{1}{l}{YES} & YES   & \multicolumn{1}{l}{YES} & YES   & \multicolumn{1}{l}{YES} & \multicolumn{1}{l}{YES} & \multicolumn{1}{l}{YES} \\
Conditioning on notified & \multicolumn{1}{l}{YES} & \multicolumn{1}{l}{YES} & YES   & \multicolumn{1}{l}{YES} & YES   & \multicolumn{1}{l}{YES} & YES   & \multicolumn{1}{l}{YES} & \multicolumn{1}{l}{YES} & \multicolumn{1}{l}{YES} \\
R-squared & \multicolumn{1}{l}{0.00392} & \multicolumn{1}{l}{0.0449} & .     & \multicolumn{1}{l}{.} & .     & \multicolumn{1}{l}{.} & 0.0494 & \multicolumn{1}{l}{0.0900} & \multicolumn{1}{l}{0.0507} & \multicolumn{1}{l}{0.222} \\
DepVarMean & \multicolumn{1}{l}{0.197} & \multicolumn{1}{l}{0.197} & 0.197 & \multicolumn{1}{l}{0.197} & 0.197 & \multicolumn{1}{l}{0.197} & 0.197 & \multicolumn{1}{l}{0.519} & \multicolumn{1}{l}{0.453} & \multicolumn{1}{l}{0.453} \\
\% Treated &       &       & 0.519 &       & 0.453 &       & 0.355 &       &       &  \\
\bottomrule
\bottomrule
\end{tabular}%
}
\end{center}

\end{landscape}

\pagebreak


\subsection*{First Stage for ATT regressions}

\begin{center}
\scriptsize{% Table generated by Excel2LaTeX from sheet 'FS_ATT'
\begin{tabular}{lllllllll}
\cmidrule{2-9}\multicolumn{1}{r|}{} & \multicolumn{4}{c|}{Plaintiff} & \multicolumn{4}{c|}{Defendant} \\
\cmidrule{2-9}      & \multicolumn{1}{c}{Calculator} & \multicolumn{1}{c}{Calculator} & \multicolumn{1}{c}{Calculator} & \multicolumn{1}{c}{Calculator*EP} & \multicolumn{1}{c}{Calculator} & \multicolumn{1}{c}{Calculator} & \multicolumn{1}{c}{Calculator} & \multicolumn{1}{c}{Calculator*EP} \\
\cmidrule{2-9}ITT   & 0.679*** & 0.722*** & 0.648*** & -0.000286 & 0.395*** & 0.635*** & 0.381*** & 0.00219 \\
      & (0.0161) & (0.0215) & (0.0193) & (0.00171) & (0.0170) & (0.0220) & (0.0185) & (0.00315) \\
ITT*EP &       &       & 0.192*** & 0.861*** &       &       & 0.0837** & 0.488*** \\
      &       &       & (0.0250) & (0.0174) &       &       & (0.0296) & (0.0341) \\
Court 7 & 0.00758 & 0.0223 & 0.00426 & 0.00101 & 0.0201 & 0.0306 & 0.0186 & 0.00304 \\
      & (0.0364) & (0.0402) & (0.0364) & (0.00641) & (0.0291) & (0.0518) & (0.0289) & (0.00938) \\
Court 9 & -0.0317 & -0.00906 & -0.0331 & -0.0106 & 0.00543 & -0.0132 & 0.00482 & -0.00495 \\
      & (0.0403) & (0.0452) & (0.0408) & (0.00938) & (0.0270) & (0.0530) & (0.0272) & (0.00953) \\
Court 11 & -0.00328 & 0.0138 & 0.000422 & -0.0103 & 0.0660* & 0.0701 & 0.0676* & 0.00966 \\
      & (0.0457) & (0.0535) & (0.0458) & (0.00597) & (0.0333) & (0.0411) & (0.0334) & (0.00675) \\
Court 16 & -0.0261 & -0.0106 & -0.0289 & 0.00403 & 0.00389 & 0.0410 & 0.00266 & 0.00973 \\
      & (0.0413) & (0.0427) & (0.0405) & (0.00602) & (0.0232) & (0.0436) & (0.0233) & (0.00991) \\
Notified & 0.0512*** &       & 0.0436*** & 0.0104** & 0.286*** &       & 0.282*** & 0.0624*** \\
      & (0.0105) &       & (0.0107) & (0.00458) & (0.0182) &       & (0.0182) & (0.00622) \\
Constant  & 0.00775 & 0.0150 & 0.0120 & 0.000434 & -0.128*** & -0.0150 & -0.126*** & -0.0293*** \\
      & (0.0296) & (0.0319) & (0.0294) & (0.00494) & (0.0220) & (0.0289) & (0.0221) & (0.00835) \\
      &       &       &       &       &       &       &       &  \\
\midrule
Observations & 3114  & 1285  & 3114  & 3114  & 3114  & 1285  & 3114  & 3114 \\
Conditioning on notified & NO    & YES   & NO    & NO    & NO    & YES   & NO    & NO \\
R-squared & 0.387 & 0.448 & 0.401 & 0.842 & 0.252 & 0.343 & 0.255 & 0.476 \\
\bottomrule
\bottomrule
\end{tabular}%
}
\end{center}


\section{Take up regression (calculator and survey)}


\begin{center}
\scriptsize{% Table generated by Excel2LaTeX from sheet 'Take_up'
\begin{tabular}{rrrr}
\toprule
      & \multicolumn{1}{c}{Take up plaintiff} & \multicolumn{1}{c}{Take up defendant} & \multicolumn{1}{c}{Take up at least one} \\
\midrule
      &       &       &  \\
Female & \multicolumn{1}{l}{-0.00206} & \multicolumn{1}{l}{0.0262} & \multicolumn{1}{l}{0.0237} \\
      & \multicolumn{1}{l}{(0.0218)} & \multicolumn{1}{l}{(0.0234)} & \multicolumn{1}{l}{(0.0191)} \\
c\_antiguedad & \multicolumn{1}{l}{-0.00285} & \multicolumn{1}{l}{0.00339} & \multicolumn{1}{l}{-0.000830} \\
      & \multicolumn{1}{l}{(0.00204)} & \multicolumn{1}{l}{(0.00212)} & \multicolumn{1}{l}{(0.00170)} \\
c\_indem & \multicolumn{1}{l}{-0.000000314**} & \multicolumn{1}{l}{-4.04e-08} & \multicolumn{1}{l}{-0.000000274**} \\
      & \multicolumn{1}{l}{(0.000000143)} & \multicolumn{1}{l}{(0.000000144)} & \multicolumn{1}{l}{(0.000000128)} \\
dummy\_reinst & \multicolumn{1}{l}{0.0527**} & \multicolumn{1}{l}{-0.00210} & \multicolumn{1}{l}{0.0388*} \\
      & \multicolumn{1}{l}{(0.0230)} & \multicolumn{1}{l}{(0.0244)} & \multicolumn{1}{l}{(0.0201)} \\
Public Lawyer & \multicolumn{1}{l}{0.0456**} & \multicolumn{1}{l}{-0.0412*} & \multicolumn{1}{l}{0.0102} \\
      & \multicolumn{1}{l}{(0.0206)} & \multicolumn{1}{l}{(0.0228)} & \multicolumn{1}{l}{(0.0191)} \\
Constant  & \multicolumn{1}{l}{0.647***} & \multicolumn{1}{l}{0.435***} & \multicolumn{1}{l}{0.770***} \\
      & \multicolumn{1}{l}{(0.0348)} & \multicolumn{1}{l}{(0.0371)} & \multicolumn{1}{l}{(0.0313)} \\
      & \multicolumn{1}{l}{} & \multicolumn{1}{l}{} & \multicolumn{1}{l}{} \\
Observations & \multicolumn{1}{l}{1826} & \multicolumn{1}{l}{1816} & \multicolumn{1}{l}{1838} \\
R-squared & \multicolumn{1}{l}{0.00741} & \multicolumn{1}{l}{0.00407} & \multicolumn{1}{l}{0.00557} \\
Dep Var Mean & \multicolumn{1}{l}{0.695} & \multicolumn{1}{l}{0.410} & \multicolumn{1}{l}{0.793} \\
\bottomrule
\end{tabular}%
}
\end{center}


%%%%%%%%%%%%%%%%%%%%%%%%%%%%%%%%%%%%%%%%%%%%%%%%%%%%%%%%%%%%%%%%%%%%%%%%%%%%%%%%
\end{document}
