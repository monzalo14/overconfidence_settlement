\documentclass[12pt]{article}
\usepackage{graphicx}
\usepackage{float}
\usepackage{amsmath}
\usepackage{amscd}
\usepackage{hyperref}
\usepackage{enumerate}
\usepackage{amsfonts}
\usepackage{amssymb}
\usepackage[utf8]{inputenc}
\usepackage{amsthm}
\usepackage{subcaption}
\usepackage{listings}
\usepackage{tikz}
\usepackage{color} %red, green, blue, yellow, cyan, magenta, black, white
\usepackage{fullpage}
\usepackage{mathtools}
\usepackage{booktabs}
\usepackage{longtable}

\definecolor{mygreen}{RGB}{28,172,0} % color values Red, Green, Blue
\definecolor{mylilas}{RGB}{170,55,241}

\newtheorem{theorem}{Teorema}
\newtheorem{lem}[theorem]{Lemma}
\newtheorem{dfn}{Definición}
\newtheorem{cor}[theorem]{Corolario}
\newtheorem{obs}{Obs}
\newtheorem{rem}{Remark}
\newtheorem{prob}{Problema}

\newcommand*\circled[1]{\tikz[baseline=(char.base)]{
            \node[shape=circle,draw,inner sep=.05pt] (char) {#1};}}
            


\newtheoremstyle{named}{}{}{\itshape}{}{\bfseries}{.}{.5em} {\thmnote{#3 }#1}
\theoremstyle{named}
\newtheorem*{namedtheorem}{}



\newcounter{exercisecounter}
\newenvironment{ex}{\begin{quote}%
    \refstepcounter{exercisecounter}%
  \textbf{Ejercicio \arabic{exercisecounter}}%
  \quad
}{%
\end{quote}%
}
\newcounter{ejemplocounter}
\newenvironment{ej}{\begin{quote}%
    \refstepcounter{ejemplocounter}%
  \textbf{Ejemplo \arabic{ejemplocounter}}%
  \quad
}{%
\end{quote}%
}

\renewcommand{\d}[1]{\ensuremath{\operatorname{d}\!{#1}}}

\DeclarePairedDelimiter{\ceil}{\lceil}{\rceil}


\newcommand{\folder}{./Effect}

\begin{document}


\title{Treatment effects}

\author{Instituto Tecnológico Autónomo de México}
\date{\today}
\maketitle


\hrulefill


\section{\Huge{Some Summary Statistics}}

\vspace{7mm}

\subsection*{Presence of actors}


\begin{center}
\scriptsize{
\begin{table}[htbp]\centering \caption{Presence of actors \label{sumstat}}
\begin{tabular}{l c c c c c}\hline\hline
\multicolumn{1}{c}{\textbf{Variable}} & \textbf{Mean}
 & \textbf{Std. Dev.}& \textbf{Min.} &  \textbf{Max.} & \textbf{N}\\ \hline
p\_actor & 0.151 & 0.358 & 0 & 1 & 2183\\
p\_ractor & 0.827 & 0.378 & 0 & 1 & 2183\\
p\_dem & 0.011 & 0.102 & 0 & 1 & 2183\\
p\_rdem & 0.44 & 0.496 & 0 & 1 & 2183\\
\hline\end{tabular}
\end{table}
}
\end{center}

We decompose who showed up in all the possible combinations; we use 1,2,3,4 subscripts to denote the different parties 1: Employee; 2: Employee’s Lawyer;  3: Firm; 4: Firm’s Lawyer. 


\begin{center}
\scriptsize{
\begin{table}[htbp]\centering \caption{Presence of actors \label{sumstat}}
\begin{tabular}{l c c c c c}\hline\hline
\multicolumn{1}{c}{\textbf{Variable}} & \textbf{Mean}
 & \textbf{Std. Dev.}& \textbf{Min.} &  \textbf{Max.} & \textbf{N}\\ \hline
v0 & 0.131 & 0.337 & 0 & 1 & 2183\\
v1 & 0.007 & 0.083 & 0 & 1 & 2183\\
v2 & 0.368 & 0.482 & 0 & 1 & 2183\\
v3 & 0.001 & 0.037 & 0 & 1 & 2183\\
v4 & 0.029 & 0.167 & 0 & 1 & 2183\\
v12 & 0.052 & 0.223 & 0 & 1 & 2183\\
v13 & 0 & 0.021 & 0 & 1 & 2183\\
v14 & 0.004 & 0.064 & 0 & 1 & 2183\\
v23 & 0 & 0.021 & 0 & 1 & 2183\\
v24 & 0.316 & 0.465 & 0 & 1 & 2183\\
v34 & 0 & 0 & 0 & 0 & 2183\\
v123 & 0 & 0 & 0 & 0 & 2183\\
v124 & 0.083 & 0.276 & 0 & 1 & 2183\\
v134 & 0 & 0.021 & 0 & 1 & 2183\\
v234 & 0.004 & 0.064 & 0 & 1 & 2183\\
v1234 & 0.004 & 0.06 & 0 & 1 & 2183\\
\hline\end{tabular}
\end{table}
}
\end{center}

\pagebreak

\subsection*{Summary statistics for survey variables}

\begin{center}
\scriptsize{
\begin{table}[htbp]\centering \caption{SS survey variables \label{sumstat}}
\begin{tabular}{l c c c c c}\hline\hline
\multicolumn{1}{c}{\textbf{Variable}} & \textbf{Mean}
 & \textbf{Std. Dev.}& \textbf{Min.} &  \textbf{Max.} & \textbf{N}\\ \hline
Initial prob employee & 79.812 & 24.906 & 10 & 100 & 223\\
Initial amount employee & 62736.224 & 87655.907 & 0 & 600000 & 188\\
Exit prob employee & 71.076 & 28.445 & 10 & 100 & 171\\
Exit amount employee & 54801.663 & 66298.621 & 0 & 350000 & 142\\
Initial prob employee's lawyer & 77.762 & 20.026 & 2 & 100 & 965\\
Initial amount employee's lawyer & 97572.459 & 112949.152 & 0 & 700000 & 768\\
Exit prob employee's lawyer & 67.505 & 24.514 & 1 & 100 & 733\\
Exit amount employee's lawyer & 87234.616 & 96300.243 & 0 & 600000 & 568\\
Initial prob firm's lawyer & 40.363 & 22.95 & 1 & 100 & 584\\
Initial amount firm's lawyer & 59030.199 & 82031.346 & 0 & 662451 & 462\\
Exit prob firm's lawyer & 42.416 & 23.352 & 1 & 100 & 440\\
Exit amount firm's lawyer & 56965.815 & 70131.048 & 0 & 480001 & 363\\
Buys goods & 0.074 & 0.263 & 0 & 1 & 229\\
Works & 0.426 & 0.496 & 0 & 1 & 230\\
Looking for a job & 0.551 & 0.498 & 0 & 1 & 225\\
\hline\end{tabular}
\end{table}
}
\end{center}


\begin{center}
\scriptsize{% Table generated by Excel2LaTeX from sheet 'TAB_survey'
\begin{tabular}{rr}
\toprule
\multicolumn{2}{c}{Anger employee} \\
\midrule
\midrule
\multicolumn{1}{l}{A lot} & 143 \\
\multicolumn{1}{l}{Fairly} & 41 \\
\multicolumn{1}{l}{Little } & 18 \\
\multicolumn{1}{l}{Nothing} & 37 \\
\midrule
\multicolumn{2}{c}{Education employee} \\
\midrule
\midrule
\multicolumn{1}{l}{Elementary} & 16 \\
\multicolumn{1}{l}{Secondary} & 65 \\
\multicolumn{1}{l}{High-School} & 68 \\
\multicolumn{1}{l}{+ High School} & 90 \\
\midrule
\multicolumn{2}{c}{\#Cases employee's lawyer} \\
\midrule
\midrule
\multicolumn{1}{l}{1-10} & 106 \\
\multicolumn{1}{l}{11-30} & 126 \\
\multicolumn{1}{l}{31-100} & 256 \\
\multicolumn{1}{l}{+ 100} & 490 \\
\midrule
\multicolumn{2}{c}{\#Cases firm's lawyer} \\
\midrule
\midrule
\multicolumn{1}{l}{1-10} & 50 \\
\multicolumn{1}{l}{11-30} & 54 \\
\multicolumn{1}{l}{31-100} & 124 \\
\multicolumn{1}{l}{+ 100} & 368 \\
\bottomrule
\end{tabular}%
}
\end{center}


\begin{center}
\scriptsize{% Table generated by Excel2LaTeX from sheet 'NA'
\begin{tabular}{rr}
\toprule
Variable & Missing values \\
\midrule
\multicolumn{2}{c}{Employee} \\
\multicolumn{1}{l}{Probability (baseline)} & 16 \\
\multicolumn{1}{l}{Amount (baseline)} & 51 \\
\multicolumn{1}{l}{Probability (exit)} & 68 \\
\multicolumn{1}{l}{Amount (exit)} & 97 \\
\multicolumn{1}{l}{Buys durable goods} & 10 \\
\multicolumn{1}{l}{Working at the time} & 9 \\
\multicolumn{1}{l}{Looking for a job} & 14 \\
\multicolumn{1}{l}{Anger employee} & 0 \\
\multicolumn{1}{l}{Education employee} & 0 \\
\multicolumn{2}{c}{Employee's Lawyer} \\
\multicolumn{1}{l}{Probability (baseline)} & 13 \\
\multicolumn{1}{l}{Amount (baseline)} & 210 \\
\multicolumn{1}{l}{Probability (exit)} & 245 \\
\multicolumn{1}{l}{Amount (exit)} & 410 \\
\multicolumn{1}{l}{Cases employee's lawyer} & 0 \\
\multicolumn{2}{c}{Firm's Lawyer} \\
\multicolumn{1}{l}{Probability (baseline)} & 12 \\
\multicolumn{1}{l}{Amount (baseline)} & 134 \\
\multicolumn{1}{l}{Probability (exit)} & 156 \\
\multicolumn{1}{l}{Amount (exit)} & 233 \\
\multicolumn{1}{l}{Cases firm's lawyer} & 0 \\
\bottomrule
\end{tabular}%
}
\end{center}


\pagebreak

\subsection*{Update in beliefs}

The update in beliefs is measured in relative terms, this is the percentage deviation of the exit survey with respect to the entrance survey
\[\frac{exit-initial}{initial}\]

 We provide two measures: Update in beliefs in probability and payment.\\

\begin{center}
\scriptsize{
\begin{table}[htbp]\centering \caption{Update in beleifs \label{sumstat}}
\begin{tabular}{l c c c c c}\hline\hline
\multicolumn{1}{c}{\textbf{Variable}} & \textbf{Mean}
 & \textbf{Std. Dev.}& \textbf{Min.} &  \textbf{Max.} & \textbf{N}\\ \hline
update\_prob\_a & -0.134 & 0.25 & -0.9 & 0.125 & 122\\
update\_pago\_a & -0.318 & 0.443 & -1 & 0 & 89\\
update\_prob\_ra & -0.141 & 0.241 & -0.98 & 0 & 543\\
update\_pago\_ra & -0.345 & 0.437 & -1 & 0 & 494\\
update\_prob\_rd & -0.036 & 0.203 & -0.99 & 0.625 & 330\\
update\_pago\_rd & -0.303 & 0.425 & -1 & 0 & 252\\
\hline\end{tabular}
\end{table}
}
\end{center}



\begin{figure}[H]
\label{diff}
\begin{center}
\includegraphics[width=\textwidth]{./Figures/belief.pdf}
\end{center}
{\footnotesize \textit{The histograms are trimmed at the 90 percentile in the case for amount. Width of bins are \$50,000 pesos for the case of amount and 10\% for the case of probability.}}
\end{figure}


\begin{figure}[H]
\label{update}
\begin{center}
\includegraphics[width=\textwidth]{./Figures/update_belief.pdf}
\end{center}
{\footnotesize \textit{The histograms measures the update in relative terms. Width of bins is 0.25 }}
\end{figure}

\pagebreak

Now we test difference in initial and exit survey in expectations. The following table shows the results.

\begin{center}
\scriptsize{% Table generated by Excel2LaTeX from sheet 'ttest'
\begin{tabular}{rrrr}
\toprule
      & \multicolumn{1}{c}{Initial} & \multicolumn{1}{c}{Exit} & \multicolumn{1}{c}{p-value} \\
\midrule
\multicolumn{4}{c}{Probability} \\
Employee  & \multicolumn{1}{l}{80.579} & \multicolumn{1}{l}{71.908} & \multicolumn{1}{l}{0} \\
      & \multicolumn{1}{l}{(23.937)} & \multicolumn{1}{l}{(28.544)} & \multicolumn{1}{l}{} \\
Employees Lawyer & \multicolumn{1}{l}{76.575} & \multicolumn{1}{l}{67.291} & \multicolumn{1}{l}{0} \\
      & \multicolumn{1}{l}{(20.368)} & \multicolumn{1}{l}{(24.549)} & \multicolumn{1}{l}{} \\
Firms Lawyer & \multicolumn{1}{l}{40.941} & \multicolumn{1}{l}{42.203} & \multicolumn{1}{l}{0.202} \\
      & \multicolumn{1}{l}{(23.17)} & \multicolumn{1}{l}{(23.148)} & \multicolumn{1}{l}{} \\
\multicolumn{4}{c}{Payment levels} \\
Employee  & \multicolumn{1}{l}{66988.388} & \multicolumn{1}{l}{61143.114} & \multicolumn{1}{l}{0.409} \\
      & \multicolumn{1}{l}{(108493.426)} & \multicolumn{1}{l}{(78846.58)} & \multicolumn{1}{l}{} \\
Employees Lawyer & \multicolumn{1}{l}{104848.984} & \multicolumn{1}{l}{94671.345} & \multicolumn{1}{l}{0} \\
      & \multicolumn{1}{l}{(126404.274)} & \multicolumn{1}{l}{(117284.239)} & \multicolumn{1}{l}{} \\
Firms Lawyer & \multicolumn{1}{l}{64471.81} & \multicolumn{1}{l}{60609.318} & \multicolumn{1}{l}{0.312} \\
      & \multicolumn{1}{l}{(99824.872)} & \multicolumn{1}{l}{(88106.17)} & \multicolumn{1}{l}{} \\
\multicolumn{4}{c}{Payment logs} \\
Employee  & \multicolumn{1}{l}{10.361} & \multicolumn{1}{l}{10.403} & \multicolumn{1}{l}{0.58} \\
      & \multicolumn{1}{l}{(1.547)} & \multicolumn{1}{l}{(1.4)} & \multicolumn{1}{l}{} \\
Employees Lawyer & \multicolumn{1}{l}{10.88} & \multicolumn{1}{l}{10.791} & \multicolumn{1}{l}{0.002} \\
      & \multicolumn{1}{l}{(1.618)} & \multicolumn{1}{l}{(1.572)} & \multicolumn{1}{l}{} \\
Firms Lawyer & \multicolumn{1}{l}{9.577} & \multicolumn{1}{l}{9.6} & \multicolumn{1}{l}{0.831} \\
      & \multicolumn{1}{l}{(3.188)} & \multicolumn{1}{l}{(3.024)} & \multicolumn{1}{l}{} \\
\bottomrule
\end{tabular}%
}
\end{center}


\pagebreak

\subsection*{Treatments: Calculadora}

The following variables are dummies whether plaintiff and defendant (resp.) took part of the \emph{calculadora}.\\

\begin{center}
\scriptsize{
\begin{table}[htbp]\centering
\caption{\label{calcu_p_actora_by_calcu_p_dem} 
\textbf{Calculadora Actora by Calculadora Demandado}}
\begin{tabular} {@{} l r  r r @{}} \\ \hline
& \multicolumn{3}{@{} c @{}}{\textbf{Calculadora Demandado}} \\
\textbf{Calculadora Actora} & 
No &Yes &Total \\  \hline
No&      982&      163&    1,145\\
Yes&      589&      449&    1,038\\
Total&    1,571&      612&    2,183\\\hline 
\multicolumn{4}{@{}l}{\footnotesize{\emph{Source:} C:\Users\chasi_000\Dropbox\Statistics\P10\DB\DB2\Seguimiento_Juntas.dta}}
\end{tabular}
\end{table}



}
\end{center}

\pagebreak

Take up treatment regression:

\begin{center}
\scriptsize{% Table generated by Excel2LaTeX from sheet 'Take_up'
\begin{tabular}{rrrr}
\toprule
      & \multicolumn{1}{c}{Take up plaintiff} & \multicolumn{1}{c}{Take up defendant} & \multicolumn{1}{c}{Take up at least one} \\
\midrule
      &       &       &  \\
Female & \multicolumn{1}{l}{-0.00206} & \multicolumn{1}{l}{0.0262} & \multicolumn{1}{l}{0.0237} \\
      & \multicolumn{1}{l}{(0.0218)} & \multicolumn{1}{l}{(0.0234)} & \multicolumn{1}{l}{(0.0191)} \\
c\_antiguedad & \multicolumn{1}{l}{-0.00285} & \multicolumn{1}{l}{0.00339} & \multicolumn{1}{l}{-0.000830} \\
      & \multicolumn{1}{l}{(0.00204)} & \multicolumn{1}{l}{(0.00212)} & \multicolumn{1}{l}{(0.00170)} \\
c\_indem & \multicolumn{1}{l}{-0.000000314**} & \multicolumn{1}{l}{-4.04e-08} & \multicolumn{1}{l}{-0.000000274**} \\
      & \multicolumn{1}{l}{(0.000000143)} & \multicolumn{1}{l}{(0.000000144)} & \multicolumn{1}{l}{(0.000000128)} \\
dummy\_reinst & \multicolumn{1}{l}{0.0527**} & \multicolumn{1}{l}{-0.00210} & \multicolumn{1}{l}{0.0388*} \\
      & \multicolumn{1}{l}{(0.0230)} & \multicolumn{1}{l}{(0.0244)} & \multicolumn{1}{l}{(0.0201)} \\
Public Lawyer & \multicolumn{1}{l}{0.0456**} & \multicolumn{1}{l}{-0.0412*} & \multicolumn{1}{l}{0.0102} \\
      & \multicolumn{1}{l}{(0.0206)} & \multicolumn{1}{l}{(0.0228)} & \multicolumn{1}{l}{(0.0191)} \\
Constant  & \multicolumn{1}{l}{0.647***} & \multicolumn{1}{l}{0.435***} & \multicolumn{1}{l}{0.770***} \\
      & \multicolumn{1}{l}{(0.0348)} & \multicolumn{1}{l}{(0.0371)} & \multicolumn{1}{l}{(0.0313)} \\
      & \multicolumn{1}{l}{} & \multicolumn{1}{l}{} & \multicolumn{1}{l}{} \\
Observations & \multicolumn{1}{l}{1826} & \multicolumn{1}{l}{1816} & \multicolumn{1}{l}{1838} \\
R-squared & \multicolumn{1}{l}{0.00741} & \multicolumn{1}{l}{0.00407} & \multicolumn{1}{l}{0.00557} \\
Dep Var Mean & \multicolumn{1}{l}{0.695} & \multicolumn{1}{l}{0.410} & \multicolumn{1}{l}{0.793} \\
\bottomrule
\end{tabular}%
}
\end{center}

\pagebreak



\subsection*{Conciliation (Convenio) in the day of treatment}

The main variable is convenio, a dummy variable indicating whether the parts conciled or not.\\

\begin{center}
\scriptsize{
\begin{table}[htbp]\centering
\caption{\label{convenio_by_calcu_p_actora} 
\textbf{Convenio by Calculadora Actora}}
\begin{tabular} {@{} l r  r r @{}} \\ \hline
& \multicolumn{3}{@{} c @{}}{\textbf{Calculadora Actora}} \\
\textbf{Convenio} & 
No &Yes &Total \\  \hline
No concilio&    1,028&      893&    1,921\\
Concilio&      117&      145&      262\\
Total&    1,145&    1,038&    2,183\\\hline 
\multicolumn{4}{@{}l}{\footnotesize{}}
\end{tabular}
\end{table}



}
\end{center}

\begin{center}
\scriptsize{
\begin{table}[htbp]\centering
\caption{\label{convenio_by_calcu_p_dem} 
\textbf{Convenio by Calculadora Demandado}}
\begin{tabular} {@{} l r  r r @{}} \\ \hline
& \multicolumn{3}{@{} c @{}}{\textbf{Calculadora Demandado}} \\
\textbf{Convenio} & 
No &Yes &Total \\  \hline
No concilio&    1,441&      480&    1,921\\
Concilio&      130&      132&      262\\
Total&    1,571&      612&    2,183\\\hline 
\multicolumn{4}{@{}l}{\footnotesize{}}
\end{tabular}
\end{table}



}
\end{center}


\vspace{5mm}


\pagebreak


\vspace{5mm}

\section{\Huge{Regression: Treatment effects}}



The following regressions has as dependent variable \emph{\textbf{convenio}}.\\



\subsection*{Simple regression (Correlation)}

\begin{table}[H]\centering \caption{Regression with \emph{calculadora} and controlling by junta.}
\begin{center}
\scriptsize{% Table generated by Excel2LaTeX from sheet 'Simple_regression'
\begin{tabular}{rrrrrrr}
\toprule
      & \multicolumn{1}{c}{(1)} & \multicolumn{1}{c}{(2)} & \multicolumn{1}{c}{(3)} & \multicolumn{1}{c}{(4)} & \multicolumn{1}{c}{(5)} & \multicolumn{1}{c}{(6)} \\
\midrule
      &       &       &       &       &       &  \\
Calculadora\_actor & 0.0388*** & 0.0394*** &       &       & -0.00475 & -0.00177 \\
      & (0.0128) & (0.0129) &       &       & (0.0127) & (0.0128) \\
      &       &       &       &       &       &  \\
junta\_2 &       & -0.00148 &       & -0.00608 &       & -0.00451 \\
      &       & (0.0177) &       & (0.0167) &       & (0.0172) \\
      &       &       &       &       &       &  \\
junta\_7 &       & 0.0391** &       & 0.0343* &       & 0.0344* \\
      &       & (0.0187) &       & (0.0181) &       & (0.0183) \\
      &       &       &       &       &       &  \\
junta\_9 &       & 0.0178 &       & 0.0160 &       & 0.0151 \\
      &       & (0.0174) &       & (0.0170) &       & (0.0172) \\
      &       &       &       &       &       &  \\
junta\_11 &       & 0.163*** &       & 0.152*** &       & 0.152*** \\
      &       & (0.0217) &       & (0.0214) &       & (0.0215) \\
      &       &       &       &       &       &  \\
Calculadora\_dem &       &       & 0.141*** & 0.135*** & 0.145*** & 0.137*** \\
      &       &       & (0.0163) & (0.0163) & (0.0168) & (0.0167) \\
      &       &       &       &       &       &  \\
Constant & 0.105*** & 0.0573*** & 0.0832*** & 0.0427*** & 0.0847*** & 0.0429*** \\
      & (0.00834) & (0.0123) & (0.00636) & (0.0117) & (0.00832) & (0.0120) \\
      &       &       &       &       &       &  \\
Observations & 2666  & 2666  & 2656  & 2656  & 2648  & 2648 \\
R-squared & 0.00347 & 0.0399 & 0.0378 & 0.0702 & 0.0388 & 0.0711 \\
\bottomrule
\end{tabular}%
}
\end{center}
\end{table}

\subsection*{Conditioning...}

\begin{table}[H]\centering \caption{\emph{Calculadora actor}}
\begin{center}
\scriptsize{% Table generated by Excel2LaTeX from sheet 'Notification_actor'
\begin{tabular}{rrrrrrr}
\toprule
\multicolumn{1}{c}{} & \multicolumn{1}{c}{(1)} & \multicolumn{1}{c}{(2)} & \multicolumn{1}{c}{(3)} & \multicolumn{1}{c}{(4)} & \multicolumn{1}{c}{(5)} & \multicolumn{1}{c}{(6)} \\
\midrule
      &       &       &       &       &       &  \\
Calculadora\_actor & 0.0493** & 0.0324* & 0.0345*** & 0.0390 & 0.0275 & 0.0328*** \\
      & (0.0244) & (0.0191) & (0.0106) & (0.0324) & (0.0243) & (0.0113) \\
      &       &       &       &       &       &  \\
junta\_2 & 0.0164 & 0.0228 & -0.00101 & -0.00324 & -0.00825 & -0.00637 \\
      & (0.0358) & (0.0295) & (0.0129) & (0.0470) & (0.0386) & (0.0197) \\
      &       &       &       &       &       &  \\
junta\_7 & 0.0455 & 0.0763** & 0.0123 & 0.00627 & 0.0397 & 0.0117 \\
      & (0.0361) & (0.0303) & (0.0135) & (0.0487) & (0.0410) & (0.0225) \\
      &       &       &       &       &       &  \\
junta\_9 & 0.0171 & 0.0489* & 0.00674 & -0.00649 & 0.0204 & 0.00901 \\
      & (0.0342) & (0.0289) & (0.0121) & (0.0457) & (0.0385) & (0.0213) \\
      &       &       &       &       &       &  \\
junta\_11 & 0.224*** & 0.129*** & 0.140 & 0.209*** & 0.0949** & 0.260 \\
      & (0.0378) & (0.0263) & (0.0982) & (0.0512) & (0.0370) & (0.170) \\
      &       &       &       &       &       &  \\
Constant  & 0.109*** & 0.0964*** & 0.000625 & 0.140*** & 0.131*** & 0.00247 \\
      & (0.0237) & (0.0192) & (0.00770) & (0.0406) & (0.0329) & (0.0172) \\
      &       &       &       &       &       &  \\
Observations & 1085  & 1690  & 976   & 753   & 1211  & 687 \\
R-squared & 0.0523 & 0.0179 & 0.0238 & 0.0490 & 0.0117 & 0.0324 \\
Conditioning on:   & Notificacion & Notificacion & No    & Notificacion y  & Not Parcial y  & No Not Y \\
      &       & Parcial & Notificacion & Dia Treat & DT    & DT \\
\bottomrule
\end{tabular}%
}
\end{center}
\end{table}

\begin{table}[H]\centering \caption{\emph{Calculadora demandado}}
\begin{center}
\scriptsize{% Table generated by Excel2LaTeX from sheet 'Notification_dem'
\begin{tabular}{rrrrrrr}
\toprule
\multicolumn{1}{c}{} & \multicolumn{1}{c}{(1)} & \multicolumn{1}{c}{(2)} & \multicolumn{1}{c}{(3)} & \multicolumn{1}{c}{(4)} & \multicolumn{1}{c}{(5)} & \multicolumn{1}{c}{(6)} \\
\midrule
      &       &       &       &       &       &  \\
Calculadora\_dem & 0.0797*** & 0.102*** & 0.102*** & 0.0866*** & 0.120*** & 0.103*** \\
      & (0.0244) & (0.0198) & (0.0352) & (0.0289) & (0.0220) & (0.0360) \\
      &       &       &       &       &       &  \\
junta\_2 & 0.0117 & 0.00957 & -0.000347 & -0.00299 & -0.0108 & -0.0121 \\
      & (0.0341) & (0.0282) & (0.0128) & (0.0461) & (0.0376) & (0.0196) \\
      &       &       &       &       &       &  \\
junta\_7 & 0.0417 & 0.0660** & 0.0133 & 0.00455 & 0.0329 & 0.00916 \\
      & (0.0359) & (0.0301) & (0.0128) & (0.0483) & (0.0407) & (0.0211) \\
      &       &       &       &       &       &  \\
junta\_9 & 0.0200 & 0.0420 & 0.00647 & 0.00365 & 0.0255 & 0.00431 \\
      & (0.0339) & (0.0284) & (0.0125) & (0.0455) & (0.0381) & (0.0209) \\
      &       &       &       &       &       &  \\
junta\_11 & 0.222*** & 0.131*** & 0.138 & 0.210*** & 0.109*** & 0.255 \\
      & (0.0377) & (0.0258) & (0.0956) & (0.0512) & (0.0365) & (0.164) \\
      &       &       &       &       &       &  \\
Constant  & 0.0989*** & 0.0759*** & 0.00765 & 0.110*** & 0.0786** & 0.0160 \\
      & (0.0236) & (0.0188) & (0.00714) & (0.0390) & (0.0310) & (0.0150) \\
      &       &       &       &       &       &  \\
Observations & 1088  & 1692  & 964   & 756   & 1213  & 675 \\
R-squared & 0.0586 & 0.0329 & 0.0473 & 0.0577 & 0.0337 & 0.0617 \\
Conditioning on:   & Notificacion & Notificacion & No    & Notificacion y & Not Parcial y  & No Not Y \\
      &       & Parcial & Notificacion & Dia Treat & DT    & DT \\
\bottomrule
\end{tabular}%
}
\end{center}
\end{table}


\subsection*{Adding presence of employee}

\begin{table}[H]\centering \caption{Employee's presence}
\begin{center}
\scriptsize{% Table generated by Excel2LaTeX from sheet 'Presence_employee'
\begin{tabular}{rrrr}
\toprule
\multicolumn{1}{c}{} & \multicolumn{1}{c}{(1)} & \multicolumn{1}{c}{(2)} & \multicolumn{1}{c}{(3)} \\
\midrule
      &       &       &  \\
Calculadora\_actor & 0.0149 &       & -0.0144 \\
      & (0.0125) &       & (0.0125) \\
      &       &       &  \\
Calculadora\_emp &       & 0.0991*** & 0.106*** \\
      &       & (0.0163) & (0.0168) \\
      &       &       &  \\
Presence employee & 0.159*** & 0.104*** & 0.140*** \\
      & (0.0384) & (0.0243) & (0.0372) \\
      &       &       &  \\
Presence\_employee\#Calculadora\_actor & 0.0294 &       & -0.0647 \\
      & (0.0481) &       & (0.0476) \\
      &       &       &  \\
Presence\_employee\#Calculadora\_emp &       & 0.185*** & 0.208*** \\
      &       & (0.0496) & (0.0514) \\
      &       &       &  \\
Constant  & 0.0889*** & 0.0689*** & 0.0741*** \\
      & (0.00817) & (0.00629) & (0.00816) \\
      &       &       &  \\
Observations & 2666  & 2656  & 2648 \\
R-squared & 0.0409 & 0.0813 & 0.0837 \\
\bottomrule
\end{tabular}%
}
\end{center}
\end{table}

\subsection*{Instrumenting treatment with intent to treat}

...as treatment is endogenous with operation day.


\begin{table}[H]\centering \caption{IV (Second Stage). Plaintiff}
\begin{center}
\scriptsize{% Table generated by Excel2LaTeX from sheet 'IV'
\begin{tabular}{rrrrrrrr}
\toprule
      &       &       & \multicolumn{5}{c}{Plaintiff } \\
\midrule
      & (1)   & (2)   & (3)   & (4)   & (5)   & (6)   & (7) \\
      &       &       &       &       &       &       &  \\
ITT   & 0.0377*** & 0.0324*** &       &       &       &       &  \\
      & (0.0120) & (0.0123) &       &       &       &       &  \\
Calculator (instrumented with ITT) &       &       & 0.0554*** & 0.0509*** & 0.0673** & 0.0532** & 0.0357*** \\
      &       &       & (0.0176) & (0.0178) & (0.0324) & (0.0266) & (0.0124) \\
Court 7 &       & 0.0255 &       & 0.0259 & 0.0116 & 0.0357 & 0.00728 \\
      &       & (0.0168) &       & (0.0166) & (0.0343) & (0.0292) & (0.0124) \\
Court 9 &       & 0.00685 &       & 0.0110 & -0.00680 & 0.0133 & 0.00457 \\
      &       & (0.0159) &       & (0.0157) & (0.0333) & (0.0285) & (0.0116) \\
Court 11 &       & 0.142*** &       & 0.133*** & 0.178*** & 0.108*** & 0.130 \\
      &       & (0.0196) &       & (0.0191) & (0.0359) & (0.0271) & (0.0923) \\
Court 16 &       & -0.00772 &       & -0.0186 & -0.0282 & -0.0362 & -0.00154 \\
      &       & (0.0166) &       & (0.0164) & (0.0325) & (0.0276) & (0.0117) \\
Notified &       &       &       & 0.127*** & 0     & 0.0935*** & 0 \\
      &       &       &       & (0.0123) & (.)   & (0.0183) & (.) \\
Constant & 0.0924*** & 0.0594*** & 0.0914*** & 0.00727 & 0.123*** & 0.0470 & 0.000249 \\
      & (0.00961) & (0.0149) & (0.00987) & (0.0147) & (0.0300) & (0.0291) & (0.0102) \\
      &       &       &       &       &       &       &  \\
Observations & 3114  & 3114  & 3114  & 3114  & 1285  & 1991  & 1123 \\
R-squared & 0.00281 & 0.0339 & 0.00374 & 0.0718 & 0.0457 & 0.0271 & 0.0187 \\
Conditioning on:  &       &       &       &       & Notification & Partial Not & No Not  \\
\bottomrule
\end{tabular}%
}
\end{center}
\end{table}
\begin{table}[H]\centering \caption{IV (Second Stage). Defendant}
\begin{center}
\scriptsize{% Table generated by Excel2LaTeX from sheet 'IV'
\begin{tabular}{rrrrrr}
\toprule
\multicolumn{1}{c}{} & \multicolumn{5}{c}{Defendant} \\
\midrule
      & (3)   & (4)   & (5)   & (6)   & (7) \\
      &       &       &       &       &  \\
Calculator (instrumented with ITT) & 0.0973*** & 0.0875*** & 0.0766** & 0.0677** & 0.220*** \\
      & (0.0307) & (0.0306) & (0.0369) & (0.0338) & (0.0767) \\
Court 7 &       & 0.0245 & 0.0108 & 0.0356 & 0.0100 \\
      &       & (0.0163) & (0.0340) & (0.0290) & (0.0127) \\
Court 9 &       & 0.00891 & -0.00639 & 0.0119 & 0.00619 \\
      &       & (0.0154) & (0.0331) & (0.0282) & (0.0126) \\
Court 11 &       & 0.127*** & 0.173*** & 0.109*** & 0.133 \\
      &       & (0.0189) & (0.0357) & (0.0270) & (0.0855) \\
Court 16 &       & -0.0203 & -0.0321 & -0.0378 & 0.00867 \\
      &       & (0.0161) & (0.0319) & (0.0272) & (0.0131) \\
Notified &       & 0.105*** & 0     & 0.0843*** & 0 \\
      &       & (0.0149) & (.)   & (0.0188) & (.) \\
Constant & 0.0909*** & 0.0189 & 0.125*** & 0.0535* & -0.00627 \\
      & (0.0100) & (0.0123) & (0.0291) & (0.0274) & (0.0120) \\
      &       &       &       &       &  \\
Observations & 3114  & 3114  & 1285  & 1991  & 1123 \\
R-squared & 0.0347 & 0.0863 & 0.0500 & 0.0384 & . \\
Conditioning on:  &       &       & Notification & Partial Not & No Not  \\
\bottomrule
\end{tabular}%
}
\end{center}
\end{table}


\subsubsection*{First stage}


\begin{table}[H]\centering \caption{IV (First stage). Plaintiff}
\begin{center}
\scriptsize{% Table generated by Excel2LaTeX from sheet 'Firststage'
\begin{tabular}{rrrrrr}
\toprule
      & \multicolumn{5}{c}{Actora} \\
\midrule
      & \multicolumn{1}{c}{(3)} & \multicolumn{1}{c}{(4)} & \multicolumn{1}{c}{(5)} & \multicolumn{1}{c}{(6)} & \multicolumn{1}{c}{(7)} \\
\multicolumn{1}{l}{} & \multicolumn{1}{l}{} & \multicolumn{1}{l}{} & \multicolumn{1}{l}{} & \multicolumn{1}{l}{} & \multicolumn{1}{l}{} \\
\multicolumn{1}{l}{junta\_2} & \multicolumn{1}{l}{} & \multicolumn{1}{l}{0.0259} & \multicolumn{1}{l}{0.0279} & \multicolumn{1}{l}{0.0429} & \multicolumn{1}{l}{0.00500} \\
\multicolumn{1}{l}{} & \multicolumn{1}{l}{} & \multicolumn{1}{l}{(0.0253)} & \multicolumn{1}{l}{(0.0372)} & \multicolumn{1}{l}{(0.0336)} & \multicolumn{1}{l}{(0.0386)} \\
\multicolumn{1}{l}{} & \multicolumn{1}{l}{} & \multicolumn{1}{l}{} & \multicolumn{1}{l}{} & \multicolumn{1}{l}{} & \multicolumn{1}{l}{} \\
\multicolumn{1}{l}{junta\_7} & \multicolumn{1}{l}{} & \multicolumn{1}{l}{0.0146} & \multicolumn{1}{l}{0.0109} & \multicolumn{1}{l}{0.0363} & \multicolumn{1}{l}{-0.0141} \\
\multicolumn{1}{l}{} & \multicolumn{1}{l}{} & \multicolumn{1}{l}{(0.0235)} & \multicolumn{1}{l}{(0.0354)} & \multicolumn{1}{l}{(0.0309)} & \multicolumn{1}{l}{(0.0361)} \\
\multicolumn{1}{l}{} & \multicolumn{1}{l}{} & \multicolumn{1}{l}{} & \multicolumn{1}{l}{} & \multicolumn{1}{l}{} & \multicolumn{1}{l}{} \\
\multicolumn{1}{l}{junta\_9} & \multicolumn{1}{l}{} & \multicolumn{1}{l}{-0.0106} & \multicolumn{1}{l}{0.000000682} & \multicolumn{1}{l}{-0.0164} & \multicolumn{1}{l}{-0.0102} \\
\multicolumn{1}{l}{} & \multicolumn{1}{l}{} & \multicolumn{1}{l}{(0.0241)} & \multicolumn{1}{l}{(0.0366)} & \multicolumn{1}{l}{(0.0328)} & \multicolumn{1}{l}{(0.0357)} \\
\multicolumn{1}{l}{} & \multicolumn{1}{l}{} & \multicolumn{1}{l}{} & \multicolumn{1}{l}{} & \multicolumn{1}{l}{} & \multicolumn{1}{l}{} \\
\multicolumn{1}{l}{junta\_11} & \multicolumn{1}{l}{} & \multicolumn{1}{l}{0.000240} & \multicolumn{1}{l}{0.0108} & \multicolumn{1}{l}{0.0123} & \multicolumn{1}{l}{0.0165} \\
\multicolumn{1}{l}{} & \multicolumn{1}{l}{} & \multicolumn{1}{l}{(0.0231)} & \multicolumn{1}{l}{(0.0324)} & \multicolumn{1}{l}{(0.0276)} & \multicolumn{1}{l}{(0.0959)} \\
\multicolumn{1}{l}{} & \multicolumn{1}{l}{} & \multicolumn{1}{l}{} & \multicolumn{1}{l}{} & \multicolumn{1}{l}{} & \multicolumn{1}{l}{} \\
\multicolumn{1}{l}{Notificado} & \multicolumn{1}{l}{} & \multicolumn{1}{l}{0.0564***} & \multicolumn{1}{l}{0} & \multicolumn{1}{l}{0.0648***} & \multicolumn{1}{l}{0} \\
\multicolumn{1}{l}{} & \multicolumn{1}{l}{} & \multicolumn{1}{l}{(0.0153)} & \multicolumn{1}{l}{(.)} & \multicolumn{1}{l}{(0.0210)} & \multicolumn{1}{l}{(.)} \\
\multicolumn{1}{l}{} & \multicolumn{1}{l}{} & \multicolumn{1}{l}{} & \multicolumn{1}{l}{} & \multicolumn{1}{l}{} & \multicolumn{1}{l}{} \\
\multicolumn{1}{l}{dia\_tratamiento} & \multicolumn{1}{l}{0.682***} & \multicolumn{1}{l}{0.681***} & \multicolumn{1}{l}{0.727***} & \multicolumn{1}{l}{0.699***} & \multicolumn{1}{l}{0.654***} \\
\multicolumn{1}{l}{} & \multicolumn{1}{l}{(0.0112)} & \multicolumn{1}{l}{(0.0118)} & \multicolumn{1}{l}{(0.0178)} & \multicolumn{1}{l}{(0.0145)} & \multicolumn{1}{l}{(0.0205)} \\
\multicolumn{1}{l}{} & \multicolumn{1}{l}{} & \multicolumn{1}{l}{} & \multicolumn{1}{l}{} & \multicolumn{1}{l}{} & \multicolumn{1}{l}{} \\
\multicolumn{1}{l}{Constant} & \multicolumn{1}{l}{0.00911***} & \multicolumn{1}{l}{-0.0193} & \multicolumn{1}{l}{0.00103} & \multicolumn{1}{l}{-0.0490*} & \multicolumn{1}{l}{0.0158} \\
\multicolumn{1}{l}{} & \multicolumn{1}{l}{(0.00343)} & \multicolumn{1}{l}{(0.0156)} & \multicolumn{1}{l}{(0.0201)} & \multicolumn{1}{l}{(0.0251)} & \multicolumn{1}{l}{(0.0209)} \\
\multicolumn{1}{l}{} & \multicolumn{1}{l}{} & \multicolumn{1}{l}{} & \multicolumn{1}{l}{} & \multicolumn{1}{l}{} & \multicolumn{1}{l}{} \\
\multicolumn{1}{l}{Observations} & \multicolumn{1}{l}{2666} & \multicolumn{1}{l}{2666} & \multicolumn{1}{l}{1085} & \multicolumn{1}{l}{1690} & \multicolumn{1}{l}{976} \\
\multicolumn{1}{l}{R-squared} & \multicolumn{1}{l}{0.381} & \multicolumn{1}{l}{0.385} & \multicolumn{1}{l}{0.453} & \multicolumn{1}{l}{0.400} & \multicolumn{1}{l}{0.359} \\
\multicolumn{1}{l}{Conditioning on: } & \multicolumn{1}{l}{} & \multicolumn{1}{l}{} & \multicolumn{1}{l}{Notificacion } & \multicolumn{1}{l}{Not Parcial } & \multicolumn{1}{l}{No Not } \\
\bottomrule
\end{tabular}%
}
\end{center}
\end{table}
\begin{table}[H]\centering \caption{IV (First Stage). Defendant}
\begin{center}
\scriptsize{% Table generated by Excel2LaTeX from sheet 'Firststage'
\begin{tabular}{rrrrrr}
\toprule
\multicolumn{1}{c}{} & \multicolumn{5}{c}{Defendant} \\
\midrule
\multicolumn{1}{c}{} & \multicolumn{1}{c}{(3)} & \multicolumn{1}{c}{(4)} & \multicolumn{1}{c}{(5)} & \multicolumn{1}{c}{(6)} & \multicolumn{1}{c}{(7)} \\
\multicolumn{1}{l}{ITT} & \multicolumn{1}{l}{0.388***} & \multicolumn{1}{l}{0.395***} & \multicolumn{1}{l}{0.635***} & \multicolumn{1}{l}{0.554***} & \multicolumn{1}{l}{0.104***} \\
\multicolumn{1}{l}{} & \multicolumn{1}{l}{(0.0112)} & \multicolumn{1}{l}{(0.0119)} & \multicolumn{1}{l}{(0.0179)} & \multicolumn{1}{l}{(0.0148)} & \multicolumn{1}{l}{(0.0138)} \\
Court 7 & \multicolumn{1}{l}{} & \multicolumn{1}{l}{0.0201} & \multicolumn{1}{l}{0.0306} & \multicolumn{1}{l}{0.0231} & \multicolumn{1}{l}{-0.0153} \\
      & \multicolumn{1}{l}{} & \multicolumn{1}{l}{(0.0228)} & \multicolumn{1}{l}{(0.0385)} & \multicolumn{1}{l}{(0.0340)} & \multicolumn{1}{l}{(0.0252)} \\
Court 9 & \multicolumn{1}{l}{} & \multicolumn{1}{l}{0.00543} & \multicolumn{1}{l}{-0.0132} & \multicolumn{1}{l}{-0.00962} & \multicolumn{1}{l}{-0.0116} \\
      & \multicolumn{1}{l}{} & \multicolumn{1}{l}{(0.0227)} & \multicolumn{1}{l}{(0.0397)} & \multicolumn{1}{l}{(0.0349)} & \multicolumn{1}{l}{(0.0246)} \\
Court 11 & \multicolumn{1}{l}{} & \multicolumn{1}{l}{0.0660***} & \multicolumn{1}{l}{0.0701*} & \multicolumn{1}{l}{-0.0209} & \multicolumn{1}{l}{-0.0241} \\
      & \multicolumn{1}{l}{} & \multicolumn{1}{l}{(0.0229)} & \multicolumn{1}{l}{(0.0358)} & \multicolumn{1}{l}{(0.0307)} & \multicolumn{1}{l}{(0.0697)} \\
Court 16 & \multicolumn{1}{l}{} & \multicolumn{1}{l}{0.00389} & \multicolumn{1}{l}{0.0410} & \multicolumn{1}{l}{0.00432} & \multicolumn{1}{l}{-0.0510**} \\
\multicolumn{1}{l}{} & \multicolumn{1}{l}{} & \multicolumn{1}{l}{(0.0226)} & \multicolumn{1}{l}{(0.0361)} & \multicolumn{1}{l}{(0.0327)} & \multicolumn{1}{l}{(0.0230)} \\
\multicolumn{1}{l}{Notified} & \multicolumn{1}{l}{} & \multicolumn{1}{l}{0.286***} & \multicolumn{1}{l}{0} & \multicolumn{1}{l}{0.175***} & \multicolumn{1}{l}{0} \\
\multicolumn{1}{l}{} & \multicolumn{1}{l}{} & \multicolumn{1}{l}{(0.0147)} & \multicolumn{1}{l}{(.)} & \multicolumn{1}{l}{(0.0208)} & \multicolumn{1}{l}{(.)} \\
\multicolumn{1}{l}{Constant} & \multicolumn{1}{l}{0.0154***} & \multicolumn{1}{l}{-0.128***} & \multicolumn{1}{l}{-0.0150} & \multicolumn{1}{l}{-0.104***} & \multicolumn{1}{l}{0.0363**} \\
\multicolumn{1}{l}{} & \multicolumn{1}{l}{(0.00409)} & \multicolumn{1}{l}{(0.0179)} & \multicolumn{1}{l}{(0.0271)} & \multicolumn{1}{l}{(0.0305)} & \multicolumn{1}{l}{(0.0182)} \\
\multicolumn{1}{l}{} & \multicolumn{1}{l}{} & \multicolumn{1}{l}{} & \multicolumn{1}{l}{} & \multicolumn{1}{l}{} &  \\
\multicolumn{1}{l}{Observations} & \multicolumn{1}{l}{3114} & \multicolumn{1}{l}{3114} & \multicolumn{1}{l}{1285} & \multicolumn{1}{l}{1991} & \multicolumn{1}{l}{1123} \\
\multicolumn{1}{l}{R-squared} & \multicolumn{1}{l}{0.151} & \multicolumn{1}{l}{0.252} & \multicolumn{1}{l}{0.343} & \multicolumn{1}{l}{0.277} & \multicolumn{1}{l}{0.0350} \\
\multicolumn{1}{l}{Conditioning on: } & \multicolumn{1}{l}{} & \multicolumn{1}{l}{} & \multicolumn{1}{l}{Notification} & \multicolumn{1}{l}{Partial Not} & \multicolumn{1}{l}{No Not } \\
\bottomrule
\end{tabular}%
}
\end{center}
\end{table}


\subsection*{Controlling for number of cases (litigios) }



\begin{table}[H]\centering \caption{Number of cases}
\begin{center}
\scriptsize{% Table generated by Excel2LaTeX from sheet 'Num_litigios'
\begin{tabular}{rrrrrr}
\toprule
      & \multicolumn{1}{c}{(1)} & \multicolumn{1}{c}{(2)} & \multicolumn{1}{c}{(3)} & \multicolumn{1}{c}{(4)} & \multicolumn{1}{c}{(5)} \\
\midrule
      & \multicolumn{1}{l}{} & \multicolumn{1}{l}{} & \multicolumn{1}{l}{} & \multicolumn{1}{l}{} & \multicolumn{1}{l}{} \\
Calculator plaintiff & \multicolumn{1}{l}{} & \multicolumn{1}{l}{-0.0249} & \multicolumn{1}{l}{} & \multicolumn{1}{l}{-0.0649} & \multicolumn{1}{l}{-0.0544} \\
      & \multicolumn{1}{l}{} & \multicolumn{1}{l}{(0.0428)} & \multicolumn{1}{l}{} & \multicolumn{1}{l}{(0.111)} & \multicolumn{1}{l}{(0.111)} \\
Calculator defendant & \multicolumn{1}{l}{0.127***} & \multicolumn{1}{l}{} & \multicolumn{1}{l}{0.169***} & \multicolumn{1}{l}{} & \multicolumn{1}{l}{0.165***} \\
      & \multicolumn{1}{l}{(0.0425)} & \multicolumn{1}{l}{} & \multicolumn{1}{l}{(0.0568)} & \multicolumn{1}{l}{} & \multicolumn{1}{l}{(0.0570)} \\
Cases defendant: 11-30 & \multicolumn{1}{l}{0.0287} & \multicolumn{1}{l}{} & \multicolumn{1}{l}{0.107} & \multicolumn{1}{l}{0.105} & \multicolumn{1}{l}{0.105} \\
      & \multicolumn{1}{l}{(0.0846)} & \multicolumn{1}{l}{} & \multicolumn{1}{l}{(0.129)} & \multicolumn{1}{l}{(0.128)} & \multicolumn{1}{l}{(0.130)} \\
Cases defendant: 31-100 & \multicolumn{1}{l}{0.0168} & \multicolumn{1}{l}{} & \multicolumn{1}{l}{0.0149} & \multicolumn{1}{l}{0.0120} & \multicolumn{1}{l}{0.0113} \\
      & \multicolumn{1}{l}{(0.0707)} & \multicolumn{1}{l}{} & \multicolumn{1}{l}{(0.102)} & \multicolumn{1}{l}{(0.103)} & \multicolumn{1}{l}{(0.103)} \\
Cases defendant: +100 & \multicolumn{1}{l}{-0.0527} & \multicolumn{1}{l}{} & \multicolumn{1}{l}{-0.0480} & \multicolumn{1}{l}{-0.0590} & \multicolumn{1}{l}{-0.0520} \\
      & \multicolumn{1}{l}{(0.0620)} & \multicolumn{1}{l}{} & \multicolumn{1}{l}{(0.0904)} & \multicolumn{1}{l}{(0.0905)} & \multicolumn{1}{l}{(0.0905)} \\
Cases plaintiff 11-30 & \multicolumn{1}{l}{} & \multicolumn{1}{l}{-0.0189} & \multicolumn{1}{l}{-0.0157} & \multicolumn{1}{l}{-0.0180} & \multicolumn{1}{l}{-0.0118} \\
      & \multicolumn{1}{l}{} & \multicolumn{1}{l}{(0.0433)} & \multicolumn{1}{l}{(0.106)} & \multicolumn{1}{l}{(0.108)} & \multicolumn{1}{l}{(0.107)} \\
Cases plaintiff 31-100 & \multicolumn{1}{l}{} & \multicolumn{1}{l}{-0.0174} & \multicolumn{1}{l}{-0.0851} & \multicolumn{1}{l}{-0.0798} & \multicolumn{1}{l}{-0.0848} \\
      & \multicolumn{1}{l}{} & \multicolumn{1}{l}{(0.0385)} & \multicolumn{1}{l}{(0.0912)} & \multicolumn{1}{l}{(0.0919)} & \multicolumn{1}{l}{(0.0919)} \\
Cases plaintiff +100 & \multicolumn{1}{l}{} & \multicolumn{1}{l}{0.00209} & \multicolumn{1}{l}{-0.0502} & \multicolumn{1}{l}{-0.0456} & \multicolumn{1}{l}{-0.0469} \\
      & \multicolumn{1}{l}{} & \multicolumn{1}{l}{(0.0365)} & \multicolumn{1}{l}{(0.0860)} & \multicolumn{1}{l}{(0.0881)} & \multicolumn{1}{l}{(0.0875)} \\
Constant & \multicolumn{1}{l}{0.108} & \multicolumn{1}{l}{0.154***} & \multicolumn{1}{l}{0.126} & \multicolumn{1}{l}{0.350**} & \multicolumn{1}{l}{0.182} \\
      & \multicolumn{1}{l}{(0.0686)} & \multicolumn{1}{l}{(0.0500)} & \multicolumn{1}{l}{(0.127)} & \multicolumn{1}{l}{(0.156)} & \multicolumn{1}{l}{(0.164)} \\
      & \multicolumn{1}{l}{} & \multicolumn{1}{l}{} & \multicolumn{1}{l}{} & \multicolumn{1}{l}{} & \multicolumn{1}{l}{} \\
Observations & \multicolumn{1}{l}{596} & \multicolumn{1}{l}{978} & \multicolumn{1}{l}{310} & \multicolumn{1}{l}{310} & \multicolumn{1}{l}{310} \\
R-squared & \multicolumn{1}{l}{0.0153} & \multicolumn{1}{l}{0.00127} & \multicolumn{1}{l}{0.0271} & \multicolumn{1}{l}{0.0190} & \multicolumn{1}{l}{0.0281} \\
\bottomrule
\end{tabular}%
}
\end{center}
\end{table}


\pagebreak


\subsection*{Update in beliefs}



\begin{table}[H]\centering \caption{Update in probability}
\begin{center}
\scriptsize{% Table generated by Excel2LaTeX from sheet 'Update_beleifs_prob'
\begin{tabular}{rrrr}
\toprule
      & \multicolumn{3}{c}{Settlement} \\
\midrule
      & \multicolumn{1}{c}{(Employee)} & \multicolumn{1}{c}{(Employee's Lawyer)} & \multicolumn{1}{c}{(Firm's Lawyer)} \\
      & \multicolumn{1}{c}{} & \multicolumn{1}{c}{} & \multicolumn{1}{c}{} \\
Update prob & -0.0505 & 0.00193 & 0.0156 \\
      & (0.128) & (0.0576) & (0.106) \\
Constant  & 0.179*** & 0.125*** & 0.219*** \\
      & (0.0351) & (0.0151) & (0.0209) \\
      &       &       &  \\
Observations & 151   & 672   & 403 \\
R-squared & 0.00104 & 0.00000195 & 0.0000645 \\
IndVarMean & -0.130 & -0.139 & -0.0279 \\
\bottomrule
\end{tabular}%
}
\end{center}
\end{table}


\begin{table}[H]\centering \caption{Update in payment}
\begin{center}
\scriptsize{% Table generated by Excel2LaTeX from sheet 'Update_beleifs_pago'
\begin{tabular}{rrrr}
\toprule
      & \multicolumn{3}{c}{Settlement} \\
\midrule
      & \multicolumn{1}{c}{(Employee)} & \multicolumn{1}{c}{(Employee's Lawyer)} & \multicolumn{1}{c}{(Firm's Lawyer)} \\
      & \multicolumn{1}{c}{} & \multicolumn{1}{c}{} & \multicolumn{1}{c}{} \\
Update payment & 0.192 & -0.109 & 0.0508 \\
      & (0.147) & (0.0821) & (0.0751) \\
Constant  & 0.224*** & 0.121*** & 0.218*** \\
      & (0.0395) & (0.0169) & (0.0264) \\
      &       &       &  \\
Observations & 121   & 482   & 278 \\
R-squared & 0.00729 & 0.00556 & 0.00147 \\
IndVarMean & -0.0467 & -0.105 & -0.110 \\
\bottomrule
\end{tabular}%
}
\end{center}
\end{table}


\subsection*{Which conciliator is better? }


\begin{table}[H]\centering \caption{Dummies by each conciliator}
\begin{center}
\scriptsize{% Table generated by Excel2LaTeX from sheet 'Conciliator'
\begin{tabular}{rrrrr}
\toprule
      & \multicolumn{2}{c}{Demandada} & \multicolumn{2}{c}{Actora} \\
\midrule
      & \multicolumn{1}{c}{(1)} & \multicolumn{1}{c}{(2)} & \multicolumn{1}{c}{(1)} & \multicolumn{1}{c}{(2)} \\
      &       &       &       &  \\
Calculator & 0.150*** & 0.182*** & 0.0393** & 0.0447 \\
      & (0.0168) & (0.0450) & (0.0161) & (0.0402) \\
ANA   & 0.0140 & 0.0330 & 0.0143 & 0.0716 \\
      & (0.0364) & (0.0385) & (0.0367) & (0.0791) \\
LUCIA & -0.00626 & -0.0195 & 0.00408 & -0.0139 \\
      & (0.0366) & (0.0225) & (0.0381) & (0.0676) \\
JACQUIE & 0.0336 & -0.00277 & 0.0214 & 0.0138 \\
      & (0.0379) & (0.0244) & (0.0403) & (0.0727) \\
MARINA & -0.0737** & -0.0426* & -0.0801** & -0.0613 \\
      & (0.0333) & (0.0242) & (0.0348) & (0.0665) \\
KARINA & 0.0957 & 0.0967 & 0.0967 & -0.0500 \\
      & (0.0728) & (0.0976) & (0.0727) & (0.120) \\
MARIBEL & 0.0470 & -0.0568 & 0.0594 & -0.0680 \\
      & (0.0498) & (0.0590) & (0.0510) & (0.0889) \\
DEYANIRA & 0.136** & 0.105 & 0.124* & 0.0894 \\
      & (0.0638) & (0.0804) & (0.0643) & (0.106) \\
GUSTAVO & 0     & 0     & 0     & 0 \\
      & (.)   & (.)   & (.)   & (.) \\
CORRAL & 0.0381 & 0.0746 & 0.0503 & 0.0818 \\
      & (0.0462) & (0.0599) & (0.0464) & (0.0845) \\
AGUSTIN & -0.0773** & -0.0621*** & -0.0803** & -0.0391 \\
      & (0.0353) & (0.0186) & (0.0367) & (0.0686) \\
MARGARITA & 0.0813** & 0.0747*** & 0.0870** & 0.0584 \\
      & (0.0353) & (0.0201) & (0.0369) & (0.0701) \\
LUPITA & -0.0177 & 0.00850 & -0.0223 & -0.0521 \\
      & (0.0275) & (0.0288) & (0.0279) & (0.0462) \\
ISAAC & 0.00397 & -0.0250 & 0.00491 & 0.0163 \\
      & (0.0303) & (0.0267) & (0.0314) & (0.0533) \\
HIGUERA & -0.0111 & 0.0180 & -0.0124 & 0.0349 \\
      & (0.0252) & (0.0249) & (0.0256) & (0.0453) \\
DOCTOR & -0.0197 & 0.00412 & -0.0133 & 0.00231 \\
      & (0.0371) & (0.0322) & (0.0381) & (0.0552) \\
CESAR & -0.0406 & -0.0395 & -0.0504 & -0.0757 \\
      & (0.0376) & (0.0298) & (0.0389) & (0.0491) \\
Constant  & 0.0746*** & 0.0621*** & 0.107*** & 0.102*** \\
      & (0.0195) & (0.0186) & (0.0220) & (0.0313) \\
      &       &       &       &  \\
Interaction Term & NO    & YES   & NO    & YES \\
Observations & 1888  & 1888  & 1898  & 1898 \\
R-squared & 0.0824 & 0.0918 & 0.0392 & 0.0444 \\
\bottomrule
\end{tabular}%
}
\end{center}
\end{table}

\begin{table}[H]\centering \caption{Interaction term}
\begin{center}
\scriptsize{% Table generated by Excel2LaTeX from sheet 'Conciliator'
\begin{tabular}{rrr}
\toprule
\multicolumn{1}{c}{Interaction Terms} & \multicolumn{1}{c}{Defendant} & \multicolumn{1}{c}{Plaintiff } \\
\midrule
      &       &  \\
Calculator\#ANA & -0.0313 & -0.121 \\
      & (0.0751) & (0.0854) \\
Calculator\#LUCIA & 0.0569 & -0.0453 \\
      & (0.0815) & (0.0830) \\
Calculator\#JACQUIE & 0.0802 & -0.0601 \\
      & (0.0782) & (0.0877) \\
Calculator\#MARINA & -0.0856 & 0.0477 \\
      & (0.0686) & (0.0769) \\
Calculator\#KARINA & -0.0344 & 0.164 \\
      & (0.133) & (0.134) \\
Calculator\#MARIBEL & 0.161* & 0.142 \\
      & (0.0853) & (0.0938) \\
Calculator\#DEYANIRA & 0.0703 & 0.00278 \\
      & (0.104) & (0.108) \\
Calculator\#CORRAL & -0.0316 & -0.00780 \\
      & (0.0720) & (0.0763) \\
Calculator\#AGUSTIN & -0.0246 & -0.0422 \\
      & (0.0724) & (0.0649) \\
Calculator\#MARGARITA & 0.00763 & 0.0275 \\
      & (0.0728) & (0.0669) \\
Calculator\#LUPITA & -0.0852 & 0.0349 \\
      & (0.0614) & (0.0574) \\
Calculator\#ISAAC & 0.0909 & -0.0145 \\
      & (0.0694) & (0.0649) \\
Calculator\#HIGUERA & -0.0624 & -0.0561 \\
      & (0.0528) & (0.0524) \\
Calculator\#DOCTOR & -0.0601 & 0.00915 \\
      & (0.0651) & (0.0576) \\
Calculator\#CESAR & 0.00838 & 0.0142 \\
      & (0.0739) & (0.0594) \\
\bottomrule
\end{tabular}%
}
\end{center}
\end{table}

\subsection*{Some last controls }


\begin{table}[H]\centering \caption{Covariates employee survey}
\begin{center}
\scriptsize{% Table generated by Excel2LaTeX from sheet 'SS_control_survey'
\begin{tabular}{rrrrrr}
\toprule
\multicolumn{1}{c}{Variable } & \multicolumn{1}{c}{Obs} & \multicolumn{1}{c}{Mean} & \multicolumn{1}{c}{Std. Dev.} & \multicolumn{1}{c}{Min} & \multicolumn{1}{c}{Max} \\
\midrule
Enojo &       &       &       &       &  \\
Mediano & 213   & 0.1737089 & 0.3797515 & 0     & 1 \\
Poco  & 213   & 0.0704225 & 0.2564605 & 0     & 1 \\
Nada  & 213   & 0.1596244 & 0.3671202 & 0     & 1 \\
      &       &       &       &       &  \\
      &       &       &       &       &  \\
Compra bienes & 207   & 0.0628019 & 0.2431945 & 0     & 1 \\
      &       &       &       &       &  \\
Estudios &       &       &       &       &  \\
Secundaria & 213   & 0.2723005 & 0.4461923 & 0     & 1 \\
Preparatoria & 213   & 0.286385 & 0.4531364 & 0     & 1 \\
Mas que prepa & 213   & 0.370892 & 0.4841815 & 0     & 1 \\
      &       &       &       &       &  \\
Trabaja & 206   & 0.4223301 & 0.4951338 & 0     & 1 \\
Busca trabajo & 201   & 0.5472637 & 0.499004 & 0     & 1 \\
\bottomrule
\end{tabular}%
}
\end{center}
\end{table}

\begin{table}[H]\centering \caption{Controlling for some answers in employee survey}
\begin{center}
\scriptsize{% Table generated by Excel2LaTeX from sheet 'Answers_employee'
\begin{tabular}{rrrr}
\toprule
      & \multicolumn{1}{c}{(1)} & \multicolumn{1}{c}{(2)} & \multicolumn{1}{c}{(3)} \\
\midrule
Calculator plaintiff & -0.0284 &       & -0.0599 \\
      & (0.118) &       & (0.112) \\
Calculator defendant &       & 0.173*** & 0.175*** \\
      &       & (0.0583) & (0.0584) \\
Anger: fairly & 0.0667 & 0.0910 & 0.0915 \\
      & (0.0726) & (0.0715) & (0.0718) \\
Anger: little & 0.134 & 0.128 & 0.126 \\
      & (0.119) & (0.115) & (0.112) \\
Anger: nothing & 0.201** & 0.218** & 0.215** \\
      & (0.0934) & (0.0876) & (0.0880) \\
Buys durable goods & -0.109 & -0.130 & -0.136 \\
      & (0.0949) & (0.0982) & (0.101) \\
Secondary & 0.150*** & 0.123** & 0.123** \\
      & (0.0522) & (0.0538) & (0.0543) \\
High School & 0.236*** & 0.209*** & 0.212*** \\
      & (0.0641) & (0.0642) & (0.0656) \\
+ High School & 0.152*** & 0.116** & 0.118** \\
      & (0.0508) & (0.0535) & (0.0547) \\
Works at the time & 0.0391 & 0.0329 & 0.0335 \\
      & (0.0738) & (0.0709) & (0.0712) \\
Looking for a job & 0.000235 & 0.00360 & 0.00415 \\
      & (0.0732) & (0.0690) & (0.0693) \\
Constant  & -0.0185 & -0.0826 & -0.0275 \\
      & (0.125) & (0.0685) & (0.118) \\
      &       &       &  \\
Observations & 221   & 221   & 221 \\
R-squared & 0.0667 & 0.112 & 0.113 \\
\bottomrule
\end{tabular}%
}
\end{center}
\end{table}


\vspace{5mm}

\begin{center}
\scriptsize{
\begin{table}[htbp]\centering \caption{Covariates iniciales \label{sumstat}}
\begin{tabular}{l c c c c c}\hline\hline
\multicolumn{1}{c}{\textbf{Variable}} & \textbf{Mean}
 & \textbf{Std. Dev.}& \textbf{Min.} &  \textbf{Max.} & \textbf{N}\\ \hline
gen & 0.446 & 0.497 & 0 & 1 & 2108\\
c\_antiguedad & 4.049 & 5.462 & 0.003 & 44.822 & 2108\\
c\_indem & 55236.205 & 79282.709 & 0 & 1364212.625 & 2108\\
reinst & 0.49 & 0.5 & 0 & 1 & 2108\\
tipo\_abogado\_ac & 1.144 & 0.516 & 1 & 3 & 2108\\
\hline\end{tabular}
\end{table}
}
\end{center}

\begin{table}[H]\centering \caption{Controlling for some initial variables}
\begin{center}
\scriptsize{% Table generated by Excel2LaTeX from sheet 'Control_iniciales'
\begin{tabular}{rrrr}
\toprule
\multicolumn{1}{c}{} & \multicolumn{1}{c}{(1)} & \multicolumn{1}{c}{(2)} & \multicolumn{1}{c}{(3)} \\
\midrule
Calculator plaintiff & 0.0365** &       & 0.0196 \\
      & (0.0152) &       & (0.0149) \\
Calculator defendant &       & 0.154*** & 0.152*** \\
      &       & (0.0158) & (0.0158) \\
Female & 0.0161 & 0.0105 & 0.0109 \\
      & (0.0149) & (0.0145) & (0.0145) \\
Tenure & -0.00272*** & -0.00328*** & -0.00324*** \\
      & (0.000943) & (0.000945) & (0.000946) \\
Severance pay & -0.000000203*** & -0.000000197*** & -0.000000190** \\
      & (7.24e-08) & (7.42e-08) & (7.46e-08) \\
Reinstatement & -0.0350** & -0.0325** & -0.0335** \\
      & (0.0153) & (0.0149) & (0.0150) \\
Public lawyer & 0.000722 & 0.00648 & 0.00572 \\
      & (0.0158) & (0.0151) & (0.0151) \\
Constant  & 0.136*** & 0.0958*** & 0.0834*** \\
      & (0.0267) & (0.0237) & (0.0259) \\
      &       &       &  \\
Observations & 2108  & 2108  & 2108 \\
R-squared & 0.0120 & 0.0597 & 0.0604 \\
\bottomrule
\end{tabular}%
}
\end{center}
\end{table}

%%%%%%%%%%%%%%%%%%%%%%%%%%%%%%%%%%%%%%%%%%%%%%%%%%%%%%%%%%%%%%%%%%%%%%%%%%%%%%%%
\end{document}
