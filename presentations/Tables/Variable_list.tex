% Table generated by Excel2LaTeX from sheet 'Variable_list'
\begin{tabular}{p{2cm}p{12cm}}
\toprule
\multicolumn{1}{l}{\textbf{VARIABLE}} & \multicolumn{1}{l}{\textbf{DESCRIPTION}} \\
\midrule
\midrule
\textbf{Public Lawyer*} & If the lawyer is public or private. \\
\textbf{Female*} & Dummy for female plaintiff. \\
\textbf{At will worker*} & If the worker is an at will employee. \\
\textbf{Tenure*} & Employee's tenure with the employer. \\
\textbf{Daily wage*} & Daily salary. \\
\textbf{Weekly hours*} & Number of hours that the plaintiff worked on a weekly basis. \\
\textbf{Reinstatement*} & If the plaintiff claims reinstatement. \\
\textbf{Severance pay*} & If the plaintiff claims constitutional indemnity (three months of integrated salary) that the law dictates for unjustified dismissal. \\
\textbf{Lost wages*} & If the plaintiff claims lost wages/back pay. \\
\textbf{Tenure bonus} & If the plaintiff claims the payment of "tenure bonus" (12 days per year worked, up to two times minimum wage). \\
\textbf{End of year bonus} & If the plaintiff claims end of year bonus. \\
\textbf{Vacation pay} & If the plaintiff claims accrued vacation days not taken. \\
\textbf{Overtime*} & If the plaintiff claims overtime pay. \\
\textbf{Twenty days x year*} & If the plaintiff claims the payment of compensation (20 days per year worked) that the law dictates for unjustified dismissal for a worker who has the right to be reinstated but the employer refuses to reinstate, or for an at-will employee who cannot ask for reinstatement. \\
\textbf{Sunday bonus} & If the plaintiff claims the payment of bonus for working on Sundays (the law provides for a 25\% premium for working Sundays). \\
\textbf{Weekly rest} & If the plaintiff claims the payment of weekly rest day (3 times daily wage). \\
\textbf{Compulsory rest} & If the plaintiff claims the payment of legal holidays worked (3 times daily wage). \\
\textbf{Profit sharing} & If the plaintiff claims the proportional payment of profits generated by the company that correspond under the labor law to employees. \\
\textbf{SARIMSS*} & If the plaintiff claims the payment of employer contributions that were not made to these institutions, or retroactive registration in the institutions (SAR: retirement savings, IMSS: social security, INFONAVIT: worker's housing fund). \\
\textbf{Nullification of documents} & If the plaintiff requests the nullification of documents, including for example a signed letter of resignation that the worker was made to sign as a blank page, upon being hired, and which later was filled out by the employer with the text of a voluntary letter of resignation. \\
\textbf{Co-defendant*} & If at least one of the codefendants is the IMSS or INFONAVIT or SAR. \\
\textbf{Total asked} & The total quantifiable peso amount of the worker's claim. \\
\textbf{Industry*} & Defendants industry coded as the North American Industry Classification System (NAICS), which provides a consistent framework to classify companies and business establishments to allow for a high level of comparability in business statistics among the North American countries. \\
\bottomrule
\bottomrule
\end{tabular}%
